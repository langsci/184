% add all extra packages you need to load to this file 
\usepackage{graphicx}
\usepackage{tabularx}
\usepackage{amsmath} 
\usepackage{multicol} 
%%%%%%%%%%%%%%%%%%%%%%%%%%%%%%%%%%%%%%%%%%%%%%%%%%%%
%%%                                              %%%
%%%           Examples                           %%%
%%%                                              %%%
%%%%%%%%%%%%%%%%%%%%%%%%%%%%%%%%%%%%%%%%%%%%%%%%%%%%
% remove the percentage signs in the following lines
% if your book makes use of linguistic examples
\usepackage{langsci/styles/langsci-optional} 
\usepackage{langsci/styles/langsci-lgr}
\usepackage{morewrites} 
%% if you want the source line of examples to be in italics, uncomment the following line
% \def\exfont{\it}


\usepackage{pgfplots}


%ch 04
\usepackage{verbatim}

\usepackage{tikz-dependency}
\usetikzlibrary{positioning} 
%ch 06
\usepackage{enumitem}

%\usepackage{adjustbox}

%ch 10
\usepackage{color}
\usepackage{multirow}
%\usepackage{multicol}
\usepackage{rotating}
%ch10
% For trees, uncomment the following lines
%\usepackage{qtree}
\usepackage{tikz-qtree}
% % has strange side effects
 \tikzset{every tree node/.style={align=center, anchor=north}}
% \tikzset{every roof node/.append style={inner sep=0.1pt,text height=2ex,text depth=0.3ex}}


\usepackage{multirow}

\usepackage{langsci/styles/langsci-gb4e}  
\usepackage[linguistics,edges]{forest} 
\forestset{
  pretty nice empty nodes/.style={
    for tree={
      calign=fixed edge angles,
      parent anchor=south,
      delay={if content={}{
          inner sep=0pt,
          edge path={\noexpand\path [\forestoption{edge}] (!u.parent anchor) -- (.south)\forestoption{edge label};}
        }{}}
    }
  },  
  fairly nice empty nodes/.style={
	      delay={where content={}{shape=coordinate,for parent={
		    for children={anchor=north}}}{}}
  }
}


 
 
 \usetikzlibrary{backgrounds}
\tikzset{
  basic/.style = {line width=1pt,draw=black},
  L0/.style = {align=left,  fill=green!30},
  L1/.style = {align=left, fill=green!20,},
  L2/.style = {align=left, fill=pink!60, },
  L3/.style = {align=left, fill=pink!10},
  t0/.style={text width=2cm,},
  t1/.style={text width=6cm,},
  t2/.style={text width=9em},
  t3/.style={text width=5em},
}