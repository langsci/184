\documentclass[output=paper]{langsci/langscibook}
\author{Aggeliki Fotopoulou\affiliation{Institute for Language and Speech Processing, Athena RIC, Greece}%
\lastand Paraskevi Giouli \affiliation{Institute for Language and Speech Processing, Athena RIC, Greece}}
\title{MWEs and the Emotion Lexicon: typological and cross-lingual considerations}\shorttitlerunninghead{MWEs and the Emotion Lexicon}

\abstract{The work presented in this paper is aimed at studying predicates that pertain to
the semantic field of emotions, the focus being on Modern \ili{Greek} verbal
Multiword Expressions (MWEs) and their counterparts in \ili{French}. A core
lexicon of verbal MWEs denoting emotion was extracted from existing
Modern \ili{Greek} lexical resources; the initial list was further extended
and revised manually in view of corpus evidence. A classification of MWEs is proposed based on syntactic, selectional and semantic
properties; an attempt to map the expressions identified onto their
\ili{French} counterparts was also made. The cross-linguistic study reveals
similarities and discrepancies in the two languages, and highlights the
interaction between MWEs structure and their underlying semantics, in
that the intensity of the emotion denoted and the degree of fixedness
of the relevant expressions seem to be highly correlated in both
languages.}

\maketitle

\begin{document}

\section{Introduction}


The availability of user-generated content over the web and the
increasing need to make the most out of it has brought about a shift of
interest from factual information to the identification of subjective
information (as opposed to facts) expressed by people or groups of
people with respect to a specific topic. To this end, the task of
determining the so-called \textit{private states} (that is,
beliefs, feelings, and speculations) expressed in running text and the
entities involved has been the focus of attention in the field of
\isi{Natural Language Processing (NLP)}. Therefore, identification of
expressions denoting emotion or emotional state in textual data and
their classification is of paramount importance. In this respect, MWEs
can hardly be overlooked since they constitute a significant proportion
of the emotion lexicon.

We hereby present work aimed at treating \is{verbal MWEs} verbal multi-word predicates
that pertain to the \textit{semantic field of emotions} from a
cross-lingual perspective and systematising their lexical, syntactic
and semantic properties. In this context, verbal MWEs in Modern \ili{Greek} denoting emotion or emotional state were selected from existing
language resources; their lexico-semantic properties were also
retrieved from these resources; new entries were encoded following the
same principles. All MWEs were further assigned \textit{semantic
features }inherent to the semantic field. At the next stage, their
mapping onto their counterparts in \ili{French} was performed. The
comparative study of \ili{Greek}\is{Greek} and \ili{French} MWEs resulted in the identification of
cross-lingual similarities and discrepancies. Moreover, correlations
between lexical features and the underlying semantics of MWEs were also
revealed. Our working hypothesis was that despite idiosyncrasies, MWEs
that belong to a given semantic class share features that are
characteristic for this class; moreover, these field-specific features
are attested cross-linguistically. One step further, the
(cross-lingual) treatment of MWEs might be useful not only from a
purely linguistic point of view but also for NLP applications.


The paper is outlined as follows. An overview of background work on the
study of the emotion lexicon and of MWEs is presented in §2,
 §3 outlines the methodological framework adopted,
whereas the selection process of the lexical data is described in
§4. The lexicon of emotion MWEs and the syntactic, selectional
and semantic properties encoded are presented in §5; we discuss
our findings in §6 and elaborate further on cross-lingual
considerations in §7. Finally, our conclusions and prospects for
future research are outlined in §8.





\section{Background work}



The seminal work at the syntax-semantics interface by \citet{levin1993} involves large-scale
classification of English verbal predicates on the basis
of shared meaning and syntactic properties; in this work, more than
3000 verbs have been grouped into semantically coherent verb classes each
depicting a syntactic configuration that reflects verb meaning. A more
fine-grained semantic classification of \ili{French} verb and noun predicates
denoting feeling, emotion and psychological states has also been
performed (Mathieu 1999; Mathieu 2006), aimed at a wide range of NLP
applications. \ili{French} nominal and verbal
predicates denoting emotion and their lexicalised word combinations
have been studied (\citealt{leeman1991}; \citealt{gross1995locale}; \citealt{balibar1995}; \citealt{tutin2006})  from a different point of view. Finally, a comparative analysis of English and \ili{French}
single-word verbal predicates denoting emotion \citep{mathieu2010} reports on properties shared among the two languages on the
grounds of syntax and semantics, unveiling at the same time the
idiosyncrasies of each language. 



As far as MWEs are concerned, the systematic treatment of \ili{French} fixed
expressions has been carried out \citep{gross1982}.  In this work, the
classification and the analysis of c.\ 20000 \ili{French} verbal MWEs consists in
the formal representation of their syntactic properties, selectional
restrictions and the distinction between fixed and non-fixed
constituents. Along the same lines, the classification of \ili{Greek} fixed
expressions (c. 6000 entries) has been performed on the basis of the
same formal principles and criteria (\citealt{fotopoulou1993}; \citealt{Mini2009}).



The present study is part of a larger effort aimed at developing lexical
resources that encompass the \ili{Greek} emotion lexicon, i.e., words and
phrases that refer to emotional states and emotion-related mental
events. Previous work involves treatment of nouns and verbs. In this
context, 130 \ili{Greek} noun predicates denoting emotion (\textit{Nsent}) were
identified and classified on the basis of the verbs' syntactic, semantic and
distributional properties (\citealt{pantazara2008}; \citealt{fotopoulou2009}). In this context, support verbs (\textit{Vsup}) and other verbs
expressing diverse modalities (aspect, intensity, control, etc.) were
identified and encoded as properties; these properties reveal the
restrictions nouns impose on the lexical choice of verbs. Similarly,
339 \ili{Greek} verbal predicates denoting emotion (\textit{Vsent}) were
classified into homogenous syntactico-semantic classes based on their
syntactic, lexical and semantic properties (Giouli \& Fotopoulou 2012);
a number of syntactic features (i.e., argument structure,
alternations), selectional restrictions imposed on the verbs' subject and
object complements, emotion type, polarity and intensity were also
defined and encoded formally. 


In this respect, this work is further aimed at enriching the set
of lexical resources pertaining to the semantic field of emotions
with a lexicon that comprises verbal MWEs denoting emotion or emotional
state. Moreover, the \ili{Greek} MWEs were mapped onto their \ili{French} counterparts.
The ultimate goal was not only to develop a bi-lingual lexical
resource, but also to test the hypothesis that, despite the
idiosyncrasies that are inherent to MWEs in general, a certain degree
of regularity (in terms of inherent properties) can be observed within
a semantic class. To this end, we opted for reusing and extending
existing lexical resources that encompass verb MWEs in \ili{Greek} and \ili{French}.

\section{Methodological framework}



The resources that form the basis of the present study have been
developed using the methodological framework Lexicon-Grammar (LG)
\citep{gross1975methodes}. Being a model of syntax limited to the elementary
sentences of a natural language, the theory argues that the unit of
meaning is not located at the level of the word, but at the level of
sentence of the form \textit{Subject} –
\textit{Verb} – \textit{Object}. Therefore, the
elementary sentence is  transformed to its predicate-argument
structure, and the main complements (subject, one or more objects) are
separated from other complements (adjuncts) on the basis of formal
criteria. Distributional properties associated with words, i.e., types
of prepositions, semantic features inherent to nouns in subject and
object positions, etc. are also taken into account, resulting in a more
fine-grained classification and in the creation of homogenous word
classes. Finally, transformation rules, construed as equivalence
relations between sentences,  generate additional equivalent structures.
All this information (argument structure, distributional properties and
permitted transformational rules) is formally encoded in the so-called
\textit{LG tables}.



Each table is defined by a set of distinct properties (syntactic,
distributional, and semantic) and includes all the lexical items sharing
these properties. Predicates with more than one usage or meaning are
treated as separate lexical items possibly represented in different
tables, and the syntactic and semantic properties are assigned to each
entry as appropriate. In this sense, entries in one table are considered to form a homogeneous class. In an LG table, the set of properties
that describe the entries are encoded as headers of the
columns, whereas entries are listed at separate rows. At
the intersection of a row corresponding to a lexical item (entry) and a
column corresponding to a property, the cell is set to `+' if the
property is valid for the given entry or `-' if it is not. 



Similarly, MWEs are also treated as \textit{elementary
sentences} for which all possible fixed and non-fixed (or variable)
arguments (if any) are consistently and uniformly encoded. The
formalism provides the mechanism for encoding properties that are
appropriate for the identification and processing of MWEs. More
precisely, the MWE structure is represented as a Part-of-Speech sequence.
According to the LG notation, \textit{Ν }denotes a non-fixed nominal,
whereas, \textit{C} signifies a fixed one; numbers are used to
represent the syntactic function of fixed or non-fixed constituents. In
this sense, \textit{N0} is used to represent a non-fixed noun in
subject position whereas, \textit{C0} denotes a fixed subject.
Similarly, \textit{N1}, \textit{N2}, \textit{N3}, etc., along with
\textit{C1}, \textit{C2}, \textit{C3} etc. denote complements in object
position (or complements of prepositional phrases), marked also for
fixedness. It should be noted, however, that the internal structure of
the noun phrase is not represented explicitly in general; patterns
depict the elementary sentence or structure characterising each MWE class,
whereas information regarding modifiers, determiners, etc. allowed for
by certain expressions is provided in the form of
\textit{features} or \textit{properties}.
\textit{Selectional restrictions} over the non-fixed or
variable elements of MWEs as well as syntactic phenomena (e.g., passive
alternation, etc.) - if any - are also encoded formally. Finally, other
grammatical phenomena such as agreement features are accounted for. 



For example, the MWE in (\ref{ex:3:1}) below comprises two fixed (or lexicalised)
elements, a verb and a noun in subject position, and two variable
elements, namely a nominal phrase in accusative and a possessive
pronoun (\textit{Poss}) that modifies the fixed nominal constituent. The variant
nominal phrase is most often realised as a weak personal pronoun in
pre-verbal position (\textit{Ppv}); agreement in number and person between the two variable
elements is mandatory:

\begin{exe}
\ex \label{ex:3:1}
my devils catch me `to become very angry'\\
\glll με πιάνουν τα διαόλια μου / *σου / *του Γιάννη \\
me pianun ta diaolia mu / *su /  *tu Jani\\
me catch.\textsc{3pl} the devils.\textsc{nom.pl.poss} my / your / the John.\textsc{gen}\\
\glt %lit. 'my / your / John's devils catch me'
`to become very angry'
\end{exe}

In this case, a generic syntactic pattern like the one depicted in (\ref{ex:3:2a})
below is used to describe a class in a LG table.

\begin{exe}
\ex
\begin{xlist} \label{ex:3:2a}
\ex Ppv V C0 Poss
\ex \label{ex:3:2b} Ppv-1 V C0 Poss-1
\end{xlist}
\end{exe}


The agreement attested between variable elements is then
depicted via co-indexing as shown in (\ref{ex:3:2b}).



An example of MWE representation within the LG framework is illustrated
in Table \ref{fig:03:01} below; the table comprises \isi{verbal MWEs} with the underlying
structure \textit{N0 V Prep C1} \citep{fotopoulou1993}.

%\begin{figure}[H]
%\includegraphics[width=\textwidth]{figures/figfotop1.jpg}
%\caption{LG table of verbal MWEs (sample).}
%\label{fig:03:01:off}
%\end{figure}

\begin{table}[H]
\small
\begin{tabular}{|c|c|c|c|c|c|c|c|c|c|c|}
\hline
\rotatebox[origin=c]{90}{N0 =: +Hum} & \rotatebox[origin=c]{90}{N0 =: -Hum} &  \multicolumn{5}{|c|}{$\langle$ E $\rangle$} & \rotatebox[origin=c]{90}{N1 =: Npc} & \rotatebox[origin=c]{90}{N0V} & \rotatebox[origin=c]{90}{ N0Vamt N1 Prep C2 } & \rotatebox[origin=c]{90}{PhraseVsup} \\
\hline
- & +     & ακτινοβολώ  & από & E & ευτυχία & E      & - & - & + & - \\
\hline
+ & -     & αφρίζω      & από & (E+τη) & λύσσα & (E+Poss-0)   & - & - & - & - \\
\hline
+ & -     & βράζω       & σε  & το    & ζουμί & Poss-0    & - & - & - & - \\
\hline
+ & -     & γελάω       & με  & την    & καρδιά & Poss-0  & + & + & - & - \\
\hline
+ & -     & έρχομαι     & σε  & τα     & λογικά & Poss-0  & - & - & + & + \\
\hline
+ & -     & έρχομαι     & σε  & τα     & συγκαλά & Poss-0 & - & - & + & - \\
\hline
+ & -     & κάθομαι     & σε  & τα     & αβγά & Poss-0    & - & - & - & - \\
\hline
+ & -     & κάθομαι     & σε  & τα    & αγκάθια & E  & - & - & - & - \\
\hline
+ & -     & κιτρινίζω   & από & τον     & φόβο & Poss-0   & - & + & - & - \\
\hline
+ & -     & λύνομαι     & σε  & τα     & γέλια & E   & - & - & - & - \\
\hline
\end{tabular}
\caption{LG table of verbal MWEs (sample).}
\label{fig:03:01}
\end{table}

It becomes evident, therefore, that the LG framework together with the
requirement of substantial coverage leads to a
\textit{uniform} and \textit{consistent}
description of elementary sentences and the formal encoding of
properties across languages in a comparable manner. In this respect,
one of the main advantages of LG is that it allows comparisons between
languages and facilitates the construction of
\textit{cross-language} resources.

\section{Data selection}


The initial list of \ili{Greek} and \ili{French} MWEs that pertain to the semantic field of
emotions was manually compiled from data listed in existing LG tables
for \ili{Greek} (\citealt{fotopoulou1993}; \citealt{Mini2009}) and \ili{French} \citep{gross1982}. The selection of the \ili{Greek} MWEs was performed as a two-stage
procedure: (a) manual identification of candidate MWEs that pertain to
the semantic field emotion, and (b) validation of these candidate MWEs
for inclusion or deletion on the basis of formal criteria besides
intuitive judgments. The initial list of MWEs was further updated and
extended drawing on corpus evidence. More precisely, \ili{Greek} MWEs were
selected manually from a suite of specialised corpora \citep{giouli2014} that were developed and annotated in view of guiding
sentiment analysis. In this sense, our work is corpus-based and thus
empirical rather than purely intuitive.



Since the scope of the current work is limited to clear instances of
emotion denoting predicates (i.e., verbal MWEs), a formal distinction
between direct and indirect affective expressions that correspond to
emotion concepts was in order. For this reason, a set of lexical
semantic tests (lexical substitution, paraphrasing, etc.) was adopted as
a formal device guiding the selection of \ili{Greek} verbal emotion predicates.
Therefore, a candidate MWE is selected for inclusion in the
lexicon if at least one of the following criteria is met:


\textit{Criterion 1}: A candidate Emotion MWE is selected if it can be replaced by a sequence
that comprises one of the verbs \textit{feel} or
\textit{cause} and a \textit{noun that denotes
emotion} (\textit{Nsent}), that is, if there exists a \textit{Nsent}
that is related with the concept \textit{\textsc{emotion}}
via the \textit{\textsc{is-a}} relation, and the relation
\textit{MWE is semantically equivalent to ``feel/cause Nsent}" is true. For example, the
expression in (\ref{ex:3:3}) is semantically equivalent to an expression of the
form \textit{to feel} \textsc{emotion}, where
\textit{\textsc{emotion}} is \textit{panic}:

\begin{exe}
\ex \label{ex:3:3}
\glll με πιάνει πανικός \\
me piani panikos\\
me catches panic.\textsc{nom}\\
\glt %lit. ‘panic catches me’\\
‘to panic’
\end{exe}


\textit{Criterion 2}: A candidate Emotion MWE is selected if it can be replaced by a verb
predicate that denotes emotion (\textit{Vsent}), that is, if there
exists a \textit{Vsent} defined as a conceptualization of a
\textsc{feel-emotion} or
\textsc{cause-feel-emotion} event and the relation
\textit{MWE is semantically equivalent to \textnormal{``}Vsent}" is \textit{true}. For
example, the expression in (\ref{ex:3:4}) is semantically equivalent with the
\textit{Vsent} φοβάμαι ‘to be frightened’:

\begin{exe}
\ex \label{ex:3:4}
\glll πάγωσε το αίμα μου \\
pajose to ema mu\\
froze the blood.\textsc{nom} my\\
\glt %lit. ‘my blood froze’\\
‘I was terrified’
\end{exe}

\textit{Criterion 3}: A candidate Emotion MWE is selected if it can be replaced by the verb
\textit{to be} and an adjective that denotes emotion
(\textit{Asent}), that is, if there exists an \textit{Asent} defined as
conceptualizing an
\textsc{experiencer}-\textsc{emotion}
or
\textsc{trigger}-\textsc{emotion}
entity, and the relation \textit{MWE is semantically equivalent to \textnormal{``}to be Asent}" is true.
In the example (\ref{ex:3:5}) below, the expression is semantically equivalent to
an expression of the form \textit{to be Asent
}(είμαι έκπληκτος ‘to be surprised’):

\begin{exe}
\ex \label{ex:3:5}
\glll μένω με το στόμα ανοικτό \\
meno me to stoma anikto\\
stay with the mouth open\\
\glt % lit. ‘to stay with the mouth open’ \\
‘to aghast’
\end{exe}

Finally, the selection of \ili{French} MWEs denoting emotion and their mapping
onto their \ili{Greek} counterparts was  performed manually. First,
translations or translational equivalents of the \ili{Greek} MWEs were either
provided by human translators or extracted from standard mono- and
bilingual lexicographic \linebreak resources, such as the
\textit{Trésor de la Langue Française
Informatisé}\footnote{The resources are available online 
(\href{http://atilf.atilf.fr/tlf.htm\#_blank}{http://atilf.atilf.fr/tlf.htm};
\url{http://www.wordreference.com/}).} and
\textit{WordReference.com}. In certain cases, translations
were obtained using English as a pivot language. These translations
were checked against entries in existing LG tables that define the
typologies of \ili{French} MWEs \citep{gross1982}; once an expression was spotted, it
was selected and aligned to its \ili{Greek} counterpart(s). 



The afore-mentioned process resulted in the identification of 607 \ili{Greek}
and 520 \ili{French} MWEs that constitute the linguistic data of the current
study. As one might expect, the numbers show that there is no 1:1
correspondence between \ili{Greek} and \ili{French} MWEs denoting emotion. In fact, the
process of translating the list of \ili{Greek} MWEs to the target language
proved that the transition from one language to the other was not
always straightforward. The outcome of this procedure can be summed as
follows (see also §6.2):

\begin{itemize}
\item a \ili{Greek} MWE is mapped onto a \ili{French} MWE;
\item more than one \ili{Greek} MWEs are mapped onto a single \ili{French} MWE;
\item a single \ili{Greek} MWE corresponds to more than one \ili{French} MWEs;
\item one or more \ili{Greek} MWEs correspond to a single-word \ili{French} verb rather than a
MWE.
\end{itemize}


\section{Description of the MWEs Εmotion Lexicon}
\is{Emotion Lexicon}

Data encoding was performed after data selection. The challenge of
representing MWEs in lexical resources is to ensure that the
variability along with extra features required by the different types
of MWEs can be captured efficiently (\citealt{calzolari2002}; \citealt{copestake2002}). To this end, features and properties that are appropriate
for the robust computational treatment of MWEs were retained from the
existing LG tables where applicable. MWEs extracted from corpora were
encoded from scratch. Syntactic information involves the argument structure
of the elementary sentence (by also depicting fixed and variable
elements), modification information (if permitted), syntactic
alternations, and selectional restrictions imposed over the variable
elements of the MWE (often in subject and object(s) position).
Additionally, all MWEs were coupled with information about their type
in terms of compositionality, syntactic rigidity idiosyncrasies,
and lexical choice. Moreover, semantic features that are relevant to
the semantic field to which eac of these predicates adheres are also encoded, namely:
emotion type, polarity and \is{emotion, intensity} intensity. In this way, the typologies of
emotion MWEs in \ili{Greek} and \ili{French} were consolidated and \is{cross-linguistic} cross-lingual analogies
or discrepancies were identified. In the remainder, we will elaborate
further on the encoding of verb MWEs. As we have already mentioned
above, linguistic information is encoded formally in both the \ili{Greek} and \ili{French}
tables, and this common representation facilitates the extraction of shared patterns - if any.


\subsection{Emotion MWEs: fixed expressions - SVCs }
\label{section51}
\is{classification}


In this Section, we present the classification of verbal MWEs included in the emotion lexicon. Entries were assigned a value
corresponding to the type they belong to, namely (a) fixed (or
idiomatic) expressions and (b) support (or light) verb constructions
(SVCs).



The identification of fixed expressions
involves lexical, morphosyntactic and semantic criteria (\citealt{gross1982}; \citealt{Gross1988a}; \citealt{lamiroy2003}), to be taken into account, namely:
\textit{non-compositionality},\footnote{We distinguish between
\textit{composability/decomposability} \citep[496]{Nunberg1994} and 
\textit{compositionality/non compositionality}. Composability concerns
the property of phrase elements to ``[c]arry
identifiable parts of the idiomatic meaning”.} i.e., the meaning of the expression cannot
be computed from the meanings of its constituents\textit{;
non-substitutability}, i.e., at least one of the expression
constituents does not enter in alternations at the paradigmatic axis;
and \textit{non-modifiability}, in that they enter in
syntactically rigid structures, posing further constraints over
modification, transformations etc. To this end, linguistic tests were
applied to all MWEs. The examples that follow conform to the criteria
mentioned and are classified as \textit{fixed expressions}:

\begin{exe}
\ex \label{ex:3:6}
\glll δαγκώνω την λαμαρίνα \\
dagono tin lamarina\\
bite the panel.\textsc{acc}\\
\glt %lit. ‘to-bite the panel’\\
‘to be in love’
\end{exe}

\begin{exe}
\ex \label{ex:3:7}
\gll serrer les dents \\
to.clench the teeth\\
\glt ‘to grit one's teeth/to be stressed or angry’
\end{exe}

On the other hand, identification of SVCs for inclusion in the emotion
lexicon is based on the following criteria:

\textit{SVCs Criterion 1}: SVCs comprise a \textit{support verb} (\textit{Vsup}) and a \textit{predicative
noun denoting emotion (Nsent)}; support or light verbs of this type
bear no meaning and are simply carriers of tense and person;

\textit{SVCs Criterion 2}: SVCs comprise specific (modal) verbs expressing
\textit{diverse modalities} (aspect, intensity, control,
etc.) and a \textit{Nsent}. These verbs are considered as \textit{Vsup
variants}.


In this respect, SVCs are - to some extent - characterised by
semantic transparency due to the fact that the predicative noun, which
carries the predicative function within the SVC, is used in one of its
literal senses. Basic support verbs are έχω/\textit{avoir }‘to have’,
είμαι \textit{Prep/être Prep }‘be Prep’, κάνω/\textit{faire }‘to make’,
the operator verb δίνω/\textit{donner }‘to give’\textit{,} and the
causative verbs προκαλώ/\textit{défier, provoquer }‘to cause’, προξενώ/\textit{provoquer }‘to cause’, αφήνω/\textit{laisser }‘to leave’) which
have an effect on structures with the basic \textit{Vsup}. In practice,
however, SVCs are highly \textit{idiosyncratic} and for this
reason, it is quite difficult to predict which \textit{Vsup} combines
with a noun (Abeille 1988). In the case of emotion MWEs, a close
inspection of the data, showed that domain-specific verbs assume the
function of a basic \textit{Vsup}. \ili{Greek} SVCs in this
semantic field usually select for the verbs νιώθω ‘to feel’ or
αισθάνομαι ‘to feel’ (see (\ref{ex:3:8})); similarly
their \ili{French} counterparts select for the verbs \textit{éprouver} ‘to feel’
and \textit{ressentir }‘to feel’, as shown in the example (\ref{ex:3:9}) below.
These constructions are semantically equivalent with single-word verb
predicates denoting emotion.

\begin{exe}
\ex \label{ex:3:8}
\glll νιώθω χαρά \\
niotho chara\\
feel joy.\textsc{acc}\\
\glt %lit. ‘to-feel joy’\\
‘to feel joy’
\end{exe}

\begin{exe}
\ex \label{ex:3:9}
\gll ressentir de la joie \\
to.feel of the joy\\
\glt ‘to feel joy’
\end{exe}

Additionally, certain verbs selected by the \textit{Nsent} predicates
that function as \textit{Vsup} variants may further denote the degree or
\is{emotion, intensity} intensity of the emotion. From a \is{cross-linguistic} cross-linguistic perspective, these
\textit{Vsup} variants usually form a pair of translational equivalents
in \ili{Greek} and \ili{French} as shown in the examples (\ref{ex:3:10}) and (\ref{ex:3:11}) respectively:

\begin{exe}
\ex \label{ex:3:10}
\glll  πετάω από χαρά / τη χαρά μου \\
petao apo chara / ti chara mu\\
fly from joy / the joy my\\
\glt %lit. ‘to-jump from joy/ my joy’\\
‘to be very happy’
\end{exe}

\begin{exe}
\ex \label{ex:3:11}
\gll sauter de joie\\
to.jump of joy\\
\glt ‘to be very happy’
\end{exe}


Classification of MWEs as fixed expressions or SVCs is not always
straightforward or clear-cut, as shown in  §5.2.2 and  §6.1. In
fact, some expressions seem to comprise an intermediate class placed in
between fixed expressions and SVCs. In other words, there seems to be a
continuum between fixed expressions and SVCs (or between fixed and free
expressions in other cases). These expressions may be considered (under
syntactic and semantic conditions) as \textit{semi-fixed}. A study of
these expressions related to the \textit{degree of fixedness} is
currently in progress \citep{constant2016}. 

\subsection{Syntactic properties}

Syntactic (and semantic) information is extracted from the LG tables\is{Lexicon-Grammar (LG)} for
those MWEs that were accounted for in the past; new MWEs selected for
the purposes of the current study were encoded as appropriate.
Syntactic information in the LG tables comprises the 
\textit{argument structure of each MWE, the syntactic
alternations} defined for the particular MWE, and
\textit{selectional restrictions }imposed over the variable
elements of the expressions. The encoding of modifiability concerns
specifically the fixed modifiers of SVCs. In the next sections, we elaborate on these
aspects.

\subsubsection{Argument Structure}



MWE verbal expressions (fixed non-compositional and SVCs) that denote an
emotion bear no syntactic idiomaticity, since they generally conform to
the argument structure of the main verb and there is nothing
exceptional in their syntactic behavior. This information is only
implicitly encoded in the LG tables. In this respect, naming
conventions of the initial tables correspond to specific configurations
cross-linguistically, and this information can be easily and
effectively retained in the current lexical resource. Information with
respect to the underlying structure and the syntactic function of the
(fixed and variable) constituent(s) further shows that verbal MWE
predicates conform to the following patterns: (i) fixed subject MWEs,
(ii) fixed complement MWEs, and (iii) any combination of the above.
These types are presented in detail in the following paragraphs. 



\textit{Fixed Subject MWEs} comprise a verb and an \textit{NP
}in subject position; these are both lexicalised. Complements (if any)
are represented as variant elements. According the LG notation, the
generic syntactic pattern that describes MWEs of this type is
\textit{C0 V Ω.} The symbol \textit{Ω}\footnote{\ We will
not discuss  the possible forms assumed by
\textit{Ω} in detail.} is used to denote one or more complements a
predicate subcategorises for, without further specifying their form. In
the LG tables, however, the form and function of variable elements are
further encoded. For example, the patterns \textit{C0 V N1} and
\textit{C0 V Prep C1 N2gen} used to describe \ili{Greek} and \ili{French} expressions in
(\ref{ex:3:12}) and (\ref{ex:3:13}) below, further license a variable nominal phrase in
object position or as the complement of a \textit{PP} modifier
respectively:

\begin{exe}
\ex \label{ex:3:12}
cold sweat bathes me `I am terrified' \\
\glll κρύος ιδρώτας έλουσε την Άννα \\
krios idrotas eluse tin Anna\\
cold sweat.\textsc{nom.sbj}  bathed the Anna.\textsc{acc.obj}\\
\glt %lit. ‘cold sweat bathed Anna’\\
‘Anna was terrified’
\end{exe}

\begin{exe}
\ex \label{ex:3:13}
\gll  la haine niche dans le coeur de Anna \\
the hate.\textsc{sbj} nests in the heart of Anna\\
\glt ‘Anna hates’
\end{exe}

It should be noted, however, that the variable complement is usually
employed in its cliticised form as shown in (\ref{ex:3:14}); this property is also
encoded in the LG tables. 

\begin{exe}
\ex \label{ex:3:14}
cold sweat baths me `I am terrified'\\
\glll  την έλουσε κρύος ιδρώτας \\
 tin eluse krios idrotas\\
her.\textsc{obj} bathed cold sweat.\textsc{sbj}\\
\glt %lit. ‘cold sweat bathed her’\\
‘she was terrified’
\end{exe}

Similarly, \ili{Greek} SVCs may comprise an aspectual variant of a \textit{Vsup
}and a predicative noun denoting an emotion in subject position:

\begin{exe}
\ex \label{ex:3:15}
\glll  με πιάνει πανικός \\
me piani panikos\\
me catches panic.\textsc{nom.sbj}\\
\glt %lit. ‘panic catches me’\\
‘to panic’
\end{exe}

\textit{Fixed Complement MWEs.} Verbal MWEs of this type
comprise a verb and one lexicalised complement. Most often, this
lexicalised complement is an \textit{NP }in direct object
position. The subject is represented as a variable argument of the
elementary sentence; the generic syntactic pattern that describes fixed
verbal MWEs of this type is \textit{N0 V C1, }whereas the
syntactic pattern of SVCs is \textit{N0 Vsup Nsent}:

\begin{exe}
\ex \label{ex:3:16}
\glll   δαγκώνω την λαμαρίνα \\
dagono tin lamarina\\
bite the panel.\textsc{obj}\\
\glt %lit. ‘to-bite the panel’\\
‘to be in love’
\end{exe}

\begin{exe}
\ex \label{ex:3:17}
\gll avoir du chagrin \\
to.have of grief\\
\glt ‘to be sad’\\
\end{exe}

\textit{Fixed PP Complement MWEs} comprise a verb and a lexicalised
prepositional phrase (\textit{PP}) complement. The variable \textit{NP}
in subject position along with other non-fixed elements (if any) is
also represented as appropriate. The generic pattern that describes
this class is of the form \textit{N0 V Prep C1}. In  (\ref{ex:3:18}), the \ili{Greek} MWE consists of the verb κάθομαι ‘to sit’
and the lexicalised \textit{PP} στα καρφιά ‘on the nails’.
Similarly, the \ili{French} MWE in (\ref{ex:3:19}) consists of the verb \textit{rire }‘to
laugh’ and the PP \textit{aux larmes }‘to tears’:


\begin{exe}
\ex \label{ex:3:18}
\glll  κάθομαι στα καρφιά \\
kathome sta karfia\\
sit to.the nails\\
\glt %lit. ‘to-sit to the nails’\\
‘to be anxious’, ‘to be on tenterhooks’
\end{exe}

\begin{exe}
\ex \label{ex:3:19}
\gll  rire aux larmes \\
to.laugh to.the tears\\
\glt ‘to roar with laughter’
\end{exe}

\textit{Fixed}\textit{ Adjunct MWEs} comprise a verb plus an
adjunct (often adverb) that are both lexicalised; other variable
complements are depicted in the structure of the relative elementary
sentence:

\begin{exe}
\ex \label{ex:3:20}
\glll   φέρω βαρέως \\
fero vareos\\
carry heavily\\
\glt %lit. ‘to-carry heavily’\\
‘to be very sad’
\end{exe}

\begin{exe}
\ex \label{ex:3:21}
\gll ils s’ aiment comme deux tourtereaux \\
they REFL love like two lovebirds\\
\glt ‘they are in love’
\end{exe}

Finally, a number of verbal MWEs have a \textit{syntactic}
structure that is a combination of the configurations presented;
these structures are exhaustively represented in the resource:

\begin{exe}
\ex \label{ex:3:22}
\glll  μου ανεβαίνει το αίμα στο κεφάλι \\
mu aneveni to ema sto kefali\\
me.\textsc{gen} raises the blood.\textsc{nom} to.the head\\
\glt %lit. ‘the blood raises me tot he head’\\
‘to become very angry’
\end{exe}

\begin{exe}
\ex \label{ex:3:23}
\gll  la moutarde monte au nez \\
the mustard raises to.the nose\\
\glt ‘to become very angry’
\end{exe}

\begin{exe}
\ex \label{ex:3:24}
\gll  avoir froid dans le dos\\
to.have cold in the back\\
\glt ‘to be terrified’
\end{exe}


\subsubsection{Modification}
\label{section522}

\textit{Fixed non-compositional verbal expressions} do not
allow for any modification over the fixed constituents. On the
contrary, SVCs are considered as syntactically more flexible
constructions, and adjectival modification is allowed over the
\textit{Nsent}. However, constructions with a \textit{Vsup} do not
conform to a uniform pattern of modification \citep{moustaki2008}. 
Adjectival modification within the MWE is found to be
\textit{free}, \textit{semi-fixed} or even
\textit{fixed}. Modification in both languages involves
\textit{intensifiers} or - more generally -
\textit{grade indicators}  like μεγάλος/\textit{grand} ‘big’, λίγος/\textit{ petit} ‘few’,
φοβερός/\textit{intense }‘awful’,
 άκρατος/\textit{intense }‘awful’, etc.:


\begin{exe}
\ex \label{ex:3:25}
\glll ο Γιάννης νιώθει ένα παθολογικό / υπαρξιακό / αόριστο / *δυνατό άγχος\\
o Janis niothi ena patholojiko / iparksiako / aoristo / dinato anchos\\
the John feels a pathological / existential / vague / strong anxiety\\
\glt ‘John feels a pathological / existential / vague / *strong anxiety’
\end{exe}

\begin{exe}
\ex \label{ex:3:26}
\gll Jean éprouve une angoisse pathologique / vague / sourde / mortelle / de mort / existentielle \\
John feels an anxiety pathological / vague / silent / deadly / of death / existential\\
\glt ‘John feels a pathological / vague / silent / deadly / existential anxiety’
\end{exe}

\begin{exe}
\ex \label{ex:3:27}
\glll με έπιασε μαύρη απελπισία / *λύπη \\
 me epiase mavri apelpisia / lipi\\
me cought black dispair.\textsc{nom} / sorrow.\textsc{nom}\\
\glt %lit. ‘a black despair / sorrow caught me’\\
‘I was in total despair’
\end{exe}

\begin{exe}
\ex \label{ex:3:28}
\gll  J' ai eu une peur bleue / *tristesse bleue. \\
I have had a fear blue / sadness blue\\
\glt %lit.‘I had a blue fear / sadness‘\\
‘I was terrified’
\end{exe}
 
The\textit{ fixed} modifiers, i.e., modifiers that seem to be
idiosyncratic to a given \textit{Nsent} cannot be employed
productively. We note that in example (\ref{ex:3:27}) , the adjective
μαύρη ‘black’ is only used as a modifier of the nominal
predicate απελπισία ‘despair’, which cannot be described
literally as being of black colour. Similarly, the \ili{French} adjective
\textit{bleu }‘blue’ in (\ref{ex:3:28}) is only used with the nominal predicates
\textit{peur }‘fear’. These expressions are also encoded as fixed in
the LG tables. Actually, this is evidence of the existence of grey
zones between SVCs and fixed expressions (cf. \sectref{section51}).



To conclude, \ili{Greek} and \ili{French} \textit{Nsent} predicates in a SVC select from
a variety of modifiers in an idiosyncratic manner. Moreover, the
respective \ili{Greek} and \ili{French} expressions seem to present a variable degree of
fixedness depending on the \textit{Nsent} and the modifier selected.
Free and semi-fixed modifiers are not encoded in the lexicon so far; on
the contrary, fixed modifiers of the predicative noun are encoded as
fixed elements of the expression.


\subsubsection{Syntactic alternations}


Information relative to syntactic alternations encoded in the LG tables\is{Lexicon-Grammar (LG)}
was also kept in the lexical resource.  The
causative-inchoative alternation is a syntactic property that involves
verbs (or pairs of verbs) which have an intransitive and a transitive
usage. The inchoative form (intransitive) denotes a \textit{change of
state}, and the causative form (transitive) denotes a \textit{bringing about of a
change of state}. A number of emotive MWEs were found to enter this
alternation. The following cases have been attested in the LG tables:

\textit{First case}: a pair of two MWEs each one comprising a distinct verb whereas all the
other fixed elements are identical. The two verbs (which are often
predicates denoting movement) normally enter (or signal) the
transitive-intransitive alternation:

\begin{exe}
\settowidth \jamwidth{(CAUS)}
\ex \label{ex:3:29}
to take one out of one's clothes `to make someone angry'\\
\glll o Γιάννης την \textbf{βγάζει} τη Μαρία από τα ρούχα της  \\
o Janis tin vjazi ti Maria apo ta rucha tis {}\\
the John.\textsc{sbj} her.\textsc{obj}  takes.out the Maria.\textsc{obj} from the clothes hers {}\\
\jambox{(CAUS)}
\glt %lit. ‘John takes Maria out of her clothes’\\
‘John makes Maria very angry’
\end{exe}

\begin{exe}
\settowidth \jamwidth{(INCHO)}
\ex \label{ex:3:30}
to get out of one's clothes `to be made angry'\\
\glll η Μαρία \textbf{βγήκε} από τα ρούχα της  \\
i Maria vjike apo ta rucha tis {}\\
the Maria.\textsc{sbj} went.out from the clothes hers {}\\ \jambox{(INCHO)}
\glt %lit. ‘Maria went-out of her clothes’\\
‘Maria was made very angry’
\end{exe}

\begin{exe}
\settowidth \jamwidth{(CAUS)}
\ex \label{ex:3:31}
to send someone to the seventh sky `to make someone happy'\\
\gll Eric \textbf{envoie} Léa au septième ciel  \\
Eric sends Lea to.the seventh sky {}\\ \jambox{(CAUS)}
\glt %lit.‘Eric sends Lea to the seventh heaven’ \\

‘Eric makes Lea very happy’
\end{exe}

\begin{exe}
\settowidth \jamwidth{(INCHO)}
\ex \label{ex:3:32}
to go up to the seventh sky `to be happy'\\
\gll Léa \textbf{monte} au septième ciel \\
Lea goes-up to.the seventh sky {}\\  \jambox{(INCHO)}
\glt %lit. ‘Lea goes up to the seventh heaven.’\\
‘Lea is in the seventh heaven’
\end{exe}

\textit{Second case}: MWEs that comprise a verb that enters the transitive-intransitive
alternation (ergativity):

\begin{exe}
\settowidth \jamwidth{(CAUS)}
\ex \label{ex:3:33}
to turn someone's lights on `to make someone angry'\\
\glll  o Γιάννης μου \textbf{άναψε} τα λαμπάκια	 \\
o Janis mu anapse ta labakia {}\\
the John.\textsc{sbj} I.\textsc{gen} turned.on the lights.\textsc{obj} \\ \jambox{(CAUS)}
\glt %lit. ‘John turned the lights on for me.’\\
‘John made me very angry’
\end{exe}

\begin{exe}
\settowidth \jamwidth{(INCHO)}
\ex \label{ex:3:34}
my lights turn on `I get angry'\\
\glll μου \textbf{άναψαν} τα λαμπάκια	 \\
 	mu anapsan ta labakia\\
	I.\textsc{gen} turned.on the lights.\textsc{sbj}\\ \jambox{(INCHO)}
\glt  	%lit.‘the lights turned on for me.’ \\
‘I got very angry’
\end{exe}

Similarly, other syntactic properties were encoded in the LG tables
where applicable (i.e., passivisation, genitive-dative alternation,
etc.).


\subsubsection{Selectional restrictions}


Α number of \textit{selectional restrictions} imposed on
the \textit{variable} elements of the MWEs (in subject and
object(s) position) were encoded as properties in the LG tables. Like
their single word counterparts, verbal MWEs denoting emotion select a
nominal element that is obligatorily [\textit{+human}]. Being at the
heart of the syntax-semantics interface, this information relates to
the participants of the emotion event. An emotion event generally
involves an \textsc{Experiencer} (that is, the individual experiencing
the psychological state) and a \textsc{Theme} (that is, the content or
object of the psychological state) or - occasionally - a
\textsc{Cause.} These participants, however, are not realised in a
uniform way in single word verbal predicates. In this respect, the
distinction between \is{SubjectExperiencer} \textit{SubjectExperiencer} (SubjExp)
and \is{ObjectExperiencer} \textit{ObjectExperiencer} (ObjExp) single word verbal
predicates has been established \citep{beletti1998} based on the
syntactic distribution of the verbal arguments and the associated
Semantic Roles. The former project the \textsc{Experiencer} of the
emotion as their structural subject and the \textsc{Theme} or the
\textsc{Stimulus} as their structural object; the latter realise the
\textsc{Theme} or the \textsc{Stimulus} as the subject and the
\textsc{Experiencer} as their object. This information is of relevance
to a number of NLP applications, and although it has not been encoded
in the LG tables, it can be deduced easily. In fact, as it has been
shown \citep{giouli2014} for the single-word verbal predicates
denoting emotion, the \textit{N0} or \textit{N1} complements with the
[\textit{+human}] restriction can be mapped onto the
\textsc{Experiencer} participant in the emotion event.



This is  true for MWEs too; here the \textsc{Experiencer} is
realised not as a structural subject but in object position. In this
sense, the non-fixed element that bears the semantic restriction
[\textit{+human}] corresponds unambiguously to the \textsc{Experiencer}
of the emotion. In the following examples, the \textsc{Experiencer} of
the emotion is expressed by the subject of the \ili{Greek} and \ili{French} expressions as
shown in (\ref{ex:3:35}) and (\ref{ex:3:36}) respectively, or by the direct object as
depicted in (\ref{ex:3:37}) and (\ref{ex:3:38}) below:

\begin{exe}
\ex \label{ex:3:35}
\glll  η Άννα πετάει από χαρά \\
i Anna petai apo chara\\
 the  Anna.\textsc{sbj.exp} flies of joy\\
\glt  %lit.‘Anna jumps of joy’\\
‘Anna is very happy’
\end{exe}

\begin{exe}
\ex \label{ex:3:36}
\gll Anna rayonne de joie \\
Anna.\textsc{sbj.exp} shines of joy\\
\glt ‘Anna is very happy’
\end{exe}

\begin{exe}
\ex \label{ex:3:37}
to take one out of one's clothes `to make someone very angry'\\
\glll ο Γιάννης με έβγαλε από τα ρούχα μου \\
o Janis me evjale apo ta rucha mu\\
the John.\textsc{sbj} me.\textsc{obj.exp} took.out of the clothes mine\\
\glt  %	lit.‘John took me out of my clothes’\\
‘John made me very angry’
\end{exe}

\begin{exe}
\ex \label{ex:3:38}
\gll ce film m' a ému aux larmes \\
this film.\textsc{sbj} me.\textsc{obj.exp} has touched in tears\\
\glt ‘this film moved me to tears’
\end{exe}

Additionally, other \textit{selectional restrictions} imposed
on the variable elements of the verbal MWE are encoded. These
restrictions further specify the type of complements (nominal,
prepositional, sentential) that these predicates sub-categorise for. In
this respect, prepositions selected by the MWE predicates are formally
depicted and encoded.





\subsection{Semantic classification}



The semantic classification\is{classification} of the studied \ili{Greek} and \ili{French} MWEs  was aimed at
grouping them under pre-defined emotional concepts and at
distinguishing semantically between expressions that are near synonyms.
This was attempted following a schema defined for single-word \ili{Greek} verbs
denoting emotion in \citep{giouli2012}  along three dimensions:
(a) \textit{emotion type} (b) \is{emotion, polarity} \textit{emotion
polarity} (c) \textit{emotion intensity} \is{emotion, intensity} and (d)\is{aspect of the emotion event}
\textit{aspect} of the emotion event. The semantic
classification of verbal MWE predicates was performed separately by two
experienced linguists in the form of primarily intuitive semantic
grouping; at the next stage, discrepancies between the annotations thus
obtained were discussed and resolved, whereas cases for which no
agreement could be consolidated were left aside for future treatment.
The outcome of this procedure was the definition of specifications that
would be applicable for distinguishing between semantic classes.



Emotion is described as a set of two or more dimensions; the most common
ones are \textit{polarity}, i.e., \textit{positive
or negative connotation} of emotion and the
\textit{intensity} or strength of the emotion. The notion of
semantic polarity, or the semantic orientation of words, i.e., whether
they denote a positive or a negative emotion has also been the focus of
attention in many studies aimed at sentiment analysis (\citealt{esuli2006}; \citealt{wilson2005}) inter alia. In our approach, the
encoding schema provides for the annotation of the \is{emotion, polarity} \textit{a
priori polarity} of the emotion denoted, which subsumes one of the
following values: (a) \textit{positive}, i.e., predicates
which express a pleasant feeling (b) \textit{negative},
i.e., predicates which express an unpleasant feeling (c)
\textit{neutral}, i.e., predicates that denote an emotion
that is neither positive not negative and (d)
\textit{ambiguous}, i.e., predicates expressing a feeling,
the polarity of which is context-dependent (e.g., surprise).



Polarity identification results in a coarse - yet quite effective -
classification\is{classification} of emotion expressions; a more fine-grained one was
attempted on the basis of emotion types. Psychological considerations
of sentiment claim that some emotions are more basic than others,
therefore, they should be universal to all human languages. The
identification of basic emotions is based upon specific functional and
physiological criteria, yet languages are claimed to possess
inventories that comprise a great number of emotion predicates that
cannot be easily accommodated within such fairly straightforward
schemes. To this end, different dimensions of emotion can be used to
delineate senses. In the work presented here we adopted an extended
version of the typological model defined by \cite{plutchik2001}.  The
initial model comprises eight basic emotions:
\textit{anger}, \textit{fear},
\textit{sadness}, \textit{disgust},
\textit{surprise}, \textit{anticipation},
\textit{acceptance} and \textit{joy}. On the
basis of corpus evidence derived from a tri-lingual corpus (English, \ili{Greek}, Spanish) annotated for sentiment (Giouli et al. 2013), the
initial list of basic emotions was further extended with a set of
complex emotions, such as \textit{love \textup{and }hate} or
emotions of (self-)appraisal (e.g., \textit{shame},
\textit{respect}) that were not considered by Plutchik. To
better account for the conceptual representation of the emotion
vocabulary, the final set of emotion types includes 15 new classes,
namely: \textit{admiration}, \textit{boredom},
\textit{disappointment}, \textit{envy},
\textit{gratitude}, \textit{hate},
\textit{indifference}, \textit{jealousy},
\textit{love}, \textit{relaxedness},
\textit{remorse}, \textit{resentment},
\textit{respect}, and \textit{shame}. \ili{Greek} and \ili{French}
MWEs were assigned an emotion concept; this classification results in
grouping \ili{Greek} and \ili{French} verbal MWEs under emotion concepts.



Moreover, to model the semantic distinction between near synonyms that
occur within a semantic class such as φοβάμαι ‘to
be scared’, πανικοβάλλομαι ‘to panic’,
μου κόπηκαν τα ήπατα ‘to be very frightened’ etc.,
entries were further coupled with the feature
\textit{intensity }(or\textit{ strength)}. The
following values are provided for by the schema for the feature
\textit{strength}: \textit{low}, \textit{medium},
\textit{high}, and \textit{uncertain}. In fact,
emotion verbal predicates have been shown to possess scalar qualities
\citep{fellbaum2012}. In this respect, groups of verbs
that were assigned the same emotion type were checked in order to
identify different degrees of intensity of the same underlying emotion.
In this respect, intuitive judgments of trained lexicographers were
systematised and a number of linguistic tests were defined aimed at the
consistent annotation and the ordering of predicates according to the
intensity of the emotion they denote.



In both languages, intensity was proved to be dependent on the following
aspects: (a) degree of fixedness (b) modifier selected (in SVCs) and
(c) the \textit{Vsup} selected. More precisely, the majority of verbal
idioms were judged to express an emotional state or event of high
intensity; these were further marked as not accepting any modifier.
Similarly, the \textit{Vsup} of an SVC seemed to have an impact on the
value assigned to the feature intensity. Ultimately, a number of
\textit{Vsup} function as an intensifier of the emotion denoted. In
this respect, the verbs έχω/\textit{avoir} ‘to have’,
νιώθω/\textit{éprouver} ‘feel’ and αισθάνομαι/\textit{ressentir} ‘to
feel’ in \ili{Greek} and \ili{French} respectively usually denote an emotion that bears
the value \textit{medium} for the feature \textit{intensity}; on the
contrary, when the verbs πετάω ‘fly’ and \textit{rayonner}
‘shine’ are employed instead, the entire expression is marked as
denoting the same emotion, yet with an intensity marked as
\textit{high}. Modification of the \ili{Greek} and \ili{French} expressions is permitted
only when the \textit{Vsup} that evokes a \textit{medium} intensity of an
emotion is employed as shown in (\ref{ex:3:39}) and (\ref{ex:3:41}); when the \textit{Vsup} denoting
an emotional state of \textit{high} intensity is employed modification
is blocked (\ref{ex:3:40}) and (\ref{ex:3:42}):


\begin{exe}
\ex \label{ex:3:39}
\glll η Άννα νιώθει χαρά / μεγάλη χαρά \\
 i Anna niothi chara / mejali chara\\
 the Anna feels joy / big joy\\
\glt % lit.‘Anna feel joy / big joy’\\
‘Anna is happy / very happy’
\end{exe}

\begin{exe}
\ex \label{ex:3:40}
\glll η Άννα πετάει από χαρά / *μεγάλη χαρά {} {} {}\\
i Anna petai apo chara / *mejali chara {} {} {}\\
the Anna flies of joy / big joy \\
\glt %lit.‘Anna jumps of joy’\\
‘Anna is very happy’
\end{exe}

\begin{exe}
\ex \label{ex:3:41}
\gll Anna éprouve de la joie / une grande joie \\
 Anna feels of the joy / a big joy\\
\glt ‘Anna is happy / very happy’
\end{exe}

\begin{exe}
\ex \label{ex:3:42}
\gll Anna rayonne de joie / *rayonne d’  une grande joie \\
Anna shines of joy / shines of a big joy\\
\glt ‘Anna is very happy’
\end{exe}



Finally, the encoding schema also provides values for the feature
\is{aspect of the emotion event} \textit{aspect}, i.e., the perspective taken on the internal
temporal organization of the emotion event. Different values of
\textit{aspect} distinguish different ways of viewing the
internal temporal constituency of the same event. The schema adopted
\ provides the values \textit{inchoativeAspect},
\textit{terminativeAspect},
\textit{durativeAspect} and
\textit{frequentiveAspect}. The encoding at this level,
however, has been finalised only for the \ili{Greek} MWEs.

\section{Discussion}


At the final stage of our study, an examination of the interplay between
syntactic, semantic and lexical features of the studied MWEs was
performed. Moreover, \is{cross-linguistic} cross-lingual similarities and differences were
identified. As has already been mentioned, our working hypothesis was
that despite idiosyncrasies, MWEs that pertain to a given semantic
class share features that are characteristic for this class; moreover,
these features can be even attested cross-linguistically. As has
already been mentioned in \sectref{section51} above, MWE identification and
classification employs lexical and morphosyntactic besides semantic
criteria (\citealt{gross1982}; \citealt{Gross1998}; \citealt{lamiroy2003}), however, do not apply in all cases in a
uniform way, and the variability attested brings about the notion \is{degree of fixedness}
\textit{degree of fixedness} \citep{gross1996}.  On the one hand,
fixed expressions bear a meaning that cannot be computed based on the
meaning of their constituents and the rules used to combine them. SVCs,
on the other hand, have a rather transparent meaning due to the
presence of the \textit{Nsent }which retains its original sense.
However, a number of problems are posed and the limits between SVCs and
verbal fixed expressions (see also  \sectref{section51}) are in some cases
fuzzy: despite the semantic transparency entailed by the
\textit{Νsent}, the overall structure is often susceptible to a number
of constraints as shown in examples (\ref{ex:3:43}) and (\ref{ex:3:44}) below:


\begin{exe}
\ex \label{ex:3:43}
\glll φωτίστηκε το πρόσωπo του Νίκου από χαρά\\
fotistike to prosopo tu Niku apo chara\\
was.lit.up the face.\textsc{nom} the Nikos by happinesss\\
\glt %lit.‘the face of Nikos was-lit-up by hapiness’ \\
‘Nikos’ face lit up with happiness’
\end{exe}

\begin{exe}
\ex \label{ex:3:44} 
\glll *φωτίστηκε ο Νίκος από χαρά\\
fotistike o Nikos apo chara\\
was.lit.up the Nikos.\textsc{nom} by hapiness\\
\end{exe}



According to a study on verbal MWEs \citep{balibar1995}, expressions
like the one depicted in (\ref{ex:3:43}) are defined as
\textit{semi-fixed \textup{ones}}. In this respect, the
verbal MWEs under study were found to be placed along the continuum
\textit{fixed}, \textit{semi-fixed} and SVCs. Consequently,
the class of semi-fixed expressions constitutes a grey zone, the
intermediate mentioned in \sectref{section51} and \sectref{section522} above. However, in this work, we opted for classifying semi-fixed expressions that
comprise a predicative noun \textit{Nsent} as SVCs.



One step further, the correlation between the features
\textit{non}-\textit{compositionality}/\linebreak \textit{fixedness}
and the attributes \textit{polarity} and
\textit{intensity} was examined. Our underlying assumption
was that the degree of fixedness of the relevant expressions and the
polarity/intensity of the emotion denoted are highly correlated. In
this respect, the focus was placed on the values assigned for the
feature \textit{intensity} of the emotion denoted and their correlation to the
aspects of MWE category (i.e., fixed expression or SVC). The majority
of the considered \ili{Greek} MWEs, that is 410 expressions, were attributed the
value \textit{Negative} for the feature \textit{Polarity},
whereas only 169 were encoded as \textit{Positive} and 133
as \textit{Neutral}. Of these, 97 MWEs denote
\textit{anger}, 73 denote \textit{fear}, and 105
denote \textit{sadness}; 90 expressions were identified as
expressing \textit{joy} and 30 a
\textit{surprise} event. The remaining expressions are
distributed across the remaining conceptual categories. Another
interesting remark concerns verbal idiomatic \textit{non-compositional}
expressions; most of the expressions (260) that have been assigned the
value \textit{negative} for the feature
\textit{polarity} are also encoded as being of type
\textit{fixed} (as opposed to 150 expressions classified as
SVCs). Additionally, fixed expressions were – in most cases –
attributed a value \textit{high} for the feature\textit{
intensity}. Of the approximately 300 \textit{fixed }expressions 210 are
assigned the value \textit{high }for the feature \textit{intensity}. On
the contrary, SVCs in both languages do not constitute a uniform class,
and the overall emotion intensity denoted depends largely on the
\textit{Vsup} selected rather than the \textit{Nsent }itself. Three
cases are identified:


\begin{itemize}
\item 
The \textit{Vsup} is selected by all \textit{Nsent }predicates; these
verbs\footnote{For example, έχω/\textit{avoir} ‘to have’,
νιώθω/\textit{éprouver} ‘to feel’ and
αισθάνομαι/\textit{resentir} ‘to feel’.} adhere to a
productive and relatively open paradigmatic axis, and syntactic
variability is allowed to some extent. In these cases, the
\textit{intensity} of the emotion denoted is determined on the basis of
the semantics of the \textit{Nsent}; any possible modifier functions as
an intensifier of the emotion denoted. 
\item 
The \textit{Vsup} selection is subject to lexical restrictions, and
syntactic variability is not allowed.\footnote{For example,
ανατριχιάζω/\textit{frissoner} ‘to shiver’, λάμπω/\textit{briller} ‘to shine’,
λιώνω/\textit{fondre} ‘to dissolve’, etc.} In this case, the \textit{Vsup}
contributes to the intensity and/or some aspectual
meaning of the emotion denoted. The overall intensity of the
emotion expression is determined on the basis of the semantics of the
\textit{Nsent}, and the \textit{Vsup} functions as an intensifier.
\item 
The \textit{Vsup }selection is extremely limited or unique, and a strong
lexicalization is attested; syntactic variability is not allowed and
the \textit{Vsup} is an intensive or aspectual variant that has a
strong impact on the intensity of the emotion denoted:
\end{itemize}


\begin{exe}
\ex \label{ex:3:45}
\glll με τρώει η ζήλια / *στενοχώρια / *λύπη \\
me  troi i zilia / stenochoria / lipi\\
me eats the jealousy.\textsc{nom} /  worry.\textsc{nom} / regret.\textsc{nom}\\
\glt %lit.‘ jealousy eats me’\\
‘to be devoured by jealousy’
\end{exe}

\begin{exe}
\ex \label{ex:3:46} être rongé par la jalousie
\glt  to.be gnawed by the jealousy\\
‘to be devoured by jealousy’
\end{exe}



\section{Cross-lingual considerations}
\is{cross-linguistic}

Research on idioms reported in \cite{villavicencio2004} shows that
there is remarkable variation in MWEs across languages. \ Similar
variations are attested in the data used in the current research. As
one might expect, there is no one-to-one correspondence between
syntactic patterns in the two languages. It is worth looking at SVCs
and fixed expressions separately here.



\ili{Greek} and \ili{French} SVCs present a number of similarities in terms of the
underlying syntax and semantics. In some cases, even a direct
lexico-syntactic correspondence is observed for a cross-lingual MWE
pair with similar semantics as illustrated in (\ref{ex:3:47}) and (\ref{ex:3:48}) below.
Furthermore, semantic transparency in SVCs implies more correspondences
at least at the level of syntactical patterns - we have demonstrated this with examples (\ref{ex:3:8}) and
(\ref{ex:3:9}). As one might expect, differences between the \ili{Greek} and \ili{French}
expressions are limited to basically those that exist in general
between the two languages, i.e., usage of determiners and the
indefinite article, case marking for NPs in subject and object position
in \ili{Greek} as opposed to PP complements in \ili{French}, etc.

\begin{exe}
\ex \label{ex:3:47}
to give to the nerves `to cause anger'\\
\glll δίνω στα νεύρα \\
dino sta nevra\\
give to.the nerves\\
\glt %lit.‘to-give to the nerves’\\
‘to cause anger’
\end{exe}

\begin{exe}
\ex \label{ex:3:48} donner sur les nerfs
\glt to.give on the nerves\\
‘to cause anger’
\end{exe}



In other cases, \ili{Greek} and \ili{French} SVCs share the same syntactic structure and
underlying semantics, yet their lexical composition is different. The
differences are attested both in the lexical choice of the
\textit{Vsup} and/or the overall structure of the verbal expression.
For example, the \ili{French} verb \textit{nager} ‘to swim’ seems to
be more productive than its \ili{Greek} counterpart πλέω ‘to sail’ as
shown in (\ref{ex:3:49}) and (\ref{ex:3:50}) below. The latter is only employed in a rather
fixed configuration and selects only one \textit{Nsent}, showing, thus,
a limited (or even fixed) distribution: 


\begin{exe}
\ex \label{ex:3:49} 
\gll nager dans le bonheur/ la joie/ l’  optimisme/ l’ amour\\
 to.swim in the happiness/ the joy/ the optimism/ the love \\
\glt ‘to be very happy/ happy/  very optimistic/  in love’
\end{exe}

\begin{exe}
\ex \label{ex:3:50}
\glll πλέω σε πελάγη ευτυχίας/ *στην ευτυχία/ *στην αισιοδοξία/ *στην αγάπη \\
pleo   se pelaji   eftichias/ stin eftichia/ stin esiodoksia/ stin ajapi\\
sail in seas happiness.\textsc{gen}/ in.the happiness/ in.the optimism/ in.the love\\
\glt %lit. ‘to sail in seas of happiness / in happiness / in optimism / in love\\
‘to be very happy/ happy/ optimistic/ full of love’
\end{exe}


Being \textit{conceptual metaphors} (usually obsolete),
\textit{fixed expressions} present in \linebreak some cases
considerable similarities in both lexical choice and structure
cross-linguistically. Again, differences are limited to the usage of
determiners, argument realization, selection of prepositions, etc.
Often, the lexicalised nominal element (that assumes the function of
the direct object) denotes a part of the body (\textit{Npc}) as
exemplified below. These expressions open a slot that is filled by a
variable noun in genitive case in \ili{Greek} and a PP complement in \ili{French} (\textit{à N} ‘to N’). This element is usually realised as a cliticised
pronoun - in both \ili{Greek} (\ref{ex:3:51}) and  \ili{French} (\ref{ex:3:52})) - and it designates the
beneficiary of the event expressed by the predicate (\citealt{leclere1976}; \citealt{fotop1993genitif}). This genitive (in \ili{Greek}) and PP (in \ili{French}) is a specific
case with semantic and syntactic features; \cite{leclere1976} has offered the term 
\textit{datif étendu}) for this genitive:

\begin{exe}
\ex \label{ex:3:51}
μου κόβονται τα ήπατα `my liver is cut'\\
\glll του κόπηκαν τα ήπατα \\
tu kopikan ta ipata\\
he.\textsc{gen} cut the liver.\textsc{pl.nom}\\
\glt %lit.‘the livers were cut for him’\\
`to be frightened’
\end{exe}

\begin{exe}
\ex \label{ex:3:52}
\gll lui casser les pieds \\
 him to.break the feet\\
\glt ‘to get on one’s nerves’
\end{exe}


In some cases, similarities are even attested in terms of argument
structure. For example, the\textit{ }\textit{\textup{\ili{Greek}}}
verbal expression depicted in (\ref{ex:3:53}) and its \ili{French} counterpart shown in (\ref{ex:3:53}) 
are encoded as entries in \ili{Greek} and \ili{French} tables. Each table features MWEs
that share the same properties and lexico-syntactic constraints; this
means that the resulting tables are to a large extent homogenous.
Therefore, correspondences between homogenous LG tables in \ili{Greek} and \ili{French}
can be obtained and mappings of MWEs from one language to the other are
feasible.


\begin{exe}
\ex \label{ex:3:53}
\glll βγαίνω από τα ρούχα μου \\
vjaino apo ta rucha mu\\
get.out from the clothes mine\\
\glt %lit.‘to-go-out from my clothes’\\
‘to be very angry’
\end{exe}

\begin{exe}
\ex \label{ex:3:54}
\gll sortir de ses gonds\\
 to.get.out of one’s pumps\\
\glt ‘to be very angry’
\end{exe}

Additionally, there are many verbal idiomatic expressions which have no
direct or precise equivalent in the other language and they correspond
to a single word verbal predicate, as shown in the \ili{Greek} example (\ref{ex:3:55})
which is attributed the \ili{French} verb \ \textit{gâcher }‘to spoil’:

\begin{exe}
\ex \label{ex:3:55}
to me he/she/it takes it out sour `he/she/it makes it unpleasant to me'\\
\glll του το βγάζω ξινό \\
tu to vjazo xino \\
he.\textsc{gen} it take.out sour\\
\glt %lit ‘to-take-out him it sour’ \\
‘to make unpleasant’
\end{exe}

Semantically almost equivalent expressions that still
present differences in aspectual
meaning and/or the intensity of the emotion have been
identified in the Greek and (to a large extent) in the French data. Sense discrimination and the alignment of Greek and French MWEs can be
enhanced on the basis of the values assigned to those emotion-related
attributes: a set of MWEs are classified under the same
emotion concept, yet sense discrimination is further enhanced on the
basis of the values assigned to emotion-related attributes.


\section{Conclusions and future research}


MWEs pose challenges with respect to their identification, analysis and
representation  both to linguistic theory and to applications.  In this study, we aimed at
consolidating the \isi{typologies of emotion MWEs} in \ili{Greek} and \ili{French} and at
finding cross-lingual analogies and asymmetries. The
syntactic, lexical and semantic properties of the \ili{Greek} and \ili{French} verbal
constructions were systematically examined, by taking also into account
the semantic properties of the semantic field, namely the features
intensity and polarity of the emotion denoted. We have
shown that, despite existing idiosyncrasies, in both languages the MWEs in the semantic
field of emotion share  properties. Moreover,
syntactic, semantic and lexical features of emotion MWEs seem to have
an impact on the semantics of the expression in terms of
emotion-related features. Future work will be oriented towards (a)
investigating the properties of semi-fixed expressions, taking into
account the degree of fixedness (b) studying the aspectual variants of
SVCs in both languages (c) revising the coding used in the emotion
Lexicon according to new studies and data and (d) populating the
lexical resource with new expressions.


\section*{Acknowledgements}

The authors would like to thank the anonymous reviewers and the editors
for their valuable comments that greatly contributed to improving the
manuscript. They are especially grateful to Manfred Sailer for his
constructive suggestions and support during the review process.

\section*{Abbreviations}


\begin{table}[H]
\begin{tabular}{ll}
\lsptoprule
Full form  & Abbreviation \\
\midrule
adjective that denotes emotion & Asent   \\
cause & \textsc{caus}   \\
experiencer & \textsc{exp}   \\
inchoative & \textsc{incho}   \\
zero element & E    \\
Lexicon Grammar  & LG \\
noun that denotes emotion & Nsent  \\
pre-verbal position & Ppv  \\
reflexive pronoun & \textsc{refl} \\
support verb construction & SVC  \\
verb that denotes emotion & Vsent   \\
suport verb & Vsup  \\
\lspbottomrule
\end{tabular}
\caption{Abbreviations.}
\end{table}

%\section*{References}

\printbibliography[heading=subbibliography,notkeyword=this]
%\nocite{*}





\end{document}
