\documentclass[output=paper]{langsci/langscibook}
\ChapterDOI{10.5281/zenodo.1182587}
\author{Sascha Bargmann\affiliation{Goethe University Frankfurt/Main, Germany}\lastand 
Manfred Sailer\affiliation{Goethe University Frankfurt/Main, Germany}}
\title{The syntactic flexibility of semantically non-decomposable idioms}

\abstract{
\cite{Nunberg1994}
caused a shift in perspective from a monolithic view of all idioms towards a word-level approach 
for semantically decomposable idioms. We take that idea one step further and argue that a semantically non-decomposable idiom 
of syntactically regular shape can also be analyzed in terms of individual word-level lexical entries. 
We suggest that these 
entries combine according to the standard rules of syntax and that the restrictions on the 
syntactic flexibility of a semantically  non-decomposable idiom follow exclusively from the interaction of the special semantics of these entries with the semantic and 
pragmatic constraints of the relevant syntactic constructions in a particular language. 
In our analysis, the words constituting a non-decomposable idiom make partially identical semantic contributions. 
We formulate our analysis in Lexical Resource Semantics 
\citep{Richter:Sailer:04}.
}

\maketitle

% \chapterDOI{} %will be filled in at production

% \epigram{}



\begin{document}



\section{Introduction}\label{ch01:Sec-Introduction}

In this paper, we make a theoretical point for loosening the close ties that \cite{Nunberg1994} claim exist between the semantic decomposability and the syntactic structure of idioms. We argue for a more uniform syntactic treatment of idioms within and \is{cross-linguistic parallelism}across languages, saying that  semantically non-decomposable idioms (henceforth abbreviated as SNDIs) like \textit{kick the bucket} can and should be analyzed as consisting of individual word-level lexical entries that combine according to the standard rules of syntax and contribute a piece of the \isi{meaning} of the idiom.

We mainly base our case on the contrast between English and German when it comes to verb placement, constituent fronting, and passivization (§\ref{Sec-SyntacticSemanticFlexibility} and §\ref{Sec-ConstructionSpecificRestrictions}). Our findings suggest that the differences in the syntactic flexibility of idioms might be due to differences among the semantic and pragmatic constraints that hold for the involved syntactic constructions in a particular language, rather than to differences in the syntactic encoding of the idioms themselves.

The central aspect of our analysis (§\ref{Sec-Analysis}) is that SNDIs are syntactically analyzed as combinations of individual words, and that these words can make identical semantic contributions to the overall meaning of the idiom. We formulate our analysis in \is{Lexical Resource Semantics}\is{semantics!Lexical Resource Semantics} 
Lexical Resource Semantics (\citealt{Richter:Sailer:04}).

Before we conclude the paper (§\ref{Sec-Conclusion}), we give a short outlook on the behavior of SNDIs in Estonian and French (§\ref{Sec-OtherLanguages}), which provides further evidence for our argument.



\section{Some data and a former approach}
\label{Sec-SyntacticSemanticFlexibility}

In this section, we will describe the behavior and architecture of SNDIs as perceived by \cite{Nunberg1994}. We will look at their analysis of \ili{English} data and challenging data from (mostly) German.



%\subsection{English SNDIs in Nunberg et al. (1994)}
\subsection{English SNDIs in Nunberg, Sag \& Wasow (1994)}

\is{multiword expression!decomposability|(}
\is{multiword expression!flexibility|(}
\is{multiword expression!non-decomposable|(}
\cite{Nunberg1994}, henceforth NSW, divide \ili{English} idioms into two categories: Idiomatically Combining Expressions (ICEs) and Idiomatic Phrases (IPs).

ICEs, exemplified here by \textit{pull strings}, consist of individual word-level lexical entries (\textit{pull} and \textit{strings}), each of which contributes a piece of the meaning of the idiom as a whole (\textit{pull} $\approx$ `use' and \textit{strings} $\approx$ `connections').

IPs, exemplified here by \textit{kick the bucket}, are syntactically and semantically monolithic, i.e.\@ the phrase as a whole is stored in the \isi{lexicon} and coupled with the overall idiomatic meaning (\textit{kick the bucket} $\approx$ `die'). In other words: NSW do not assume the meaning of an IP to be distributed over individual parts, as there are none in their opinion, not even in those cases where a division into syntactic constituents seems highly plausible because the idiom appears to have a regular syntactic structure (as is the case with \textit{kick the bucket}).

NSW base this bifold classification on the empirical observation that many \ili{English} idioms (those that they then categorize as ICEs) are syntactically \is{multiword expression!flexible} flexible to a certain degree, whereas some others (those that they then categorize as IPs) seem to be syntactically frozen. None of the sentences in (\ref{kick-the-bucket-syn}) can normally be understood in the idiomatic sense.

\begin{exe}
\ex\label{kick-the-bucket-syn}
\begin{xlist}
\ex[*]{\textit{Alex kicked the cruel bucket.}\hfill (additional adjective) \label{additional adjective}}
\ex[*]{\textit{Alex kicked a bucket.}\hfill (determiner variation) \label{determiner variation}}
\ex[*]{\textit{The bucket (that) Alex kicked was cruel.}\hfill (restrictive relative clause) \label{restrictive relative clause}}
\ex[*]{\textit{The bucket was kicked.}\hfill (passive)
\is{movement!passive}}
\ex[*]{\textit{The bucket, Alex kicked.}\hfill (NP-fronting)
\is{movement!topicalization}}
\ex[*]{\textit{It was the bucket that Alex kicked.}\hfill (\textit{it}-cleft)}
\ex[*]{\textit{What bucket did Alex kick?}\hfill (\textit{wh}-interrogative)}
\end{xlist}
\end{exe}

According to NSW, it is the syntactic monolithicity of IPs that explains their non-compatibility with the syntactic constructions in (\ref{kick-the-bucket-syn}). All the parts of an IP must be given in the exact same linear sequence provided by its phrasal lexical entry. Any disruption of that sequence results in ungrammaticality.

This syntactic monolithicity of IPs, they say, stems from their meaning not being distributed over individual parts. ICEs like \textit{pull strings}, on the other hand, allow for variations that affect the meaning of their individual components. For example, the meaning of the complement-NP's head noun can be restrictively modified or quantified over. 
IPs, in contrast, do not allow for any of these semantic operations, which is the reason for the ungrammaticality of (\ref{additional adjective})--(\ref{restrictive relative clause}).
%(\ref{additional adjective})-(\ref{restrictive relative clause}) and (\ref{determiner variation}).

All things considered, NSW observe a strong correlation between the semantic non-decomposability and the syntactic \is{multiword expression!fixedness}fixedness of IPs, which induces them to conclude that there exists a conditional dependency between the two. If an idiom is semantically non-decomposable, so they argue, it is syntactically fixed and hence to be analyzed in terms of a phrasal lexical entry, i.e.\@ a monolithic syntactic block.
\is{multiword expression!flexibility|)}
\is{multiword expression!decomposability|)}
\is{multiword expression!non-decomposable|)}



\subsection{Challenging data for Nunberg, Sag \& Wasow (1994)}


\is{multiword expression!flexibility|(}
\is{multiword expression!non-decomposable|(}
\is{multiword expression!decomposability|(}
NSW discuss the observations made for \il{German} German in earlier versions of \cite{Schenk:95} and \cite{Webelhuth:Ackermann:99} that SNDIs like \textit{den L\"offel abgeben} `die' (lit.:\,`pass on the spoon') or \textit{ins Gras bei{\ss}en} `die' (lit.:\,`bite in the grass') can undergo syntactic processes.
These include 
\is{movement!verb second} the dislocation of the finite verb to the second position (V2), see (\ref{loeffel-v2}), 
and 
\is{movement!topicalization} the dislocation of idiom chunks to the initial position (the \textit{Vorfeld}), see (\ref{loeffel-vf0}). The example in (\ref{loeffel-vf}) is taken from \citeauthor{Trotzke:Zwart:14} (\citeyear{Trotzke:Zwart:14}: 138), example (\ref{loeffel-vf1}) is a corpus example.%
\footnote{We will not provide a full morphological glossing for German, but only indicate the parts that are relevant for the discussion at hand.
%We will not provide a full morphological glossing for German, but only indicate the relevant parts, such as separable verbal particles in (\ref{loeffel-vf0}), sometimes, case if it is important for the discussion, as in (\ref{pulli-q}).
}

\begin{exe}
\ex\label{loeffel-v2} \label{Loffel} \label{loeffel}
\gll Dann gab Alex den L\"offel ab.\\
     then passed Alex the spoon on\\
\glt `Then Alex died.'
\end{exe}


\ea
\ea\label{loeffel-vf0}
\gll Den L\"offel hat er ab-gegeben. \\
the spoon has he {on-passed}\\
\glt `He died.'\label{loeffel-vf}
\ex
\gll Den L\"offel habe er noch nicht ab-geben wollen, \ldots\\
the spoon has he still not on-pass want\\
\glt `He didn't want to die yet, \ldots'%
\footnote{IDS corpora: N92/JAN.03243 Salzburger Nachrichten, 28.01.1992}\label{loeffel-vf1}
\z
\z

NSW briefly explore a purely linearization-based/phonological explanation of data like those in (\ref{loeffel}). However, \il{German}  SNDIs also allow for \is{passive} passivization, see (\ref{loeffel-pass}), a syntactic operation that cannot be analyzed as a simple word-order alternation, as it involves adding, inflecting, and often also deleting material.

\ea
\gll Hier wurde der L\"offel ab-gegeben. \\
here was the spoon on-passed \\
\glt `Someone died here.' \label{loeffel-pass}
\z

These data suggest that an IP-like analysis is less attractive for German than for \ili{English}, as there seem to be no syntactic restrictions in German that correlate with \is{multiword expression!non-decomposable} 
semantic non-decomposability.%
\footnote{\cite{Soehn:diss} pursues an IP-analysis of \il{German} German SNDIs. He accounts for the data in (\ref{Loffel}) and (\ref{loeffel-pass}) by his formulation of quite abstract phrasal lexical entries that leave many syntactic relations underspecified. A disadvantage of this account is that the lexical representation of SNDIs differs dramatically from language to language, even for syntactically very similar idioms, such as those consisting of a verb and a direct object. \citeauthor{Mueller:13Unifying} (\citeyear{Mueller:13Unifying}: 923) argues that an analysis that reflects \isi{cross-linguistic parallelism} is generally to be preferred over one that does not.}

\largerpage
It is worth noting that English SNDIs are not necessarily fully \is{multiword expression!fixedness}fixed either. We will list three commonly mentioned types of data that support this (see, for example, \citealt{Baldwin2010}) and add a fourth one.
%that can be taken as pieces of evidence for this oberservation and add a fourth one. 
First, many \ili{English} SNDIs have the same syntactic structure as any regular English V-NP combination, which sets SNDIs apart from syntactically irregular expressions like \textit{kingdom come} `paradise'.\is{multiword expression!syntactically irregular} 
Second, English SNDIs show full morphological \is{multiword expression!flexible} flexibility on their \is{multiword expression!verbal}verbal heads, see (\ref{kick-the-bucket-inflection}).

\begin{exe}
\ex\label{kick-the-bucket-inflection}
\begin{xlist}
\ex \textit{Alex kicks/kicked the bucket.} \label{kick-the-bucket-inflection-sing}
\ex \textit{Kim's kicking the bucket caused great concern.} \label{kick-the-bucket-inflection-ing}
\end{xlist}
\end{exe}

Third, SNDIs allow for certain modifiers within the complement-NP, see (\ref{kick-the-bucket-modification}).%
\footnote{Semantically, however, none of these modifiers seems to apply to the meaning of idiomatic \textit{bucket}. For suggestions on how these additional adjectives should be interpreted, see \cite{Ernst:81} and \cite{Potts:05}, among others.}

\begin{exe}
\ex\label{kick-the-bucket-modification}\textit{Alex kicked the political/proverbial/goddamn/golden bucket.}
\end{exe}

Fourth, we even find \is{passive} passive examples of \textit{kick the bucket}, see (\ref{kick-the-bucket-passive}).

\begin{exe}
\ex\label{kick-the-bucket-passive}\textit{When you are dead, you don't have to worry about death anymore. {\ldots} The bucket will be kicked.}%
\footnote{\textit{The Single Man} by John Paschal \& Mark Louis. 2000. Lincoln, NE: iUniverse. Page 195.}
\end{exe}

We will turn to such examples in §\ref{Sec-RestrictionsEnglish}. For the moment, it suffices to show that the postulated causal relation between \is{multiword expression!non-decomposable} semantic non-decomposability and 
syntactic fixedness \is{multiword expression!fixed}
loses much of its appeal in the light of these data.

We conclude that semantic non-decomposability and syntactic fixedness are not necessarily mutually dependent, i.e.\@ an SNDI can show syntactic flexibility. This is rather obvious in \il{German} German, but there are also some indications for \ili{English}.\is{multiword expression!flexibility|)}\is{multiword expression!non-decomposable|)}\is{multiword expression!decomposability|)}{}


\largerpage
\section{Construction-specific restrictions}
\label{Sec-ConstructionSpecificRestrictions}


In this section, we will look at German and \ili{English} and point out the differences between these two closely-related languages when it comes to verb placement, constituent fronting, and the \is{passive} passive voice.



\subsection{German}
\label{Sec-RestrictionsGerman}
\il{German} 

We will now go through the three mentioned syntactic processes in German and show that they impose no (or rather weak) semantic or pragmatic restrictions.



\subsubsection{V2-Movement}
\il{German} 

\is{movement!verb second|(}
In German, the position of the finite verb determines the clause type. In declarative main clauses, for example, the finite verb occurs in second position (V2), see (\ref{V2}). In subordinate clauses, it typically occurs in final position (V-final), see (\ref{V-final}).


\ea
\ea\label{V2-vs.-V-final}
\gll Alex hat gestern einen Freund mit-gebracht. \\
Alex has yesterday a friend along-brought \\
\glt `Alex brought along a friend yesterday.' \label{V2}
\ex
\gll dass Alex gestern einen Freund mit-gebracht hat \\
that Alex yesterday a friend along-brought has \\
\glt `that Alex brought along a friend yesterday' \label{V-final}
\z
\z

V-final is taken to be the basic position. V2 is taken to be derived. The dislocation of the finite verb from V-final to V2 is commonly referred to as \is{movement!verb second} V2-movement. There are only very few restrictions as to what verbs may occur in V2. All of these restrictions are either morphological or syntactic, never semantic or pragmatic (\citeauthor{Schenk:95} \citeyear{Schenk:95}: 262--263). As already mentioned, the fronted verb must be finite, compare (\ref{V2}) above with (\ref{V2-nonfinite}).%
\footnote{As pointed out to us by a reviewer, \citeauthor{Haider:97} (\citeyear{Haider:97}: 24) presents the example in (\ref{verdrei}) and suggests that some operators require the verb to be in final position to be in their semantic scope. This could be interpreted as a scopal effect of V2-movement, but \cite{Meinunger:01} shows convincingly that the data should be analyzed as a syntactic ban on stranding these operators rather than as a semantic effect of V2-movement.

\ea
\ea[\label{verdrei}]{
\gll Der Wert hat sich weit mehr als blo{\ss} verdreifacht.\\
the value has itself far more than merely tripled\\
\glt `The value has far more than merely tripled'}
\ex[*]{\textit{Der Wert verdreifachte sich weit mehr als blo{\ss}.}}
\z
\zlast

}

\begin{exe}
\ex[*]{
\gll Alex mit-gebracht gestern einen Freund hat.\\
Alex along-brought yesterday a friend has\\ \label{V2-nonfinite}}
\end{exe}

If the fronted verb is a particle verb, the particle cannot be fronted together with the verb, 
see (\ref{V2-particle-final}) and (\ref{V2-particle-fronted}).%
\footnote{We are grateful to a reviewer for bringing up data in which a particle immediately precedes a fronted finite verb, see the example in (\ref{gut klar}) taken from \citeauthor{Mueller:05} (\citeyear{Mueller:05}: 14), and, therefore, could be mistaken as counterexamples to the generalization stated above. As \cite{Mueller:05} shows, however, these data are best analyzed with the particle inside the \textit{Vorfeld} and, therefore, are compatible with the generalization.
%\marginpar{Richtige Seitenzahl raussuchen!!}

\ea
\gll {\ldots} gut \textit{klar} \textit{komm} ich nicht. \\
{} good clear come I not \\
\glt `\ldots\@ I am not coping well.' \label{gut klar}
\zlast

}

\ea
\ea[\label{V2-particle}]{
\gll Alex bringt morgen einen Freund mit. \\
Alex brings tomorrow a friend along \\
\glt `Alex will bring along a friend tomorrow.' \label{V2-particle-final}}
\ex[*]{
\gll Alex mit-bringt morgen einen Freund. \\
Alex along-brings tomorrow a friend\\
}\label{V2-particle-fronted}
\z
\z

\is{movement!verb second|)}

\subsubsection{\textit{Vorfeld} placement}\label{sec-VorfeldPlacement}

\is{movement!topicalization|(}
In a number of \il{German} German clause types, including declarative main clauses, the fronted verb is preceded by a constituent. This constituent appears in the so-called \textit{Vorfeld} `prefield'. \cite{Frey06-Contrast} argues that there are three ways that a constituent can end up in the \textit{Vorfeld}.

\begin{enumerate}
\item \textit{Formal movement}: The \textit{Vorfeld}-constituent has the same intonational and pragmatic properties that it would have at the beginning of a V-final clause. This covers pragmatically unmarked subjects, including expletives as in (\ref{vf-rain}) and (\ref{vf-expl-es}), as well as aboutness topics. Formal movement is clause-bounded.

\item \textit{Base generation}: This option is available for a small number of adverbials only. The \textit{Vorfeld}-\textit{es} in (\ref{vf-vf-es}) probably falls into this class.

\item \textit{Ā-movement}: The \textit{Vorfeld}-constituent is moved from one of a variety of positions. This movement is potentially unbounded. The moved constituent is stressed and receives a contrastive interpretation.
\end{enumerate}

The \textit{Vorfeld}-constituent can be of any syntactic category and grammatical function. Examples (\ref{vf-rain}) and (\ref{vf-expl-es}) illustrate that it can also be an expletive, i.e.\@ it need not make an independent semantic contribution. Even the \textit{Vorfeld}-\textit{es}, an expletive that is not even a dependent of the clause, is allowed, see (\ref{vf-vf-es}) from \citeauthor{Mueller:13} (\citeyear{Mueller:13}: 174).\is{expletive pronoun}

\begin{exe}\label{vf-expletive}
\ex
\begin{xlist}
\ex{
\gll Es hat geregnet.\\
it has rained\\
\glt `It rained.'\label{vf-rain}
}
\ex{
\gll Es scheint, dass Alex schl\"aft.\\
it seems that Alex sleeps\label{vf-expl-es}\\
}
\ex
\begin{xlist}\label{vf-vf-es}
\ex{
\gll Es kamen drei M\"anner herein.\\
it came three men in\\
\glt `Three men came in.'
}
\ex{
\gll dass (*es) drei M\"anner herein-kamen\\
that (it) three men in-came\\
}
\end{xlist}
\end{xlist}
\end{exe}

\cite{Fanselow:04} argues that \il{German} German allows for what he calls \is{movement!pars-pro-toto@\textit{pars-pro-toto} movement} \textit{pars-pro-toto} movement, where only part of a contrastively interpreted constituent is moved into the \textit{Vorfeld}. He provides the example in (\ref{pulli-q}) (\citeauthor{Fanselow:04} \citeyear{Fanselow:04}: 12) and argues that the question can equally well be answered by (\ref{pulli}) or (\ref{alternative-answer}). In either case, the focus is on both the dative object and the verb, even though in (\ref{pulli}) it is only the dative object that occurs in the \textit{Vorfeld}.

\begin{exe}
\ex Was ist mit dem Buch passiert? \quad `What happened to the book?' \label{pulli-q}
\begin{xlist}
\ex{
\gll Meiner FREUNDIN hab ich 's geschenkt.\\
my.\textsc{dat} girlfriend have I it given\\
\glt `I gave it to my girlfriend as a present.' \label{pulli} 
}
\ex {\textit{\textnormal{[}Meiner Freundin geschenkt\textnormal{]} hab ich's.} \label{alternative-answer}}
\end{xlist}
\end{exe}

\is{movement!topicalization|)}

\subsubsection{Passive}
\is{passive|(}

Just like V2-movement and \textit{Vorfeld}-placement, passivization has no effect on the truth conditions of a sentence. In contrast to the previous two, however, the passive does not mark the clause type. In German, just as in English, verbs that take an accusative complement usually passivize. The complement becomes the subject, and the subject becomes an optional oblique complement, see (\ref{passive-default}). In contrast to \ili{English}, however, \il{German} German also allows for the passivization of intransitive verbs, see (\ref{passive-intrans}), and of verbs that take non-accusative complements, see (\ref{passive-helfen}). All of these examples are taken from \citeauthor{Mueller:13} (\citeyear{Mueller:13}: 287--288).

\ea
\gll Karl \"offnet das Fenster. {\qquad $\longrightarrow$ } Das Fenster wird (von Karl) ge\"offnet.\\
Karl opens the window {} the window is (by Karl) opened\\
\glt `Karl is opening the window.' {\hfill} `The window is being opened (by Karl).' \label{passive-default}
\z

\begin{exe}
\ex
\begin{xlist}
\ex{
\gll Hier wird getanzt.\\
here is danced\\
\glt `People are dancing here.' \label{passive-intrans}
}
\ex{
\gll Dem Mann wird geholfen.\\
the.\textsc{dat} man is helped\\
\glt `The man is being helped.'\label{passive-helfen}
}
\end{xlist}
\end{exe}

In \il{German}  German, passivization is only possible for verbs that have a \textit{referential} subject. Consequently, verbs with an \is{expletive pronoun} \textit{expletive} subject, see (\ref{passive-regnen}) from \citeauthor{Mueller:13} (\citeyear{Mueller:13}: 293), or no subject at all, see (\ref{passive-grauen}) from \citeauthor{Mueller:13} (\citeyear{Mueller:13}: 295), do not passivize.

\begin{exe}
\ex[*]  
{\gll Heute wurde geregnet.\\
today was rained\\ \label{passive-regnen}}
\end{exe}

\begin{exe}
\ex\label{passive-grauen}
\begin{xlist}
\ex[]{
\gll Dem Student graut vor der Pr\"ufung.\\
the.\textsc{dat} student is.terrified of the.\textsc{dat} exam\\
\glt `The student is terrified by the exam.'
}
\ex[*]{
\gll Dem Student wird (vom Professor) vor der Pr\"ufung gegraut.\\
the.\textsc{dat} student is (by.the professor) of the.\textsc{dat} exam terrified\\ 
}
\end{xlist}
\end{exe}

\citeauthor{Mueller:13} (\citeyear{Mueller:13}: 289) provides the example in (\ref{ankommen}) to show that unaccusative verbs usually do not passivize.%
\footnote{\label{fn-unaccusative-passive}In those cases where unaccusative verbs do passivize, a special pragmatic effect is achieved.
\citeauthor{Mueller:13} (\citeyear{Mueller:13}: 305) illustrates this point with the example in (\ref{directive-ankommen}), which can be used to express a gene\-rally valid rule.

% Change footnote counter header
\begingroup
\newcounter{saveequation}
\setcounter{saveequation}{\value{equation}}
\setcounter{equation}{0}
\renewcommand{\thexnumi}{\roman{xnumi}}
% Change footnote counter end

\ea
\gll Hier wird nicht an-gekommen, sondern nur ab-gefahren.\\
here is not on-come but only away-driven\\
\glt `One doesn't arrive here but only depart.' \label{directive-ankommen}
\z

This special pragmatic effect makes passivization possible in cases that otherwise seem completely out, such as with \textit{haben} `have':

\ea
\gll Hier wird keine Angst gehabt.\\
here is no fear had\\
\glt `Nobody is afraid here.' / `You'd better not be afraid!'
\zlast
% Change footnote counter footer
\setcounter{equation}{\value{saveequation}}
\endgroup
% Change footnote counter footer end
}

\ea
\gll Der Zug kam an. $\longrightarrow$ * Hier wurde angekommen.\\
the train came on {} {} here was arrived\\
\glt `The train arrived.'
\label{ankommen}
\z

Overall, we follow \cite{Mueller:13} and describe the 
\il{German}German passive as  demotion of a referential subject.

\is{passive|)}

\subsection{English}
\label{Sec-RestrictionsEnglish}


We will now turn to \is{cross-linguistic parallelism} parallel constructions in \ili{English} and show that there are far stronger restrictions on fronted elements in English than in \il{German} German. V2-like verb movement in English is restricted to auxiliaries. Since we do not know of any English SNDIs with an auxiliary, we will leave verb movement aside and focus on \is{movement!topicalization} topicalization and passivization.%
\footnote{Another potentially relevant construction is locative inversion, see (\ref{en-inversion}). It involves a fronted non-subject and a verb that precedes the subject:

\begin{exe}%\label{en-inversion}
\ex 
\textit{Beneath the chin lap of the helmet sprouted black whiskers.} (\citeauthor{Ward:Birner:94} \citeyear{Ward:Birner:94}: 7)\label{en-inversion}
\end{exe}

Just as for subject-auxiliary inversion, there are very strong restrictions on the type of verb that may occur in this construction. In addition, there are strong discourse requirements. Again, we did not find an SNDI that would be a candidate for this construction, which is why we will not take it into consideration here.}



\subsubsection{Topicalization}
\is{movement!topicalization|(}

Topicalization is illustrated in (\ref{en-topicalization}) from %\citeauthor{Birner:Ward:98} (\citeyear{Birner:Ward:98}: 31).
\citeauthor{Ward:Birner:94} (\citeyear{Ward:Birner:94}: 5).

\begin{exe}
\ex\label{en-topicalization} 
%\textit{In New York, there is always something to do.}
GW: \textit{Have you finished the article yet?}\\
MR: \textit{The conclusion I still have to do.}
\end{exe}

\cite{Ward:Birner:94} argue that, in English, one of the requirements of topicali\-zation is that the meaning of the fronted constituent be (linked to) discourse-old information.

Contrary to German, \ili{English} also lacks \textit{pars-pro-toto} fronting. The English equivalent of (\ref{pulli}) is not a felicitous answer to a question like \textit{What happened to the book?} because the fronted constituent is not linked to the previous context and English does not allow to interpret the fronted constituent just as a ``pars'' to a larger ``toto'' that would include the verb.

\begin{exe}
\ex \textit{What happened to the book?} \quad
\# \textit{To my girlfriend, I gave it.}\label{en-pars}
\end{exe}

Yet another observation is important for our purpose. Reflexive pronouns can only be fronted if they are used contrastively, as in (\ref{refl-contrast}). The reflexive complement of an inherently-reflexive predicate such as \textit{perjure} cannot be used to mark a contrast. Consequently, it cannot be fronted, see (\ref{refl-inherent}).\is{expletive pronoun}

\begin{exe}
\ex\label{reflexive}
\begin{xlist}
\ex []{\textit{Herself Alex watched in the mirror, not Chris.}\label{refl-contrast}}
\ex [*]{\textit{Herself Alex perjured.} \label{refl-inherent}}
\end{xlist}
\end{exe}

We will interpret this as an indication that a \is{movement!topicalization} topicalized constituent needs to make an independent contribution to the clause in which it is contained.%
\footnote{A reviewer points out that fronting reflexive arguments of inherently-reflexive verbs is highly restricted in \il{German} German as well. A bare reflexive complement of an inherently-reflexive verb cannot occur in the \textit{Vorfeld}, see (\ref{schaem-vf}) from \citeauthor{Mueller:99} (\citeyear{Mueller:99}: 99--100), but if such a reflexive pronoun is contained in an argument-marking prepositional phrase, fronting is possible, see (\ref{schlepp-vf}), which is parallel to an example from M\"uller. There is consensus, shared also by \citeauthor{Mueller:99} (\citeyear{Mueller:99}: 387), that the contrast in (\ref{sich-vf}) is due to a prosodic constraint, namely that unstressable expressions cannot be moved to the \textit{Vorfeld}. These do not only include bare inherently-reflexive pronouns but also accusative \textit{es} `it', see (\ref{es-vf}).\is{expletive pronoun}\is{intonation}

\begin{exe}
\ex\label{sich-vf}
\begin{xlist}
\ex[*]{
\gll \textnormal{[NP:} \textit{Sich}\textnormal{]} hat Peter gesch\"amt.\\
{} himself has Peter be.ashamed.of\label{schaem-vf}\\
\glt Intended: `Peter was ashamed of himself.'
}
\ex [\label{schlepp-vf}]{
\gll \textnormal{[PP:} Mit sich\textnormal{]} {} schleppt der junge Mann einen Korb \ldots\\
{} with himself {} drags the young man a basket\\
\glt `The young man is dragging a basket \ldots'\\
%\tiny{
https://filmchecker.wordpress.com/2013/12/13/filmreview-basket-case-1982.\\Accessed 2016-02-11.
%}
}
\ex[*]{
\gll Es haben die Kinder lesen m\"ussen.\\
it.\textsc{acc} have the children read must\\
\glt Intended: `The children had to read it.'\label{es-vf}
}
\end{xlist}
\end{exe}

}
 
\is{movement!topicalization|)}

\subsubsection{Passivization}
\is{passive|(}

\citeauthor{Kuno:Takami:04} (\citeyear{Kuno:Takami:04}: 127) 
argue that subjects of \ili{English} passives are topics. Consequently, they need to be able to refer to entities in the discourse, ideally to entities that are either introduced in the previous discourse or can be inferred from it. \cite{Ward:Birner:04} characterize passive subjects as being relatively discourse-old, i.e.\@ at least not the discourse-newest element in the clause.

\cite{kaysagidioms} provide the examples in (\ref{expl-pass}) to show that expletives can occur as subjects of passive sentences.\is{expletive pronoun}

\begin{exe}
\ex\label{expl-pass}
\begin{xlist}
\ex \textit{There was believed to be another worker at the site besides the neighbors
who witnessed the incident.}\label{expl-pass-a}
\ex \textit{It was rumored that Great Britain, in apparent violation of the terms of
the Clayton-Bulwer treaty, had taken possession of certain islands in the
Bay of Honduras.} \label{expl-pass-b}
\end{xlist}
\end{exe}

If expletives have an empty semantics, this would contradict the observations from \cite{Kuno:Takami:04} and \cite{Ward:Birner:04}. \cite{kaysagidioms}  do not provide any context, so we can only check on the observation from \cite{Ward:Birner:04} that the subject is not the newest element in the sentence. We make the plausible assumption that the \is{expletive pronoun} expletive subject is co-indexed with a post-verbal constituent, namely the NP \textit{another worker} in (\ref{expl-pass-a}) and the extraposed \textit{that}-clause in (\ref{expl-pass-b}). Consequently, the expletive is at best as discourse-new as the post-verbal constituent, which satisfies the constraint.

\is{passive|)}

\section{Analysis}
\label{Sec-Analysis}
\largerpage[-2]
We will first provide the basic idea of our analysis and then show that it allows us to derive the syntactic flexibility of SNDIs in a natural way.



\subsection{A redundancy-based semantic analysis}

\is{lexicalism|(}
The picture that emerged from the discussion in §\ref{Sec-SyntacticSemanticFlexibility} was that the dif\-ference in the syntactic encoding of SNDIs and \is{multiword expression!decomposable} semantically decomposable idioms is questionable. We will propose an encoding of SNDIs in terms of individual word-level lexical entries and, based on the discussion in §\ref{Sec-ConstructionSpecificRestrictions}, derive the restrictions on their syntactic flexibility from the interaction of this encoding with the language-specific properties of the relevant syntactic constructions. This is also the position taken in \cite{kaysagidioms}, which, however, is exclusively based on \ili{English} data.

There are at least two major challenges for any analysis of idioms in terms of individual word-level lexical entries. First, a mechanism is needed to ensure the co-occurrence of the idiom's components. 
We will call this the \emph{collocational challenge}.
\is{multiword expression!collocation} 
Second, if the idiom's syntactic components combine according to the conventional rules of combinatorics, the idiom's semantics should equally emerge through the conventional mechanism of combinatorial semantics. We will call this the \emph{compositional challenge}.\is{semantics!compositionality|(}

Any approach based on the insights of NSW has presented a solution to the collocational challenge. Within 
\is{Head-driven Phrase Structure Grammar} \emph{Head-driven Phrase Structure Grammar}, for example, this is usually done by some sort of extended selectional mechanism (\citealt{Krenn:Erbach:94, Soehn:Sailer:03, Sag:07hpsg, kaysagidioms}), but more powerful collocational systems have also been used (\citealt{Riehemann:01,Sailer:diss,Soehn:diss}). Common to all of these approaches is a proliferation of lexical entries. The word \textit{kick}, for example, has lexical entries for its literal and for its idiomatic meanings. We will share this assumption and not elaborate on the collocational challenge any further -- for such an elaboration, see, for example, the analysis of 
\is{multiword expression!decomposable} semantically decomposable idioms in \cite{Webelhuth:al:14}.

What we will focus on here is the compositional challenge, which has played a major role in making the phrasal analysis of SNDIs so attractive. 
If there is no evidence that parts of an SNDI make an individual meaning contribution, why not just assign the idiom \isi{meaning} to the phrase instead of its words? 
In light of the data on the syntactic flexibility of SNDIs, however, such an analysis is not easily tenable.

\is{Sign-Based Construction Grammar|(}\is{semantics!underspecification|(}
\cite{kaysagidioms} assign the entire meaning of an SNDI to its syntactic head. 
Such a suggestion is very natural within a head-driven syntax. 
To the other words within the idiom, \citet{kaysagidioms} assign an empty semantic contribution.%
\footnote{The earliest reference to such an approach seems to be \cite{Ruhl:75}. 
Unfortunately, we could not get a copy of this paper. 
NSW explicitly reject this type of approach as failing to account for the syntactic fixedness of SNDIs.}
They achieve this by working within \is{Minimal Recursion Semantics}\is{semantics!Minimal Recursion Semantics} \emph{Minimal Recursion Semantics} (\citealt{Copestake:al:95,copestake2005mrs}), where semantic representations are encoded as lists of simple predicate-argument expressions and subordination constraints among these. 
An empty semantic contribution is simply encoded as an empty list.

This analysis is sketched in (\ref{ks-kickthebucket}). We distinguish the idiom-internal \textit{kick} from its literal homonym by representing the former as \textit{kick$_{id}$}. We proceed analogously for the other words. The semantic representation of \textit{kick$_{id}$} consists of the predicate $\mathbf{die_{id}}$, a situation $s$, and the index of the subject: $x$.

\begin{exe}
\ex Semantic analysis of \textit{kick the bucket} \`a la \cite{kaysagidioms}\label{ks-kickthebucket}
\begin{xlist}
\ex \textit{kick}$_{id}$: $\left< \mathbf{die_{id}}(s,x) \right>$
\ex \textit{the}$_{id}$: $\left<  \right>$
\ex \textit{bucket}$_{id}$: $\left<  \right>$
\end{xlist}
\end{exe}

\cite{kaysagidioms} derive the right semantics for the idiom and thereby solve the compositional challenge. They also account for the absence of an internal modification reading, as the noun \textit{bucket$_{id}$} does not make any semantic contribution that could be modified. The semantic emptiness of \textit{bucket$_{id}$} is also made responsible for the fact that \is{movement!topicalization} topicalization is not possible with \textit{kick the bucket}, as topicalization requires the topicalized constituent to be non-empty. 

\is{passive}


\largerpage[-1]
In the  light of the examples in (\ref{expl-pass}), \cite{kaysagidioms} do not impose a non-emptiness constraint on passive subjects. Instead, they
classify the idiomatic verb \textit{kick$_{id}$} as belonging to a verb class that does not allow for passivization.



%\cite{kaysagidioms} derive the right semantics for the idiom and thereby solve the compositional challenge. They also account for the absence of an internal modification reading, as the noun \textit{bucket$_{id}$} does not make any semantic contribution that could be modified. The semantic emptiness of \textit{bucket$_{id}$} is also made responsible for the fact that neither \is{topicalization} topicalization nor passivization is possible with \textit{kick the bucket}, as topicalization requires the topicalized constituent to be non-empty, and  \is{passive} passivization has such a requirement for the verb's direct object.%
%\footnote{Under standard assumptions in HPSG, any properties imposed on the complement of the active verb would be inherited by the subject of the passive verb. This would exclude expletive pronouns as subjects of passive sentences, but \cite{kaysagidioms} show with examples like (\ref{expl-pass}) above that such cases exist. It is not clear to what extent \cite{kaysagidioms} subscribe to these HPSG assumptions.}


\is{semantics!Lexical Resource Semantics|(}
While this analysis already goes a long way in what we consider the right direction, we think that a slightly different answer to the compositional challenge might get us even further. Instead of empty semantic contributions for the words \textit{bucket$_{id}$} and \textit{the$_{id}$}, we assume \emph{redundant} semantic contributions and make use of \is{Lexical Resource Semantics} \emph{Lexical Resource Semantics} (LRS, \citealt{Richter:Sailer:04}). Within this framework, \cite{Richter:Sailer:01.1,Richter:Sailer:06} argue that the co-occurrence of words that contribute the same semantic operator (such as question or negation) is common in the languages of the world and, therefore, should be analyzed that way. \cite{Sailer:10} extends this argument to lexical semantic contributions in his analysis of the \ili{English} cognate object construction. The semantic contributions of \is{linguistic sign}signs used in these works are list-based, just as in \cite{kaysagidioms}. In contrast to \cite{kaysagidioms}, however, the different lists may contain identical elements. Another dif\-ference is that the elements on the semantic contribution list need not be predicate-argument expressions but can be of any form.
\is{Sign-Based Construction Grammar|)}

Our analysis of \textit{kick the bucket} is sketched in (\ref{lrs-kickthebucket}), where we indicate the lexical semantic contributions of the idiom's words. 

\begin{exe}
\ex Redundancy-based semantic analysis of \textit{kick the bucket}:\label{lrs-kickthebucket}
\begin{xlist}
\ex \textit{kick}$_{id}$: $\left< s, \mathbf{die_{id}}, \mathbf{die_{id}}(s,\alpha), \exists s (\beta) \right>$\label{lrs-kick}
\ex \textit{the}$_{id}$: $\left< s, \exists s (\beta) \right>$\label{lrs-the}
\ex \textit{bucket}$_{id}$: $\left< s, \mathbf{die_{id}}, \mathbf{die_{id}}(s,\alpha) \right>$\label{lrs-bucket}
\end{xlist}
\end{exe}

The verb \textit{kick$_{id}$} contributes a situation $s$, the predicate $\mathbf{die_{id}}$, and the formula that combines this predicate with its two arguments -- one of them being the situation $s$. The second argument of $\mathbf{die_{id}}$ is left \is{semantics!underspecification} underspecified, as its semantics will come from the subject. This under\-specification is indicated with a lower-case Greek letter, here $\alpha$, which is used as a meta-variable over expressions of our semantic representation language. The verb also contributes an existential quantification over the situational variable: $\exists s (\beta)$. The meta-variable $\beta$ indicates that the scope of the quantifier is underspecified. 

In other words, \textit{kick$_{id}$} contributes the same kinds of elements as other verbs. Similarly, the semantic contribution of the determiner \textit{the$_{id}$} is just like that of a normal determiner. It contributes a variable and a quantification over this variable. The noun \textit{bucket$_{id}$}, just like other common nouns, contributes a referential variable and a predicate.
 
\newpage 
While the semantic contributions of the idiomatic words in (\ref{lrs-kickthebucket}) are analogous to those of non-idiomatic words, it can be seen that the contributions of \textit{the}$_{id}$ and \textit{bucket}$_{id}$ are contained in the contribution of \textit{kick}$_{id}$.%
\footnote{Technically, this effect can be achieved through selection. The selecting verb requires its complement to have the same index and to contribute the same constant: $\mathbf{die}_{id}$.}
%
This is what we refer to as \emph{redundant marking}.

When words combine to form a phrase, their meaning contributions are collected, i.e.\@ the list of semantic contributions of a phrase contains all the elements of its daughters' lists. 
For the sentence \textit{Alex kicked$_{id}$ the$_{id}$ bucket$_{id}$}, the semantic contribution list will contain all the elements listed in (\ref{lrs-kickthebucket}) plus the contribution of the word \textit{Alex}, which is just the constant $\mathbf{alex}$.

At the sentence level,  all the elements of this list must be combined into a single formula. 
To do this, each meta-variable must be assigned an element from the contribution list as its value. In our case, $\alpha$ would be assigned $\mathbf{alex}$, which results in $\mathbf{die}_{id}(s,\mathbf{alex})$. 
This formula is taken as the value of the meta-variable $\beta$. 
This leads to the intended semantic representation of the sentence: $\exists s (\mathbf{die_{id}}(s,\mathbf{alex}))$. 
The constant $\mathbf{die_{id}}$ occurs only once in this logical form, even though it is contributed by two words in the sentence -- \textit{kick}$_{id}$ and \textit{bucket}$_{id}$.
\is{semantics!underspecification}

The redundancy-based analysis of \textit{kick the bucket} will directly carry over to other SNDIs, be it in English or in other languages. In our case, the same semantic contributions would be assumed for the words in the \il{German} German idiom \textit{den L\"offel abgeben} `die'.\is{semantics!compositionality|)}
\is{semantics!Lexical Resource Semantics|)}

In the next two subsections, we will look more closely at the syntactic flexi\-bility of SNDIs. We will show that the attested behavior follows directly from the interaction of the proposed analysis of SNDIs and the construction-specific constraints presented in §\ref{Sec-ConstructionSpecificRestrictions}. We will also show some advantages of the redundancy-based approach over the one of \cite{kaysagidioms}.
\is{lexicalism|)}
 
\subsection{Syntactic flexibility of German SNDIs}
\label{Sec-AnalysisGerman}

\is{multiword expression!flexible} We will go through the three phenomena of \il{German}  German syntax discussed in §\ref{Sec-RestrictionsGerman} and look at them in the light of SNDIs.

 
\subsubsection{German SNDIs and V2-movement}
\il{German} 

\is{multiword expression!flexibility|(}
\is{multiword expression!non-decomposable|(}
\is{movement!verb second|(}
The restrictions on V2-movement are syntactic in nature and do not at all depend on the content of the verb. We hence expect that these constraints hold for the verbs in SNDIs. This is borne out. With \textit{den L\"offel abgeben}, for example, which contains a verb with the separable particle \textit{ab}, a non-finite verb following the \textit{Vorfeld} is ungrammatical, see (\ref{v2-loeffel-inf-b}), and so is fronting the finite verb together with the particle, see (\ref{v2-loeffel-abgab}).%
\footnote{\label{fn-pferd}There are idioms where the verb must be in V2-position. \citeauthor{Richter:Sailer:09} (\citeyear{Richter:Sailer:09}: 300) claim that the idiom in (\ref{pferd}) has a fixed \textit{Vorfeld} element followed by the finite form \textit{tritt}. We think that this is due to the fact that this is an idiom with a ``pragmatic point'' (\citealt{FillmoreEtAl1988}) and, thus, a certain illocutionary force is part of the idiom, which is not compatible with a V-final clause. 

\begin{exe}
\ex\label{pferd}
\begin{xlist}
\ex[]{
\gll Ich glaub, mich tritt ein Pferd!\\
I believe me.\textsc{acc} kicks a horse\\
\glt `I am very surprised.' / `I can't believe this!' 
%(\citealt{Richter:Sailer:09})
}
\ex [\#]{\textit{Ich glaub, dass mich ein Pferd tritt.}}
\end{xlist}
\end{exe}

}

\begin{exe}
\ex\label{v2-loeffel-inf}
\begin{xlist}
\ex[]{
\gll Alex hat den L\"offel ab-gegeben.\\
Alex has the spoon on-passed\\
}
\ex[*]{\textit{Alex ab-gegeben den L\"offel hat.}\label{v2-loeffel-inf-b}}
\end{xlist}
\end{exe}

\begin{exe}
\ex\label{v2-loeffel}
\begin{xlist}
\ex[]{
\gll Alex gab den L\"offel ab.\label{v2-loeffel-gab}\\
Alex passed the spoon on\\
\glt `Alex died.'
}
\ex[*]{\textit{Alex ab-gab den L\"offel.}\label{v2-loeffel-abgab}}
\end{xlist}
\end{exe}

\is{movement!verb second}

\subsubsection{German SNDIs and \textit{Vorfeld} placement}
\il{German} 


\is{movement!topicalization|(}
As we saw in §\ref{sec-VorfeldPlacement}, there are three possibilities for a constituent to be licensed in the \textit{Vorfeld}: formal movement, base generation, and Ā-movement for contrast. \cite{Fanselow:04} provides examples of \emph{Vorfeld} placement of constituents of SNDIs. One of his examples is given in (\ref{hungertuch}) (from \citeauthor{Fanselow:04} \citeyear{Fanselow:04}: 22), where the PP-constituent of the idiom \textit{am Hungertuch nagen} `be very poor' (lit.:\,`gnaw at the hunger cloth') is fronted. The sentence has a contrastive interpretation; the alternatives are various degrees of poorness.

\ea
\gll Am Hunger-tuch m\"ussen wir noch nicht nagen.\\
on.the hunger-cloth must we yet not gnaw\\
\glt `We are not down on our uppers, yet.'\label{hungertuch}
\z

\newpage 
When we apply these considerations to \textit{den L\"offel abgeben}, we see that in an active sentence, fronting the NP \textit{den L\"offel} should be unproblematic under a contrastive reading.%
\footnote{For the non-contrastive case, we find clause-initial placement of the \textit{L\"offel}-NP in V-final clauses, at least in the \is{passive} passive. This shows that the idiom-internal NP can be fronted by formal movement.

\begin{exe}
\ex \label{zu machen}
Da ist nichts mehr zu machen. \textnormal{`Nothing can be done anymore.'}
\begin{xlist}
\ex{
\gll Es sieht so aus, also ob \textnormal{[}der L\"offel jetzt endg\"ultig ab-gegeben ist\textnormal{]}.\\
it looks so out as if the spoon now definitively on-passed is\\
\glt `It looks like it is definitely over now.'
}
\ex
 \textit{Der L\"offel ist jetzt endg\"ultig ab-gegeben.}\\
\textnormal{`It is definitely over now.'}
\end{xlist}
\end{exe}
}


This is shown in (\ref{loeffel-kontrast}), where the alternatives are other consequences of serious illness.

\ea
\gll Es sind zwar viele schwer krank geworden, den L\"offel hat aber noch niemand ab-gegeben.\\
it are admittedly many heavy sick become, the spoon has but still nobody on-passed\\
\glt `Though many got seriously sick, nobody has died yet.'\label{loeffel-kontrast}
\z

These contrastive cases clearly distinguish between our analysis and that of \cite{kaysagidioms}. Since the NP \textit{den L\"offel} contributes the same situational variable as the verb \textit{abgeben}, it is easy to know to which larger ``toto'' the fronted ``pars'' belongs. In an analysis with an empty semantics of the NP, this would not be possible.
\is{movement!topicalization|)}


\subsubsection{German SNDIs and the passive}
\is{passive|(}

We expect the passivizability of SNDIs to follow from 
the interaction between the above analysis and
the general properties of the German passive discussed in
%the above analysis and from the general properties of the \il{German} German passive discussed 
in §\ref{Sec-RestrictionsGerman}. 
The German passive voice demotes the subject of an active clause. In our analysis, a passive verb requires that there be a participant filling the thematic role of the active subject and that this subject have a non-redundant index.% 
\footnote{A bit more technically, the index of the active subject must not be identical with the index of the active verb or of any of the verb's arguments. 
This restriction does not seem to be valid for German only, but can be used to derive the ungrammaticality of \textit{*Alex$_i$ was shaved by himself$_i$}. 
A reviewer pointed out that a reflexive pronoun is possible in a \textit{by}-phrase in a context that evokes alternatives to the reflexive pronoun, such as \textit{Chris was shaved by Alex and Alex was shaved by himself}. 
This exception is clearly connected to a special semantics to which our non-redundant index requirement would need to be adapted.}
%
There are additional restrictions on verbs that cannot be passivized or only with the special pragmatic effect mentioned in Footnote \ref{fn-unaccusative-passive}.

\cite{Dobrovolskij:00}
argues that a VP-idiom, \is{multiword expression!decomposable} semantically decomposable or not, can never be passivized if the literal counterpart of the idiom's verb cannot be passivized. 
His example is the semantically decomposable idiom \textit{einen Korb bekommen} `get the brush-off' (lit.:\,`receive a basket'), which can neither be passivized in its literal nor in its idiomatic reading.

Idioms with an \is{expletive pronoun} expletive subject do not passivize either. An example is \textit{Bindf\"aden regnen} `rain heavily' (lit.:\,`rain strings'), see (\ref{Bindfaeden}).

\begin{exe}
\ex[*]{
\gll Hier werden/wird Bindf\"aden geregnet. \label{Bindfaeden}\\
here are/is strings rained\\}
\end{exe}


This is expected under our analysis. The \is{semantics!Lexical Resource Semantics}\is{Lexical Resource Semantics} LRS analysis of expletives is redun-dancy-based. For weather verbs, \cite{Levine:al:LRS} assume that the \is{expletive pronoun} expletive subject has the same index as the verb. Consequently, the sentence in (\ref{Bindfaeden}) violates the constraint that the demoted subject must not have a redundant index.

A reviewer brought the example in (\ref{nass-pass}) to our attention. \citeauthor{Mueller:02} (\citeyear{Mueller:02}: 131) points out that if (\ref{nass-akt}) is the active counterpart of (\ref{nass-pass}), one is forced to allow the weather-\textit{es} to be the underlying subject of a passive. This might undermine the explanation for blocking (\ref{Bindfaeden}).

\begin{exe}
\ex\label{nass}
\begin{xlist}
\ex{
\gll Die St\"uhle wurden nass geregnet.\label{nass-pass}\\
the chairs were wet rained\\
\glt `The rain caused the chairs to become wet.'
}
\ex{
\gll Es hat die St\"uhle nass geregnet.\label{nass-akt}\\
it has the chairs wet rained\\
}
\end{xlist}
\end{exe}

Our semantic-based constraint on passivization does not run into this problem. We give a very rough sketch of the logical form of (\ref{nass}) in (\ref{nass-lf}). This formula can be paraphrased as in the following sentence. There are the eventualities $s$, $s'$, and $s''$, such that $s$ is a raining event, $s'$ is a state with wet chairs, and $s''$ is a causation event in which the raining $s$ causes the wetness $s'$. 

\begin{exe}
\ex \label{nass-lf}
 $\exists s $ $\exists s'$ $ \exists s''$ $(\mathbf{rain}(s) \wedge \mathbf{wet}(s',\mathbf{the\mbox{\textbf{-}}chairs}) \wedge \mathbf{cause}(s'',s,s'))$
\end{exe}


Following the syntactic analysis in \citeauthor{Mueller:02} (\citeyear{Mueller:02}: 241), the resultative version of \textit{regnen} comes about by a lexical rule that changes the verb's valence requirement and adds the semantic material required for the causation/result semantics. When one adapts this rule to LRS, it also changes the index of the verb from the raining event to the causation event. Consequently, resultative \textit{regnen} in (\ref{nass}) has the index $s''$ in (\ref{nass-lf}), whereas the raining -- and, by redundancy, the expletive \textit{es} -- has the index $s$. Since the underlying active subject and the passivized verb have distinct indices under this analysis, the grammaticality of (\ref{nass-pass}) is predicted. Note that this analysis, again, is possible under a redundancy analysis of expletives but hard to implement if one assumes an empty semantics for expletives.\is{expletive pronoun}

As for verbs allowing for passivization, \citeauthor{Dobrovolskij:00} (\citeyear{Dobrovolskij:00}: 561) distinguishes between idioms with idiom-external accusative objects, as in (\ref{Eis}), and those with idiom-internal accusatives, as in his example in (\ref{garaus}). For the former, there is no idiom-specific restriction on passivization.

\begin{exe}
\ex \textit{etwas auf Eis legen} `put something on hold'\label{Eis}\\
\gll Das Projekt wurde auf Eis gelegt.\\
the project was on ice put\\
\glt `The project was put on hold.'
\end{exe}

\begin{exe}
\ex\label{garaus} \textit{jemandem den Garaus machen} `kill someone'\\
\gll {\ldots} den l\"astigen Hausgenossen soll nun {\textit{\ldots}}  der Garaus gemacht werden \ldots\\
{} the.\textsc{dat} annoying housemates should now {}  the.\textsc{nom} Garaus made be\\
\glt `\ldots\@ the annoying housemates should now be killed \ldots'
\label{garaus-passive}
\end{exe}

\cite{Dobrovolskij:00} assumes that the main function of the 
\is{passive}\il{German}German passive is to promote an accusative complement. This promotion has the syntactic effect of realizing the underlying accusative complement as a subject and the semantic/pragmatic effect of assigning its referent the status of a topic. 
Based on these assumptions, he diagnoses a syntax-semantics mismatch in sentences like (\ref{garaus-passive}). 
Syntactically, he says, the idiom-internal NP is promoted, but semantically it is the idiom-external dative NP. 
In a  subject-demotion approach, no such mismatch needs to be assumed for (\ref{garaus-passive}). 
We can derive the topicality of the dative NP from the fact that it occurs in a topic position -- here, its appearance in the \emph{Vorfeld} through formal movement (see §\ref{sec-VorfeldPlacement}).

\cite{Dobrovolskij:00} only considers \is{passive} passives of transitive verbs with an agentive meaning.
Our approach does not have this limitation. We expect the passive to be possible with idioms having a non-agentive idiomatic meaning, such as \textit{den L\"offel abgeben}, for which we can indeed find examples, see   (\ref{German-passive}).

\ea
\gll Bei den Gr\"unen wird der politische L\"offel schon vor Amtsabschied ab-gegeben. \\
at the Green.party is the political spoon already before resigning {on-passed} \\
\glt `In the Green Party, people die politically already before resigning from their office.'%
\footnote{http://www.kontextwochenzeitung.de/politik/148/erst-schreien-wenn-etwas-geschafft-ist-1992.html. Accessed 2014-12-19.}
\label{German-passive}
\z

\is{movement!passive|)}
In this section, we argued that the restrictions on three syntactic processes of \il{German} German (V2-movement, fronting, and passivization) are very weak and compatible with the syntactic, semantic, and pragmatic properties of an SNDI such as \textit{den L\"offel abgeben}. We therefore expect that the idiom can occur in all of them.
\is{multiword expression!non-decomposable|)}
\is{multiword expression!flexibility|)}

\subsection{Syntactic flexibility of English SNDIs}
\label{Sec-AnalysisEnglish}

\is{multiword expression!flexibility}
\is{multiword expression!non-decomposable|(}
We saw in §\ref{Sec-RestrictionsEnglish} that \ili{English} imposes semantic constraints on  frontable constituents and on passive subjects. We will now explore the interaction of these constraints with our lexical encoding of SNDIs.

For \is{movement!topicalization} topicalization, we saw in §\ref{Sec-RestrictionsEnglish} that the topicalized constituent must be explicitly linked to the previous discourse, and that it must make an independent semantic contribution within its clause. In LRS, such a non-redundancy requirement can be expressed easily by saying that the semantic contribution of the topicalized constituent must not be properly included in the semantic contribution of the rest of the clause. In our analysis, the meaning of the NP \textit{the bucket} is fully included in the meaning of the rest of the clause. Therefore, the ban on topicalization follows directly.

Matters are slightly more complicated when we look at the \is{passive} passive voice. The constraints on a passive subject have been shown to be weaker than those on a topi\-ca\-lized constituent. We saw above that a passive subject must refer to something that has been mentioned earlier in the discourse (or that can be inferred from such an element). This does not exclude the possibility of the subject making a semantic contribution that is contained in that of the rest of the sentence -- as we saw in the cases of \is{expletive pronoun} expletive passive subjects in (\ref{expl-pass}). 

\newpage 
Consequently, if the discourse conditions \is{passive} on passive subjects are met, even \ili{English} SNDIs can be passivized. In (\ref{kick-the-bucket-passive}), repeated in (\ref{KickPassive}), \textit{kick the bucket} is topical, only the tense and the result state are new. 

\begin{exe}
\ex \textit{When you are dead, you don't have to worry about death anymore. {\ldots}
The bucket will be kicked.}\label{KickPassive}
\end{exe}

The example in (\ref{KickPassive}) is one out of admittedly few naturally occurring examples of the passive with this idiom.% 
\footnote{In a recent talk, \ia{Fellbaum, Christiane}  Christiane Fellbaum presented two other naturally occurring examples of \textit{kick-the-bucket} passives and passives of other English idioms that express the idea of ``dying''. In as far as context is included in her examples, they also satisfy the  topicality requirement. See: http://www.crissp.be/wp-content/uploads/2015/04/Talk7-Fellbaum.pdf. Accessed 2015-08-27.}
%
The following examples show \is{passive} passives for other idioms that are classified as IPs in NSW, see (\ref{logs}), or do not pass the tests for semantic decomposability, see (\ref{cow}). Example (\ref{cow}) shows particularly clearly that the \isi{meaning} of the idiom \textit{have a cow} is discourse-old, as it is explicitly mentioned in the preceding clause.%
\footnote{Note that even though the examples in  (\ref{logs}) and (\ref{cow}) may have a playful character, they do not blend the idiomatic and the non-idiomatic reading, 
as it would typically be the case in jokes or puns.
}

\begin{exe}
\ex \textit{saw logs} `snore'\label{logs}\\
\textit{I excitedly yet partially delusional turned to Alexandria to point out the sun as it set and all I see is eyelids and hear \emph{logs being sawed}. Come on! I can't say too much because I wasn't far behind as I was catching flies [= sleeping] about a minute later.}%
\footnote{http://5050experience.sportsblog.com/posts/1125677/feast.html. Accessed 2015-07-24.}
\end{exe}

\begin{exe}
\ex \textit{have a cow} `get angry'\label{cow} \\ 
\textit{There was really no need for the police to have a cow, but \emph{a cow was had}, resulting in kettling, CS gas and 182 arrests.}%
\footnote{http://www.theguardian.com/commentisfree/2012/aug/01/cyclists-like-pedestrians-must-get-angry. Accessed 2015-08-24.}
\end{exe}

An approach that assumes an empty semantics for the idiom-internal NP \textit{the bucket} runs into severe problems. We saw above that passivization is possible for SNDIs 
%but heavily restricted through strong discourse requirements.
if the strong discourse requirements are met. 
Thus, %an enforced non-passivizability of 
it would be wrong to categorically block the passivization of
\textit{kick$_{id}$}.
Our approach correctly predicts the admittedly rare occurrence of passives with this idiom.
%
Furthermore, an empty semantics for \textit{the bucket} does not allow us to relate the NP's meaning to the preceding discourse. A redundancy-based account makes the required semantic information available at the clause-initial constituent.

\newpage 
Let us conclude §\ref{Sec-Analysis} with a brief summary of our analysis. We replaced NSW's causal relation  between the semantic decomposability and the \is{multiword expression!flexible} syntactic flexibility of idioms with an approach based on  the interaction of the properties of idioms with the constraints on syntactic constructions. While, overall, our account is very similar to \cite{kaysagidioms}, an important difference is that we make use of redundant marking, a choice which we hope to have motivated above.

\is{multiword expression!non-decomposable|)}
\is{multiword expression!flexibility|)}

\section{Extension to other languages}
\label{Sec-OtherLanguages}

So far, we have only looked at 
English and German. These two closely-related languages already show considerable differences in their syntactic constructions, and these differences have far-reaching consequences for the flexibility of MWEs. In this section, we would like to briefly show that other languages have yet other constraints on similar syntactic operations and that these have a predictable effect on the flexibility of idioms.



\subsection{Estonian}
\il{Estonian|(}

\cite{Muischnek:Kaalep:10} name and describe a number of problems in applying an English-based classification of idioms to Estonian. Similar to German, Estonian allows for considerably more word-order flexibility than English. 
\citeauthor{Muischnek:Kaalep:10} (\citeyear{Muischnek:Kaalep:10}: 122) argue that Estonian has a \is{passive} passive-like construction whose function is to background a (usually human) subject, rather than to foreground an object. This is similar to the function of the passive in German. 
Consequently, passivizing intransitive verbs is possible, see (\ref{est-pass-run}).

\is{passive}
\begin{exe}
\ex \begin{tabular}[t]{llcl}
\textit{Mees} & \textit{jookseb} & $\longrightarrow$\qquad\qquad &\textit{Joostakse}\\
man& run.\textsc{present} 	&{}&	run.\textsc{impers}\\
\multicolumn{2}{l}{`The man is running.'} & 	{}	&`Somebody is running.'
\end{tabular}
%\gll Mees jookseb {\qquad $\longrightarrow$ \qquad} Joostakse\\
%man run.\textsc{present} 	{}	run.\textsc{impers}\\
%\glt `The man is running.'  	{}~~ 	`Somebody is running.'
\label{est-pass-run}
%\z
\end{exe}


In order to emphasize its subject-backgrounding function, this construction is called \emph{impersonal passive}. In contrast to German, there is no change in the morpho\-logical case of the active direct object, see (\ref{est-pass-read}). This leads us to expect that the lack of object foregrounding might be even stronger in Estonian than in German.%
\footnote{The differences between German passives and Estonian impersonal passives are discussed in detail in \cite{Blevins:03}.}

\ea
\gll Mees loeb raamatut. $\longrightarrow$ {\quad} Loetakse raamatut.\\
Man read.\textsc{present} book.\textsc{part} {} {\quad} read.\textsc{impers} book.\textsc{part}\\
\glt `The man is reading a book.' {\quad}  {\quad}  {\quad}  
\begin{tabular}[t]{@{}l}
`A book is being read';\\
`Somebody is reading a book.'
\end{tabular}\label{est-pass-read}
\z

\cite{Muischnek:Kaalep:10} state that the impersonal passive can be formed with all idioms, including SNDIs. The only condition is that the active subject be human. Kadri Muischnek (personal communication) kindly provided us with the example in (\ref{est-pass-mwe}).

\ea
\gll Kas massiliselt heideti hinge?\label{est-pass-mwe}\\
\textsc{q} massively threw.\textsc{impers} soul.\textsc{part} \\ 
\glt `Did they die massively?'%
\footnote{From the etTenTen corpus: http://www.keeleveeb.ee.}
\z

\subsection{French}
\il{Estonian|)}%to avoid vertical space put this below section title
\il{French|(}

In French, we see yet a different pattern. \cite{Abeille95} lists French idioms that do not permit internal modification but do permit the \is{passive} passive voice, such as \textit{faire un carton} `hit the bull' (lit.:\,`make a box'). These reported data suggest that French is more like German than like English when it comes to the passive. 
\cite{Lamiroy:93} provides convincing arguments that this is indeed the case. Instead of promoting a non-subject argument, the French passive also primarily demotes a subject. 
French allows for the passivization of strictly intransitive verbs, see (\ref{fr-pass-sleep}) from \citeauthor{Lamiroy:93} (\citeyear{Lamiroy:93}: 54), but not as productively as German, see (\ref{fr-pass-run}).\is{expletive pronoun}

\is{passive}
\begin{exe}
\ex\label{fr-pass}
\begin{xlist}
\ex{
\gll Il a \'et\'e dormi dans mon lit.\\
it.\textsc{expletive} has been slept in my bed\\
\glt `Someone had been sleeping in my bed.'\label{fr-pass-sleep}
}
\ex \label{fr-pass-run}
{
\gll Ils courent. {\qquad $\longrightarrow$}  * Il est fr\'equemment couru ici.\\
they run {} {} it is often run here\\
\glt `They are running.'  \qquad  `There is often someone running here.'
}
\end{xlist}
\end{exe}

We will leave the details of the \is{passive} passivizability of intransitive verbs in French aside. \cite{Gaatone:93} gives examples of passivized French SNDIs, including the one in (\ref{fr-culotte}) (see \citeauthor{Gaatone:93} \citeyear{Gaatone:93}: 47).%
\footnote{The English counterpart \textit{wear the pants} syntactically behaves like \textit{kick the bucket}. The corresponding German expression \textit{die Hosen an-haben} (lit.:\,`have the pants on') cannot be passivized since the verb \textit{haben} `have' is unpassivizable in general.} 

\ea \textit{porter la culotte} `wear the pants'\\
\gll Mme et M. Armand y r\'egnent paternellement, bien que la culotte y soit port\'ee par
madame {\ldots}\\
Mrs and Mr Armand there rule paternally even though the pants there is worn by madam\\
\glt `Mrs and Mr Armand rule there paternally even though she is the dominant part'\label{fr-culotte}
\z

\il{French|)}

In this section, we showed that our results of the German-English contrast carry over to other languages as well. Whether or not an SNDI can appear in a certain syntactic construction is dependent on the constraints on that construction in the particular language. Languages may differ significantly with regard to these constraints. For this reason, classical tests for classifying idioms, such as passivizability and fronting, cannot be easily applied across languages but need to be re-examined in each individual case.



\section{Conclusion}
\label{Sec-Conclusion}

\cite{Wasow:al:83} and \cite{Nunberg1994} have led to a shift in perspective from a monolithic, \is{lexicalism}
fully phrasal view of all idioms to a more lexical approach for  semantically decomposable idioms. We agree with \cite{kaysagidioms} in extending this lexical approach to SNDIs.%
\footnote{Parallel treatments of SNDIs and semantically decomposable idioms have recently been proposed within other frameworks as well; see a short remark in \citeauthor{Harley:Stone:13} (\citeyear{Harley:Stone:13}: fn.\,2) within a Minimalist approach and \cite{Lichte:Kallmeyer:16} for Tree Adjoining Grammar.}
%
In order to provide a solid motivation for this step, it is essential to look at a larger set of languages, in particular languages that differ in the semantic and pragmatic properties of morphosyntactically similar constructions. The present paper made a first step in that direction and looked at verb fronting,  topicalization, and passivization in German and English as well as the impersonal passive in Estonian and the passive in French. Whereas \cite{Nunberg1994} are forced to analyze English and German SNDIs in considerably different ways, the lexical analysis presented here provides a cross-linguistically uniform analysis.%
\footnote{We side with \citeauthor{Mueller:13Unifying} (\citeyear{Mueller:13Unifying}: 923), who states: ``If we can choose between several theoretical approaches, \ldots we should take the one that can capture cross-linguistic generalizations.''}

This type of analysis has consequences for the encoding of multiword expressions (MWE) in formal grammar in general. All MWEs that are of syntactically regular shape should receive a lexical encoding. The difference between  semantically decomposable and  semantically non-decomposable MWEs lies in the way in which the semantics of the MWE is distributed over the words constituting the MWE. Whereas the parts of a semantically decomposable MWE have an independent, i.e.\@ non-redundant, meaning, the parts of a semantically non-decomposable MWE do not. Differences in the syntactic flexibility of semantically decomposable and semantically non-decomposable MWEs follow exclusively from the interaction between the language-specific constraints on a syntactic operation and the semantics of the MWE's constituents.

\section*{Acknowledgements}

This paper profited from the feedback of numerous esteemed colleagues, many of whom were among the audiences at the PARSEME \emph{1st Training School} in Prague, 
the \emph{Computational Linguistics Research Colloquium} in D\"usseldorf, 
the \emph{16th Szklarska Poreba Workshop}, and the PARSEME \emph{4th General Meeting} in Valletta. Special thanks go to Doug Arnold, Gisbert Fanselow, Christopher G\"otze, Timm Lichte, Stella Markantonatou, and Susanna Salem for their valuable questions, comments, and suggestions. We are also very grateful for the work of two anonymous reviewers and to the proofreaders of LangSci. All remaining errors are ours.

\section*{Abbreviations}
 
\begin{tabularx}{.44\textwidth}{lQ}
\textsc{dat} & dative \\
ICE & idiomatically combining expression \\
\textsc{impers} & impersonal passive \\
HPSG & Head-Driven Phrase Structure Grammar \\
IP & idiomatic phrase \\
%NSW & \citealt{Nunberg1994} \\
\end{tabularx}
%no space or empty line here!
\begin{tabularx}{.55\textwidth}{lQ}
LRS & Lexical Resource Semantics \\
%\textsc{nsw} & \citealt{Nunberg1994} \\
NSW & \citealt{Nunberg1994} \\
\textsc{part} & partitive case\\
SNDI & semantically non-decomposable idiom \\
V2 & verb second \\
\\
\end{tabularx} 

{\sloppy
\printbibliography[heading=subbibliography,notkeyword=this]
}
\end{document}
