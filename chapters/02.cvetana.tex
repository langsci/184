\documentclass[output=paper]{langsci/langscibook}
\author{Svetla Koeva\affiliation{Institute for Bulgarian Language, Bulgarian Academy of Sciences}%
\and Cvetana Krstev\affiliation{Faculty of Philology, University of Belgrade}%
\and Duško Vitas\affiliation{Faculty of Mathematics, University of Belgrade}%
\and Tita Kyriacopoulou\affiliation{Université Paris-Est Marne-la-Vallée}%
\and Claude Martineau\affiliation{LIGM, Universite Paris-Est Marne-la-Vallée}%
\lastand Tsvetana Dimitrova\affiliation{Institute for Bulgarian Language, Bulgarian Academy of Sciences}}
\title{Semantic and syntactic patterns of multiword names: a cross-language study}\shorttitlerunninghead{Semantic and syntactic patterns of multiword names: a cross-language study}

\abstract{ Named entities (NEs) constitute a great challenge for computational linguistics and one of the major research topics during the last decade. They can be divided in categories describing people, location, time, organization and others. In this paper we will restrict our discussion to proper names that belong to three main classes: personal, location and organization names, and that can be either single-word nouns or multiword expressions. First, we are going to define common (language-independent) semantic patterns for proper names and then we will present the corresponding syntactic patterns in English, Bulgarian, French, Greek, and Serbian. We will compare these patterns regarding grammatical categories of dependent constituents, definiteness, distribution of clitics, word order and various alternations. Our ultimate goal is to build a universal framework for Named Entity Recognition (NER).}

\newcommand{\trigger}[1]{\textsc{#1}}

\maketitle

\begin{document}



\section{Introduction}

Proper names are usually defined as belonging to the following main
classes: \isi{personal names}, \isi{location names}, and \isi{organization names}, called
also \isi{named entities} (NEs). They can be single-word nouns or particular
types of multiword expression (MWE).

The aim of this paper is to offer a common template for description and
classification of proper names in different languages. Our objectives
are: i) to formulate semantic patterns for personal, location and
\isi{organization names} that capture the general semantics and should be, to
a great extent, language-neutral; ii) to describe language-specific
\isi{syntactic patterns} corresponding to a common semantic pattern. The
syntactic patterns provide information about the grammatical class of
the head and constituents; dependencies among the constituents; word
order and contiguity; cliticisation (if applicable – for possessive
pronoun and interrogative clitics).

This study is based on evidence gathered from five languages – \ili{English}, \ili{Bulgarian}, \ili{French}, \ili{Greek}, and \ili{Serbian} –
belonging to four different language groups (Germanic, Hellenic,
Romance, and Slavic). The utility of language-neutral \isi{semantic patterns}
lies in the fact that they can be applied to new languages, thus paving
the way towards more universal solutions for (rule-based) named entity
recognition (NER). The set of language-specific syntactic patterns
displays correspondences between morphological and syntactic
language-specific characteristics, and they may serve as transformation
rules in rule-based machine translation, cross-lingual information
extraction and summarization.


\section{Names: a general overview}

In \ili{English} \citep[96]{huddleston1988}, two different terms are often used:
\textit{proper noun} – referring to the part-of-speech of the word and
comprising only single-word proper names, e.g., \textit{John, London,
Adidas}, and \textit{proper name} – referring to the function of these words
as referential elements and comprising single- and multiword proper
names, as in \textit{John}, \textit{John Smith Junior},
\textit{London}, \textit{the United States of America, Nike, Microsoft
Corporation}. Following this distinction, proper names can be further
specified as: proper nouns (\textit{Anna, Asia, Google}), multiword
expressions (\textit{Jean-Pierre Deckles, New York, the United
Nations}), and noun phrases (\textit{Professor Deckles, New York City,
the United Nations Organization)}. Proper names – expressed either by
proper nouns or by MWEs – show common semantic and syntactic behavior
and we describe them in a uniform way.



Proper names do not “[d]escribe or specify characteristics of objects” but
are “logically connected with characteristics of the object to which
they refer” \citep[173]{searle1958}. For example, \textit{Saint Petersburg}
may refer to the second largest city in Russia (\textit{They convened
in Saint Petersburg}), a city in Florida, a city in Pennsylvania, the
fictional hometown of Tom Sawyer and Huckleberry Finn, \textit{St.
Petersburg, Missouri}, but also to a college in Florida – \textit{Saint
Petersburg College}. A particular common noun (i.e., \textit{city,
street, president, actor}) specifies the object whose instances may be
represented by a set of proper names; and such an object is always
presupposed for a given proper name even if the common noun is not
mentioned explicitly in the text. Furthermore, \isi{multiword names} may
comprise a category word (\textit{square} in \textit{Trafalgar Square};
\textit{ocean} in \textit{the Indian Ocean}), and in these cases, the
category of the particular name is always explicitly shown \citep[144]{carroll1985}.




The relation between a proper name and its category object is reflected
in WordNet where the relation between a concept and its instances is
defined as an instant hypernym (instant hyponym) relation
\citep{rodriguez1998}. The
instances (proper nouns) inherit characteristics from the concepts of
the hierarchy to which they belong. For example, the name \textit{Saint
Petersburg} is an instance of a city, and the concept \{city,
metropolis, urban centre\} links to the more general concept
\{region\}, which, in its turn, links to an even more general concept
\{location\}.



Different terms are used for common nouns that categorise the referents
of proper names as members of different classes: descriptors,
designators, category words \citep{carroll1985}, external evidence \citep{mcdonald1996}, triggers \citep{magnini2002}, trigger words. To avoid
confusion with the theory of reference, we will use the term \textit{trigger}.



Triggers depend semantically on the referent of the personal name, and
different names select different classes of triggers. In turn, triggers
determine the characteristics of the object to which the name refers.
For example, if we know that the word \textit{Washington} is a family
name, it can select the word \textit{president} or the word
\textit{actor}. Further, the word \textit{president} and the word
\textit{actor} are similar in the way they designate the concept for a
person, and this determines the fact that both nouns can co-occur with
adjectives denoting height, age, etc. The meaning of both words also
implies that they may be specified by employing expressions for
affiliation as complements (\textit{the President of the USA, the actor
at the Muppet Theatre}). However, not all words that are compatible
with the first noun are compatible with the second (\textit{stage
actor} vs. *\textit{stage president}). Therefore, the notion of
triggers is central for the classification of the \isi{semantic patterns} of
proper names and accordingly – for the description of the respective
\isi{syntactic patterns}.


\section{Grammatical features of names in \ili{Bulgarian}, \ili{English}, \ili{French}, \ili{Greek} and \ili{Serbian}: a brief overview}

\isi{personal names} are singular and inherently definite (the same applies to
location and \isi{organization names} that cannot express definiteness). Some
\ili{Bulgarian}, \ili{English}, \ili{French}, \ili{Greek}, and \ili{Serbian} location and
organization names are in singularia tantum or pluralia tantum, or
marked for definiteness: with a definite article (\ili{English}, \ili{French},
\ili{Greek}), with a definite article attached to the noun trigger with no
pre-nominal modifiers or to the leftmost modifier in \ili{Bulgarian}, and
only with the definite form of adjectives in \ili{Serbian}.



\ili{Bulgarian}, \ili{French}, \ili{Greek}, and \ili{Serbian} personal, location (apart from
cities in \ili{French} which usually do not express gender) and organization
names are marked for grammatical noun gender – masculine or feminine,
in contrast to the \ili{English} ones. Location and organization names in
\ili{Bulgarian}, \ili{Greek} and \ili{Serbian} can be marked for neuter, as well. In
\ili{Greek} and \ili{Serbian}, proper names have the nominative, accusative,
genitive and vocative case. In \ili{Serbian}, names can also be declined in
the dative, locative and instrumental, while in \ili{Bulgarian} vocative is
observed only with some forenames.



Syntactically, proper nouns are heads of noun phrases but show
restricted combinatorial properties compared to common nouns. For the
five languages discussed in this paper, the forenames can be extended
with one or more (rarely more than two) proper nouns: a nickname, a
patronym and/or a family name. Agreement in gender and number is
observed if they are of Slavic and \ili{Greek} origin. For feminine surnames
of Slavic origin in \ili{Serbian}, the agreement in gender is allowed but not
obligatory. \ili{Bulgarian}, \ili{French}, \ili{Greek} and \ili{Serbian} adjectives and
\ili{Bulgarian}, \ili{French} and \ili{Serbian} possessive pronouns change to agree in
gender and number with the nouns they modify. \ili{Greek} and \ili{Serbian}
adjectives and possessive pronouns agree in case with the head noun.



Compared to \isi{personal names}, \isi{location names} have a more diverse
structure, while \isi{organization names} show the highest complexity. Both
location and \isi{organization names} can be proper nouns or proper names,
comprising proper and common nouns or noun phrases, which begin to
function as names of geographical locations and organizations,
respectively.

\section{Names and multiword expressions}

Many names are composed of more than one word and are classified as
\isi{multiword names}. They can comprise two or more proper nouns
(\textit{Ray Jackendoff, Merill Lynch}); common and proper nouns (\ili{Bulgarian}:
\textit{Republika Bălgariya} ‘Republic Bulgaria’); adjectives and a
proper or a common noun (\textit{International Monetary Fund},
\textit{Upper Manhattan}); abbreviations (\textit{Financial Advisors
Ltd.}, \textit{John Smith Jr.}, \textit{Miami, FL}); numerals or
numbers (\textit{the Second Generative Grammar Conference; XX
Generative Linguistics Conference}); verbs and adverbs with names of
products such as books, movies, songs (\textit{Someone to Watch over
Me; Killing Me Softly}), etc.

\cite{anderson2007} provides a detailed classification of proper names, a
subset of which is relevant for our study, as follows: simple opaque
names (\textit{John}); simple names that have a resemblance to a common
word (\textit{Prudence}); names based on other names (\textit{Lincoln}
– for a boulevard); names overtly derived from other names (Slavic
family names); names based on compounds, some of them containing a name
(\textit{Queensland, Newtown}); names based on longer phrases – they
may include another name (\textit{the University of Queensland}) or not
(\textit{Long Island, Hen and Chicken Island}); and names based on
sentences (as with titles of movies).

An important feature that systematically distinguishes location and
\isi{organization names} from \isi{personal names} is that the \trigger{triggers} may
be their integral part, constituting a MWE (\ili{Bulgarian}:
\textit{Černo \trigger{more}} `Black \trigger{Sea}’,\footnote{The translations in the paper are closer to literal translations than to proper ones, e.g., \ili{Bulgarian}: \textit{Republika Bălgariya} ‘Republic Bulgaria’ instead of `\ili{Bulgarian} Republic' or `The Republic of Bulgaria', but \ili{Greek}: ο Έλληνας \trigger{Πρωθυπουργός}  `the \ili{Greek} \trigger{Prime Minister'} (instead of `premier',
for example).}
\ili{English}: \textit{\trigger{President} Roosevelt Boulevard}, \textit{First Investment \trigger{Bank}},
\ili{French}: \textit{\trigger{Banque} de France} ‘\trigger{Bank} of France’, 
\ili{Greek}: \textit{\trigger{Τράπεζα} της Ελλάδος} ‘\trigger{Bank} of Greece’, 
\ili{Serbian}: \textit{Jadransko \trigger{more}} ‘Adriatic \trigger{sea}’, \textit{Međunarodni} \textit{\trigger{sud}}
\textit{pravde} ‘International \trigger{Court} of Justice’).



\cite{carroll1985} describes the non-classifying part of the location name
as a name-stem (e. g., \textit{Trafalgar} in \textit{Trafalgar Square})
and explores rules according to which the name-stem can be used to
stand for the whole name. Not only for some \isi{location names}, but also
for some \isi{organization names} with internal triggers, the name-stem can
replace the whole name, e.g., \ili{French}: \textit{\trigger{la maison d’édition}}
\textit{Hachette} ‘the Hachette \trigger{publishing company}’ or just
\textit{Hachette, }\ili{Greek}: η Ολυμπιακή ‘the Olympic’ for Olympic
\trigger{Airways}.

Further, a \is{location names} location name may feature a \textbf{personal name} specified
by a \trigger{personal} \trigger{trigger} (\textit{\trigger{San}}
\textbf{\textit{Jorge}} \textit{River}) that cannot be omitted without
loss of the name function; similarly, an organization name may feature
a \textbf{personal name} specified by a \trigger{personal trigger}
(\textit{\trigger{San}} \textbf{\textit{Jose}} \textit{State University})
or a \textbf{location name} specified by a \trigger{location trigger}
(\textbf{\textit{Los Angeles}} \textit{\trigger{City}} \textit{College}).

For the purposes of our study, we differentiate the names on the basis
of their structure: i) whether the name is a MWE or a noun; ii) whether
the multiword name obligatorily incorporates a trigger (an internal
trigger); and iii) whether the name (either single or a MWE) is
optionally specified by a trigger (an external trigger). The external
triggers may be explicit or implicit, depending on the context (the
City of New York, New York City vs. New York):

\begin{itemize}
\item Single-word personal name (\textit{Arthur}).
\item Multiword personal name (\textit{Arthur Conan Doyle}).
\item Multiword personal name, which incorporates an internal personal
trigger. When people are famous, combinations with triggers such as
holy, aristocratic and religious titles can be widely used and are stable
(\textit{\trigger{Pope}} \textit{John Paul II}).
\item Single-word personal name; it is specified by an external personal trigger
(\textit{\trigger{uncle}} \textit{John}). Kinship terms are usually
combined with a single-word personal name.
\item Multiword personal name; it is specified by an external personal trigger
(\textit{\trigger{Professor}} \textit{Steven Pinker}).
\item Single-word location name (it can coincide or not with a personal name)
(\textit{Danube, Washington}).
\item Multiword location name (it may - partially -  coincide or not with a
personal name) (\textit{Little Rock, }\textit{\trigger{San}}\textit{
Antonio}).
\item Multiword location name, comprising an internal location trigger
(\textit{Rocky }\textit{\trigger{Mountains}}). No additional location
trigger of the same type can be added; being part of the name, the
trigger cannot be omitted either. A multiword location name may include
a personal name (and, rarely, an organization name) \textit{(Cristina
}\textit{\trigger{Fort}}).
\item Single-word location name; it is specified by an external location trigger
(\textit{\trigger{River}} \textit{Nile}).
\item Multiword location name; it is specified by an external location trigger
(\textit{\trigger{volcano}} \textit{Klyučevskaya Sopka}).
\item Single-word organization name (it may coincide or not with a personal
name or a location name) (\textit{Matalan, Poundland}).
\item Multiword organization name (it may (partially) coincide or not with a
personal or a location name) (\textit{Mercedes Benz}).
\item Multiword organization name, comprising an internal organization trigger
as an integral part of the proper name. Another organization trigger of
the same type cannot be added. The trigger, which is part of the name,
cannot be omitted either. The multiword organization name may include a
personal or a location name (\textit{\trigger{Princess}}\textit{ Basma
Youth Resource }\textit{\trigger{Center}}, \textit{Melbourne Grammar
}\textit{\trigger{School}}).
\item Single-word organization name; it is specified by an external organization
trigger (\textit{\trigger{Supermarket}}\textit{ Galaxy}).
\item Multiword organization name; it is specified by an external organization
trigger (\textit{the }\textit{\trigger{company}} \textit{Business Models
}\textit{\trigger{Inc.}})
\end{itemize}


\section{Semantic patterns for persons, locations and organizations}


Names can be grouped into different semantic classes and subclasses with
respect to the properties of their referents (explicated by triggers).
A name from a given class (personal, location or organization) selects
triggers from a particular set of semantic subclasses. For example,
complex \isi{personal names} are combined with triggers that define a
legislative job title, executive job title, judicial position, academic
position, academic title, military rank, and profession. The
permissible combinations between types of names (proper nouns, MWEs),
and semantic subclasses of triggers determine the \isi{semantic patterns}
applicable to the personal, location and \isi{organization names}. The
semantic patterns we propose show semantic compatibility valid for a
particular semantic class and describe the permissible combinatory
options. For example, a personal name can be extended with a kinship
term (i.e., \textit{the beautiful
}\textit{\trigger{step-daughter}}\textit{ of John from Paris,
}\textbf{\textit{Anne Nicole}}) and the kinship term can be specified
in various ways and restricted for possessor and location, thus the
respective semantic pattern is: (modifier: referent specification
phrase) – trigger: kinship term – (complement: possessor phrase) –
(complement: location phrase) – personal name.

As triggers refer to concepts, the semantic relations in which they are
involved should be universal and must hold among the relevant concepts
in any language. Thus, the \isi{semantic patterns} describe language-neutral
relations and can be regarded as universal structures with correlating
language-specific \isi{syntactic patterns}.



Following the detailed hierarchy employed by \cite{giuliano2009}  for
automatic classification of personal NEs, we can conclude that every
common noun that determines the referent of a personal name can be a
trigger, i.e., words such as \textit{chess-player}, \textit{singer},
\textit{footballer}, etc. \cite{magnini2002} use WordNet hierarchy
for identification of large sets of triggers – hyponyms of high-level
synsets such as \{person\}, \{location\}, \{organization\}. Some
authors suggest verb triggers appearing in the local context of NEs
\citep{zhang2004}, e.g., for water bodies (like rivers) the verb may
indicate that \textit{Sava} is a river and not a person (\textit{The
Sava flooded the village}). \ There are detailed classifications of NEs
(of more than 200 categories; cf. \citealt{sekine2004}), while other
classifications build shallow hierarchies with the major classes on the
top and sets of subtypes with different granularity at the low levels
(\citealt{ace2008}; \citealt{fleischman2002}).

In our study, we distinguish the following semantic subclasses for
person, location and \isi{organization names} and their triggers:

\begin{itemize}
\item Persons and personal triggers: legislative job title: \textit{prime
minister}; executive job title: \textit{executive officer}; judicial
position: \textit{judge}; academic position: \textit{associate
professor}; military rank: \textit{major general}; profession:
\textit{engineer}; academic title: \textit{Ph.D.}; true honorific:
\textit{Mister / Mr.}; aristocratic title: \textit{Prince}; religious
title: \textit{Bishop}; kinship term: \textit{sister}; holy title:
\textit{Saint}.
\item Locations and location triggers: natural: \textit{river}; public:
\textit{monument}; commercial: \textit{restaurant}; infrastructure:
\textit{boulevard}.
\item Organizations and organization triggers: business: \textit{company};
political: \textit{political party}; government: \textit{ministry};
media: \textit{publishing house}; human / non-government:
\textit{association}.
\end{itemize}

We classify proper names (persons, locations and organizations) in the
patterns (A) to (I) below according to their shared features.
Patterns are described in terms of the categories (a)-(d): (a) the
semantic subclass of the trigger; (b) the type of the proper name that
selects triggers; (c) obligatoriness / optionality of the trigger
manifested by an internal or external trigger with respect to the name;
(d) the semantic pattern that the proper name evokes.





\subsection{Pattern A}

(a) Semantic class of the trigger: legislative job title, executive job
title, judicial position, academic position, academic title, military
rank, profession. Specification of military ranks and top-level
legislative, executive, and judicial triggers is not allowed: *prime
minister of finance; lower level legislative triggers can be specified:
engineer in automatics. (b) Type of the proper name: personal name
extended or substituted by a family name. (c) External trigger. (d)
Semantic pattern: (referent specification phrase) – trigger – (domain
specification phrase) – (possessor phrase) – (affiliation phrase) –
(location phrase).


Example: \ili{English}: (\textit{his} | \textit{Stefan's}) (\textit{new})
\textit{\trigger{professor}} (\textit{of law}) (\textit{at the
University}) (\textit{in Plovdiv}) \textbf{\textit{Ivan
Ivanov}}\textit{.}



\subsection{Pattern B}

(a) Semantic class of the trigger: aristocratic title, religious title.
(b) Type of the proper name: personal name or family name. Some
aristocratic and religious titles are selected only by a personal name
(\textit{\trigger{Pope}}\textit{ }\textbf{\textit{Francis}}), while
others are selected by a family name (\textit{\trigger{Lord}}\textit{
}\textbf{\textit{Orsini}}). A trigger can also be part of a personal
name (for distinguished persons) but no separate pattern is defined for this type of name\textit{.} (c) External trigger. (d) Semantic pattern:
(referent specification phrase) – \trigger{trigger} – (affiliation
phrase) – (location phrase).



Example: \ili{English}: \textit{the }(\textit{new})\textit{
}\textit{\trigger{Metropolitan}} (\textit{of the Church}) (\textit{in San
Francisco})\textit{ }\textbf{\textit{Iona}}\textit{.}

\subsection{Pattern C}

(a) Semantic class of the trigger: kinship term. (b) Type of the proper
name: personal name (rarely modified by family name(s)). (c) External
trigger. (d) Semantic pattern: (referent specification phrase) –
\trigger{trigger} – (possessor phrase) – (location phrase).



Example: \ili{English}: (\textit{his} | \textit{Ivan's}) (\textit{blond})\textit{
}\textit{\trigger{step-brother}} (\textit{from Sofia})\textit{
}\textbf{\textit{Stefan}}.

\subsection{Pattern D}

(a) Semantic class of the trigger: holy title (a limited set of words).
(b) Type of the proper name: personal name (rarely modified or
substituted by a nickname). (c) External trigger. (d) Semantic pattern:
(referent specification phrase) – \trigger{trigger} – (location phrase).



Example: \ili{English}: (\textit{miraculous}) \textit{\trigger{saint}}\textit{
}(\textit{from Patara})\textit{ }\textbf{\textit{Nicholas}}.


\subsection{Pattern E}

(a) Semantic class of the trigger: true honorific. (b) Type of the
proper name: personal name extended or substituted by a family name.
(c) External trigger. (d) Semantic pattern: – trigger 


Example: \ili{English}:
\textit{\trigger{Monsieur}}\textit{ }\textbf{\textit{Ivan Ivanov}}.

\subsection{Pattern F}

(a) Semantic class of the trigger: location. (b) Type of the proper
name: location name. (c) External trigger. (d) Semantic pattern:
(referent specification phrase) – \trigger{trigger} – (specification
phrase) – (possessor phrase) – (location phrase)



Example: \ili{English}: \textit{the }(\textit{beautiful})\textit{
}\textit{\trigger{city}} (\textit{near the big river})\textit{,}
\textbf{\textit{Plovdiv}}.

\subsection{Pattern G}


(a) Semantic class of the trigger: location. (b) Type of the proper
name: location name. (c) Internal trigger. (d) Semantic pattern:
(referent specification phrase) – internal \trigger{trigger} – (location
phrase)



Example: \ili{English}: \textit{the }(\textit{beautiful})
\textbf{\textit{\trigger{Mount}}}\textbf{\textit{ Fuji}}
(\textit{in}\textit{ Japan}).

\subsection{Pattern H}


(a) Semantic class of the trigger: organization. (b) Type of the proper
name: organization name. (c) External trigger. (d) Semantic pattern:
(referent specification phrase) – \trigger{trigger} – (domain
specification phrase) – (possessor phrase) – (affiliation phrase) –
(location phrase).



Example: \ili{English}:\textit{ the }(\textit{new})\textit{
}\textit{\trigger{company}}\textit{ }(\textit{of his friends})
(\textit{in Athens})\textit{,} \textbf{\textit{Tetracom}}.

\subsection{Pattern I}

(a) Semantic class of the trigger: organization. (b) Type of the proper
name: organization name. (c) Internal trigger. (d) Semantic pattern:
(referent specification phrase) – internal \trigger{trigger} – (location
phrase).



Example: \ili{English}: \textit{the }(\textit{new}) \textbf{\textit{Hebros}}
\textbf{\textit{\trigger{Bank}}} (\textit{in Athens}).




\section{Language-specific syntactic patterns for persons, locations and organizations}



We define the \isi{semantic patterns} evoked by different types of proper
names when combined with triggers, and the \isi{syntactic patterns} that
involve combinations of: modifiers, one or several, semantically
restricted by the head proper noun and the trigger, and complements,
semantically restricted by the trigger.

The syntactic patterns are language-specific and differ for personal,
location and \isi{organization names}. The syntactic patterns may involve
combinations of adjectival modifiers in pre- or post-nominal position,
one or several; pronoun modifiers in pre-nominal position (possessive
and demonstrative); complements in post-nominal position, one or
several; and a noun modifier in pre- or post-nominal position,
alternating with a prepositional
phrase.\footnote{The \textit{noun modifier – prepositional phrase} alternation
is not described in the syntactic patterns.} Adjectival modifiers (that in \ili{Bulgarian},
\ili{French}, \ili{Greek} and \ili{Serbian} agree with the head noun in gender and
number) may indicate physical shape, status, etc. Complements may
indicate (domain) specification, affiliation, location, possessor, and
may be prepositional or case complements, depending on the language
structure.


\isi{multiword names} have the structure of a noun phrase and exhibit specific
properties with respect to constituency of the head noun and the
components, including various constraints on modifiers, complements,
clitics (in \ili{Bulgarian} and \ili{Greek}), etc.



The \isi{syntactic patterns} represent language-specific grammatical features
and dependencies and how these features and dependencies are manifested
in a particular language. One or more syntactic patterns from one or
different languages may correspond to the same semantic pattern. The
syntactic patterns, as they are presented in this paper, define
constituency and reflect the morphological and syntactic structure of a
particular language, although they do not strictly describe phrase
structure and grammatical dependencies. However, the syntactic patterns
are formal enough to code the linguistic information correctly and to
allow for the conversion to some formalism.



Syntactic patterns corresponding to the largely universal semantic
patterns, described in §5, are formulated for \ili{English},
\ili{Bulgarian}, \ili{French}, \ili{Greek}, and \ili{Serbian}.
The generalizations for semantic
patterns and respective \isi{syntactic patterns} were constructed on the
basis of observations and classifications made on dictionaries of NEs,
annotated corpora of NEs and grammars for NE recognition developed so
far (\citealt{krstev2013}; \citealt{koeva2015}).



\subsection{Syntactic pattern A (single family name or multiword personal name)}
Characteristics shared by the five
languages\footnote{
Language  specific characteristics
were also formulated but due to limitation of space they are
represented only in the syntactic patterns.}: i) Triggers are placed
to the left of the personal name; a complex trigger phrase is likely to
be an
apposition\footnote{The restrictive
apposition of a proper noun (whose omission changes the meaning of the
sentence) is covered by the syntactic patterns
(\textit{the new} \textit{\trigger{Professor}} 
\textit{of Law,} \textbf{\textit{Chris Smith}}). The non-restrictive
apposition (\textbf{\textit{Chris Smith}} \textit{, the new}
\textit{\trigger{Professor}} 
 \textit{of Law}) is not included due to
limitation of space. In this paper, the term apposition is used for the
``non-restrictive apposition’' only.}. Examples: \ili{English}: \textit{the new
}\textit{\trigger{Professor}} \textit{of Sustainable Agriculture and
Climate Change,} \textbf{\textit{Chris Smith}} or \textbf{\textit{Chris
Smith}}\textit{, the new }\textit{\trigger{Professor}} \textit{of Law at
the University's Humanities Institute;} \ili{French}: \textit{Le nouveau}\trigger{
}\textit{\trigger{professeur}} \textit{de morale de l’Université de
Fribourg,} \textbf{\textit{Thierry Collaud}} ‘the new \trigger{professor}
in ethics at the University of Fribourg, \textbf{Thierry Collaud}’or
\textbf{\textit{Thierry Collaud}}, \textit{le nouveau}\trigger{
}\textit{\trigger{professeur}} \textit{de morale de l’Université de
Fribourg }‘\textbf{Thierry Collaud,} the new \trigger{professor} in
ethics at the University of Fribourg’; ii) If no modifiers or
complements exist, the trigger is indefinite (except for \ili{Greek} where
the article is obligatory); otherwise, it is definite. Examples: \ili{Bulgarian}:
\textit{\trigger{ministăr}} \textbf{\textit{Kuneva }}‘\trigger{minister}
\textbf{Kuneva}’\textit{ }versus\textit{ noviyat}
\textit{\trigger{ministăr}} \textit{v pravitelstvoto}
\textbf{\textit{Ivan Dimitrov}} ‘new-the \trigger{minister} in
government-the \textbf{Ivan Dimitrov}’; \ili{English}: \textit{\trigger{professor}}
\textbf{\textit{Chomsky}} versus \textit{the
}\textit{\trigger{professor}} \textit{of linguistics
}\textbf{\textit{Noam Chomsky}}; iii) \isi{personal names} can be extended
with one or more (rarely more than two) proper names: a nickname, a
patronym and/or a family name, constituting a MWE. Example: \ili{Serbian}:
\textit{\trigger{Dr}} \textbf{\textit{Slavica Đukić-Dejanović}}\textit{,
}\textit{\trigger{predsednik}} \textit{Narodne skupštine} ‘\trigger{Dr.}
\textbf{Slavica Đukić-Dejanović}, \trigger{president} of National
Assembly’.



\ili{English}:\footnote{The vertical
bar – |, separates alternatives. The question mark indicates zero or
one occurrences of the preceding element. The asterisk  ``*'', indicates
zero or more occurrences of the preceding element. Parentheses ``()'',
are used to define the scope and precedence of the operators. The
equality sign \ =, indicates the semantics of the prepositions and case
phrases.}
(((DefArt|GenDet|GPossPron) Adj*
\trigger{Trigger} (PP \textit{in}|\textit{of}|\textit{for = DomSpec})? (PP
\textit{at = Aff})* (PP \textit{in = Loc})?) | (DefArt \trigger{Trigger} (PP \textit{in}|\textit{of}|\textit{for = DomSpec})?
(PP \textit{at}|\textit{of = Aff})* (PP \textit{in = Loc})?) | \trigger{IndefTrigger)}  \textbf{PerN}



\ili{Bulgarian}: (((DefAdj (PossCL)?) | \textbf{DefPossPron}) Adj* \trigger{Trigger}
(PP \textit{po}|\textit{na}|\textit{za = DomSpec})? (PP \textit{na}|\textit{pri}|\textit{v}|\textit{kăm}|\textit{ot = Aff})*)
(PP \textit{ot = Loc})?) | (\trigger{DefTrigger} (PossCL)? (PP
\textit{po}|\textit{na}|\textit{za = DomSpec})? (PP \textit{na}|\textit{pri}|\textit{v}|\textit{kăm}|\textit{ot = Aff})* (PP
\textit{ot = Loc})?) | \trigger{IndefTrigger}) \textbf{PerN}




\ili{French}: (((DefArt|PossPron) Adj* \trigger{Trigger} (PP
\textit{en}|\textit{de = DomSpec})? (PP
\textit{à =Aff})* (PP à
\textit{= Loc})?) | (DefArt
\trigger{Trigger} Adj* (PP \textit{en}|\textit{de = DomSpec})? (PP
\textit{à|de = Aff})* (PP à \textit{= Loc})?)   |
\trigger{IndefTrigger})  \textbf{PerN}



\ili{Greek}: ((DefArt Adj* (PossPron)? \trigger{Trigger} (GenP \textit{= DomSpec})?
(PP σε \textit{= Aff})* (PP σε \textit{= Loc})?) | (DefArt
\trigger{Trigger} (GenP \textit{= DomSpec})? (GenP \textit{= Aff})* (PP
σε \textit{= Loc})?) | \trigger{DefTrigger)}  \textbf{PerN}



\ili{Serbian}: ((Adv | PossPron)? DefAdj* \trigger{Trigger} (PP \textit{za}| GenP =
\textit{DomSpec})? (PP \textit{u|pri} | GenP = \textit{Aff})* (PP
\textit{u}|\textit{za}|\textit{iz} GenP = \textit{Loc})?)  \textbf{PerN}



\begin{table}[H]
\begin{tabularx}{\textwidth}{lX}
\lsptoprule
\mdseries\itshape \ili{English} & \mdseries\itshape Spain’s \trigger{Secretary} of State for Foreign
Affairs, \textbf{Gonzalo de Benito Secades}\\
\mdseries\itshape \ili{Bulgarian} & \mdseries\itshape Dăržavniyat \trigger{sekretar} na vănšnite
raboti na Ispaniya, \textbf{Gonzalo de Benito Sekades}\\
\mdseries\itshape \ili{French} & \mdseries\itshape le \trigger{ministre} espagnol des Affaires Etrangères,
\textbf{Gonzalo} \textbf{de Benito Secades}\\
\mdseries\itshape \ili{Greek} & \mdseries ο Ισπανός \trigger{Υπουργός} Εξωτερικών,
\itshape \textbf{Gonzalo de Benito Secades}\\
\mdseries\itshape \ili{Serbian} & \mdseries\itshape Državni \trigger{sekretar} za spoljne poslove Španije, \itshape \textbf{Gonzalo de Benito
Sekades}\\
\lspbottomrule
\end{tabularx}
\caption[Syntactic pattern A – an example translated in the five languages.]{Syntactic pattern A – an example translated in the five languages.\footnote{The examples do not illustrate all variants.}}
\end{table}



\subsection{Syntactic pattern B (single- or multiword personal name)}

Characteristics shared by the five languages: i) The trigger phrase is
placed in front of the personal name but the appositive order can also
be found, especially if the trigger phrase is complex. Examples: \ili{English}:
\textbf{\textit{Venerable Dionysius,}} \textit{the \trigger{Archimandrite} of St Sergius' Monastery}; \ili{Serbian}:
\textit{njegovo preosveštenstvo \trigger{episkop}}
\textit{niški} \textbf{\textit{Irinej}} ‘His Grace \trigger{Bishop}
of-Niš \textbf{Irinej}’; ii) If no trigger modifiers or complements
exist, the trigger is indefinite (except for \ili{Greek} where the article is
obligatory); otherwise, it is definite. Examples: \ili{Bulgarian}:
\textit{\trigger{Patriarh}} \textbf{\textit{Maksim}} ‘\trigger{Patriarch}
\textbf{Maxim}’; \ili{French}: \textit{\trigger{L’archevêque}} \textit{de Paris,
Monseigneur }\textbf{\textit{André Vingt-Trois}} ‘the
\trigger{Archbishop} of Paris, Monsignor \textbf{André Vingt-Trois}’;
\textit{Le bienheureux }\textit{\trigger{père}}
\textbf{\textit{Brottier}} ‘the blessed \trigger{father}
\textbf{Brottier}’; iii) \isi{personal names} can be extended with one or
more (rarely more than two) proper names: a nickname, a patronym and/or
a family name. These constituents form a complex name (MWE). Example: \ili{Greek}: ο
πρόσφατα χειροτονηθείς \trigger{Σεβασμιότατος Μητροπολίτης}
Κεφαλληνίας, \trigger{πατέρας} \textbf{Γεώργιος
Σαπουνάς} ‘the newly appointed Most Reverend \trigger{Bishop} \linebreak \trigger{ Metropolitan} of Kefalonia, \trigger{Father} \textbf{Georgios Sapounas}’.




\ili{English}: ((DefArt Adj* \trigger{Trigger} (PP \textit{of = Aff})* (PP
\textit{in = Loc})?) | (DefArt \trigger{Trigger} (PP
\textit{at}/\textit{of = Aff})* (PP \textit{in = Loc})?) | \trigger{IndefTrigger})
\textbf{PerN}



\ili{Bulgarian}: (((DefAdj (PossCL)? | DefPossPron) Adj* \trigger{Trigger} (PP
\textit{na}|\textit{pri}|\textit{v}|\textit{kăm}|\textit{ot} \textit{= Aff})? (PP \textit{ot = Loc})?) |
(\trigger{DefTrigger} (PP \textit{na}|\textit{pri}|\textit{v}|\textit{kăm}|\textit{ot} \textit{= Aff})? (PP \textit{ot
= Loc})?) | \trigger{IndefTrigger)} \textbf{PerN}



\ili{French}: ((DefArt Adj* \trigger{Trigger} (PP \textit{de = Aff})* (PP
\textit{à}|\textit{en = Loc})?) | (DefArt \trigger{Trigger}* Adj* (PP \textit{de =
Aff})* (PP \textit{à}|\textit{en = Loc})?) | \trigger{IndefTrigger}) \textbf{PerN}



\ili{Greek}: ((DefArt Adj* \trigger{Trigger} (GenP \textit{= Aff})* (GenP\textit{ =
Loc})?) | (DefArt \trigger{Trigger} (GenP = \textit{Aff} | \textit{Loc})* (PP
σε \textit{= Loc})?) | \trigger{DefTrigger}) \textbf{PerN}



\ili{Serbian}: (((PossPron | PossAdj) \trigger{Trigger}) | (DefAdj* \trigger{Trigger})
| (\trigger{Trigger} DefAdj?)) (PP \textit{u}|\textit{za}| \textit{GenP = Aff})? (PP
\textit{u}|\textit{za}|\textit{iz GenP = Loc})?) \textbf{PerN} DefAdj*

\begin{table}
\begin{tabularx}{\textwidth}{lX}
\lsptoprule

\itshape \ili{English} & \itshape the new \trigger{bishop} of the Christian Catholic Church of Switzerland,
\trigger{Dr} \textbf{Harald Rein}\\
\itshape \ili{Bulgarian} & \itshape noviyat \trigger{episkop} na Hristiyanskata katoličeska cărkva v Šveycariya,
\trigger{d-r} \textbf{Harald Rein}\\
\itshape \ili{French} & \itshape le nouvel \trigger{Evêque} de l'Eglise catholique-chrétienne de la Suisse,
\trigger{Dr} \textbf{Harald Rein}\\
\itshape \ili{Greek} & ο νέος \trigger{Αρχιεπίσκοπος} της Χριστιανικής Καθολικής Εκκλησίας της Ελβετίας,
\trigger{Δρ.} \itshape \textbf{Harald Rein}\\
\itshape \ili{Serbian} & \itshape novoimenovani \trigger{biskup} Starokatoličke crkve Švajcarske,
\trigger{dr} \textbf{Harald Rajn}\\
\lspbottomrule
\end{tabularx}
\caption{Syntactic pattern B – an example translated in the five languages.}
\end{table}




\subsection{Syntactic pattern C (single- or, rarely, multiword personal name)}

Characteristics shared by the five languages: i) The usual position of
the trigger is in front of the personal name. The reverse order
indicates apposition. Examples: \ili{Bulgarian}: \textit{moyata hubava
}\textit{\trigger{sestra}} \textbf{\textit{Ana}} ‘my-the beautiful
\trigger{sister} \textbf{Anna}’; \ili{Greek}: η \trigger{αδερφή}
του Πέτρου, η \textbf{Μαρία} ‘the \trigger{sister} of
Peter, the \textbf{Maria}’; η \textbf{Μαρία,}
η \trigger{αδερφή} του Πέτρου ‘the
\textbf{Maria}, the \trigger{sister} of Peter’. ii) The trigger is
accompanied by modifiers or complements. Examples: \ili{English}: \textit{his
older }\textit{\trigger{step-brother}} \textit{from Paris,}
\textbf{\textit{Stefan; }}\ili{Serbian}: \textit{njegov rođeni
}\textit{\trigger{brat}} \textit{četvorogodišnji} \textbf{\textit{Zoran
}}‘his birth \trigger{brother} 4-year-old \textbf{Zoran}’; iii) The
phrase headed by the trigger is definite. Examples: \ili{French}: \textit{la
}\textit{\trigger{sœur}} \textit{de Marc,} \textbf{\textit{Marie}} ‘the
\trigger{sister} of Marc, \textbf{Maria}’ vs. \textit{sa
}\textit{\trigger{sœur}}\textit{,} \textbf{\textit{Marie}} ‘his
\trigger{sister}, \textbf{Maria}’; \textit{son}\trigger{
}\textit{\trigger{beau-frère}} \textit{de Paris,} \textbf{\textit{Jean}}
‘his \trigger{brother-in-law} from Paris, \textbf{John}’.

\ili{English}: (((DefArt | GenDet | PossPron) Adj* \trigger{Trigger} (PP \textit{from
= Loc})?) | (DefArt Adj* \trigger{Trigger} (PP \textit{of =
Poss}\footnote{The possessor phrase is shown only for the kinship terms.})? (PP
\textit{from = Loc})?)) \textbf{PerN}



\ili{Bulgarian}: (((DefAdj (PossCL)?) | DefPossPron) Adj* \trigger{Trigger} (PP
\textit{ot = Loc})?) | ((\trigger{DefTigger} (PossCL)?) (PP \textit{ot =
Loc})?) | (DefAdj Adj* \trigger{Trigger} (PP \textit{na = Poss})?) |
(\trigger{DefTrigger} (PP \textit{na = Poss})?))  \textbf{PerN}



\ili{French}: (DefArt | PossPron) Adj* \trigger{Trigger} ((PP \textit{de = } \textbf{PerN}) |
(PP \textit{de = Loc})?)  \textbf{PerN}



\ili{Greek}: (DefArt Adj* \trigger{Trigger} ((GenP =  \textbf{PerN}) | (GenP =
\textit{Poss}))? (PP από DefArt \textit{= Loc})?) DefArt  \textbf{PerN}



\ili{Serbian}: ( \textbf{PerN}* \trigger{Trigger}* (DefAdj* | (GenP =  \textbf{PerN}))? (PP \textit{iz
= Loc})?)  \textbf{PerN} | (PossPron? DefAdj* \trigger{Trigger}* (PP \textit{iz =
Loc})?)  \textbf{PerN}

\begin{table}
\begin{tabularx}{\textwidth}{lX}
\lsptoprule

\mdseries\itshape \ili{English} & \mdseries\itshape his beautiful \trigger{sister-in-law} from Athens, \textbf{Maria}\\
\mdseries\itshape \ili{Bulgarian} & \mdseries\itshape negovata krasiva \trigger{snaha} ot Atina, \textbf{Maria}\\
\mdseries\itshape \ili{French} & \mdseries\itshape sa jolie \trigger{belle-soeur} d’Athènes, \textbf{Maria}\\
\mdseries \itshape \ili{Greek} & \mdseries  η όμορφη \trigger{κουνιάδα} του από την Αθήνα, η \textbf{Μαρία}\\
\mdseries\itshape \ili{Serbian} & \mdseries\itshape njegova lepa \trigger{snaja} iz Atine, \textbf{Marija}\\
\lspbottomrule
\end{tabularx}

\caption{Syntactic pattern C – an example translated in the five languages.}

\end{table}



\subsection{Syntactic pattern D (single- and, rarely, multiword personal name)}

Characteristics shared by the five languages: i) The trigger appears
before the personal name but a complex trigger phrase often occurs in
apposition. Examples: \ili{English}: \textit{\trigger{Saint}}
\textbf{\textit{Haralambos}}\textit{, the Holy }\textit{\trigger{Martyr}}
\textit{of Magnesia}; \ili{Serbian}: \textit{\trigger{Sveti}} \textit{mučenik i
arhiđakon} \textbf{\textit{Lavrentije}} ‘\trigger{Saint} martyr and
archdeacon \textbf{Lavrentije}’. ii) If no modifiers or complements
exist, the trigger is indefinite (except for \ili{Greek} where the article is
obligatory); otherwise, it is definite. Examples: \ili{Bulgarian}:
\textit{\trigger{Sveti}}
\textbf{\textit{Nikola}}
‘\trigger{Saint}
\textbf{Nicholas}’; \ili{French}:
\textit{le}
\textit{\trigger{saint}}
\textit{de l'Arcadie:}
\textbf{\textit{Charles De Menou D'Aulnay}}
‘the \trigger{saint} of  Arcadia:
\textbf{Charles De Menou D'Aulnay}’.



\ili{English}: (DefArt Adj* \trigger{Trigger} (PP \textit{of}|\textit{from = Loc})?
\textbf{PerN}) | (DefArt Adj* \trigger{Trigger} \textbf{PerN} (PP
\textit{of}|\textit{from = Loc})?) | (\trigger{Trigger} \textbf{PerN }(PP
\textit{of}|\textit{from = Loc})?)



\ili{Bulgarian}: (((DefAdj (PossCL)? | DefPossPron) Adj* \trigger{Trigger} (PP
\textit{na}|\textit{ot = Loc})?) | (DefAdj Adj* \trigger{Trigger} (PP
\textit{na}|\textit{ot = Loc})?) | (\trigger{DefTrigger} (PP \textit{na}|\textit{ot} \textit{ =
Loc})?) | \trigger{IndefTrigger}) \textbf{PerN}



\ili{French}: (DefArt Adj* \trigger{Тrigger} (PP \textit{de = Loc})? \textbf{PerN})
| (\trigger{Trigger} Adj* \textbf{PerN }(PP \textit{de = Loc})?)



\ili{Greek}: (DefArt Adj* \trigger{Trigger} (GenP \textit{= Loc})? \textbf{PerN}) |
(DefArt \trigger{Trigger} \textbf{PerN }(GenP \textit{= Loc})? )



\ili{Serbian}: (PossPron? DefAdj* \trigger{Trigger} ((PP \textit{iz}|\textit{u}|\textit{sa =
Loc})? DefAdj? (N)?) \textbf{PerN }PossAdj? | \textbf{PerN }PossAdj?
(PP \textit{iz}|\textit{u}|\textit{sa = Loc})?)

\begin{table}
\begin{tabularx}{\textwidth}{lX}
\lsptoprule

\itshape \ili{English} & \itshape the Holy \trigger{Martyr} \textbf{Chrysostomos} of Smyrna\\
\itshape \ili{Bulgarian} & \itshape Svetiyat \trigger{măčenik} \textbf{Hrisostom} ot Smirna\\
\itshape \ili{French} & \itshape le Saint \trigger{Martyr} \textbf{Chryssostomos} de Smyrne\\
\itshape \ili{Greek} & ο Άγιος \trigger{Ιερομάρτυρας} \textbf{Χρυσόστομος} Σμύρνης\\
\itshape \ili{Serbian} & \itshape Sveti \trigger{mučenik} \textbf{Hrizostom} Smirnski\\
\lspbottomrule
\end{tabularx}

\caption{Syntactic pattern D – an example translated in the five languages.}
\end{table}

\subsection{Syntactic pattern E (family name or multiword personal name)}

Characteristics shared by the five languages: i) The external trigger
usually appears before the personal name, except for some triggers that
can also appear after the personal name. Examples: \ili{English}:
\textit{\trigger{Mr.}} \textbf{\textit{Smith}}; \ili{Serbian}: \textbf{\textit{Dušan
Rašković}}\textit{, }\textit{\trigger{dipl. oec}}\trigger{.}; ii) No
modifiers and/or complements are allowed. Example: \ili{French}:
\textit{\trigger{M.}} \textbf{\textit{Dupont}} ‘\trigger{Mr.}
\textbf{Dupont}’; iii) If the trigger is indefinite, an article is not
permissible (except for \ili{Greek} where the article is obligatory).
Example: \ili{English}: *\textit{(the) }\textit{\trigger{Dr.}}
\textbf{\textit{Livingstone}}; iv) \isi{personal names} can be extended with
one or more (rarely more than two) proper names: a nickname, a patronym
and/or a family name that form a complex name (MWE). Example: \ili{Bulgarian}:
\textit{\trigger{gospodin}} \textbf{\textit{Ivan Ivanov}}
‘\trigger{Mister} \textbf{Ivan Ivanov}’.



\ili{English}: \ili{Bulgarian}: \ili{French}: \trigger{IndefTrigger} \textbf{PerN}



\ili{Greek}: DefArt \trigger{Trigger} \textbf{PerN}



\ili{Serbian}: (\trigger{Trigger} \textbf{PerN) }\trigger{|} \textbf{(PerN
}\trigger{Trigger})

\begin{table}
\begin{tabularx}{\textwidth}{lX}
\lsptoprule

\itshape \ili{English} & \itshape \trigger{Dr.} \textbf{Mary Andrew Smith}\\
\itshape \ili{Bulgarian} & \itshape \trigger{d-r} \textbf{Meri Andryu Smit}\\
\itshape \ili{French} & \itshape \trigger{Dr} \textbf{Mary Andrew Smith}\\
\itshape \ili{Greek} &  η \trigger{Δρ.} \textbf{Μαίρη Άντριου Σμιθ}\\
\itshape \ili{Serbian} & \itshape \trigger{dr} \textbf{Meri Endru Smit}\\
\lspbottomrule
\end{tabularx}

\caption{Syntactic pattern E – an example translated in the five languages.}
\end{table}

\subsection{Syntactic pattern F (single- or multiword location name)}

Characteristics shared by the five languages: i) The trigger appears
before the personal name. A heavy trigger phrase is often found in
apposition. Examples: \ili{Greek}: η όμορφη \trigger{πόλη} των Παρισίων / του Παρισιού
‘the beautiful \trigger{city} of \textbf{Paris}’; η
\textbf{Σαντορίνη}, το πιο όμορφο
\trigger{νησί} της Ελλάδας ‘the \textbf{Santorini},
the most beautiful \trigger{island} in Greece’. ii) The phrase headed by
the trigger is definite. Examples: \ili{English}: \textit{the beautiful
}\textit{\trigger{city}} \textit{of artists,} \textbf{\textit{Plovdiv}};
\textit{our beautiful }\textit{\trigger{city}}\textit{,}
\textbf{\textit{Plovdiv}}; \ili{French}: \textit{la belle }\textit{\trigger{ville}}
\textit{des mille fontaines,} \textbf{\textit{Aix-en-Provence }}‘the
beautiful \trigger{city} of thousand \ fountains,
\textbf{Aix-en-Provence}’; iii) \is{location names} Location names can be MWEs. Examples:
\ili{Bulgarian}: \textit{\trigger{Hram-pametnik}} \textit{Sveti}
\textbf{\textit{Aleksandăr Nevski}} ‘\trigger{Cathedral} Saint
\textbf{Alexander Nevski}’; \ili{Serbian}: \textit{\trigger{nadošli}}
\textbf{\textit{Beli Timok}} ‘\trigger{rising} \textbf{Beli Timok}’.



\ili{English}: (((DefArt | GenDet | PossPron) Adj* \trigger{Trigger} (PP
\textit{of = Spec})? (PP \textit{in = Loc})?) | (DefArt
\trigger{Trigger}\textbf{ }(PP \textit{of = Spec})?) |
\trigger{IndefTrigger}) \textbf{LocN}



\ili{Bulgarian}: (((DefAdj (PossCL)? | DefPossPron) Adj* \trigger{Trigger} (PP \textit{na
= Spec})? (PP \textit{v}|\textit{na}|\textit{pri = Loc})?) | (\trigger{DefTrigger} (PP
\textit{na = Spec})? (PP \textit{v}|\textit{na}|\textit{pri = Loc})?) |
\trigger{IndefTrigger}) \textbf{LocN}



\ili{French}: (((DefArt | PossPron) Adj* \trigger{Trigger} (PP \textit{de = Spec})?
(PP \textit{de = Loc})?) |
(DefArt\trigger{ Trigger} Adj* (PP
\textit{de = Spec})?)) \textbf{LocN}



\ili{Greek}: (((DefArt | PossPron) Adj* \trigger{Trigger} (GenP = \textit{Spec})?
(PP σε \textit{= Loc})?) | (DefArt \trigger{Trigger} (GenP)?))
\textbf{LocN}



\ili{Serbian}: ((PossPron | PossAdj)? DefAdj* \trigger{Trigger} (PP
\textit{u}|\textit{na}|\textit{pri}|\textit{pod}|\textit{u blizini}| \textit{GenP} = \textit{Loc})?) \textbf{LocN}


\begin{table}
\begin{tabularx}{\textwidth}{lX}
\lsptoprule

\mdseries\itshape \ili{English} & \mdseries\itshape the most romantic \trigger{city} in the world, \textbf{Paris}\\
\mdseries\itshape \ili{Bulgarian} & \mdseries\itshape nay-romantičniyat \trigger{grad} v sveta, \textbf{Pariž}\\
\mdseries\itshape \ili{French} & \mdseries\itshape la plus romantique \trigger{ville} dans le monde, \textbf{Paris}\\
\mdseries\itshape \ili{Greek} & \mdseries η πιο ρομαντική \trigger{πόλη} του κόσμου, το \textbf{Παρίσι}\\
\mdseries\itshape \ili{Serbian} & \mdseries\itshape najromantičniji \trigger{grad} na svetu, \textbf{Pariz}\\
\lspbottomrule
\end{tabularx}

\caption{Syntactic pattern F – an example translated in the five languages.}
\end{table}



\subsection{Syntactic pattern G (multiword location name)}

Characteristics shared by the five languages: i) The internal trigger is
part of the location name, thus the location name is always a MWE.
Examples: \ili{Bulgarian}: \textit{našiyat hubav }\textit{\trigger{grad}}
\textbf{\textit{Novi h}}\textbf{\textit{\trigger{an}}} ‘\textit{our-the
beautiful }\textit{\trigger{city}} \textbf{\textit{Novi han}}’; \ili{Greek}:
\textbf{ο Ινδικός}\textbf{\trigger{Ωκεανός}} ‘the
\textbf{Indian }\textbf{\trigger{Ocean}}’; ii) A location name with an
internal trigger is fixed, the order of constituents cannot be changed
and insertions are not allowed. Examples: \ili{Bulgarian}: \textit{našata
}\textbf{\textit{Stara }}\textbf{\textit{\trigger{planina}}} ‘our-the
\textbf{Stara }\textbf{\trigger{Planina}}’,
\textit{*}\textbf{\textit{\trigger{Planina}}}
\textbf{s}\textbf{\textit{tara}}; \ili{French}: \textit{le célèbre
}\textbf{\textit{\trigger{Mont}}} \textbf{\textit{Blanc}} ‘the famous
\textbf{\trigger{Mont}} \textbf{Blanc}’, *\textbf{\textit{Blanc
}}\textbf{\textit{\trigger{Mont}}}. iii) The location name may contain a
personal or a location name (rarely an organization name). Example: \ili{English}:
\textbf{\textit{Minnesota }}\textbf{\textit{\trigger{River}}}; iv) The
internal trigger may be specified by the same range of modifiers and
complements permissible for the
trigger.\footnote{For simplicity, not all variants are  presented in
the syntactic pattern (applicable also to Syntactic
pattern I in §6.9).} Example: \ili{Bulgarian}: \textbf{\textit{Černi
}}\textbf{\textit{\trigger{vrăh}}} ‘\textbf{Black
}\textbf{\trigger{Peak}}’; v) External trigger can be added if different
from the internal one. Example: \ili{English}: \textbf{\textit{\trigger{City}}}
\textbf{\textit{of Colorado }}\textbf{\textit{\trigger{Springs}}}; vi)
Heavy trigger phrases often occur as appositions. Example: \ili{Serbian}:
\textbf{\textit{Novo Brdo}}\textit{, najveći rudarski
}\textit{\trigger{grad}} \textit{u Srbiji i na celom Balkanskom
poluostrvu} ‘\textbf{New Hill}, biggest mining \trigger{town} in Serbia
and on the entire Balkan Peninsula’.



\ili{English}: ((DefArt | GenDet | PossPron) Adj* \textbf{MWLocN }(PP \textit{in =
Loc})?) | (DefArt \textbf{MWLocN }(PP \textit{in = Loc})?) |
\textbf{MWLocN}



\ili{Bulgarian}: (((DefAdj (PossCL)?) | DefPossPron) Adj* \textbf{MWLocN} (PP \textit{v}|\textit{na}|\textit{pri = Loc})?) | (\textbf{DefMWLocN} (PP \textit{v}|\textit{na}|\textit{pri =
Loc})?) | \textbf{MWLocN}



\ili{French}: ((DefArt | PossPron) Adj* \textbf{MWLocN} (PP \textit{de = Loc})?) |
(\textbf{MWLocN} (PP \textit{de = Loc})?) | \textbf{MWLocN}



\ili{Greek}: ((DefArt | PossPron) Adj* \textbf{MWLocN} (GenP\textit{ = Loc})?) |
(\textbf{DefMWLocN} (GenP \textit{= Loc})?) | \textbf{MWLocN}



\ili{Serbian}: ((PossPron | PossAdj)? DefAdj* (PP \textit{u}|\textit{na}|\textit{pri}|\textit{pod}|\textit{u blizini }|
\textit{GenP = Loc})?) \textbf{MWLocN} | \textit{\ GenP = Loc})?) |
\textbf{MWLocN}

\begin{table}
\begin{tabularx}{\textwidth}{lX}
\lsptoprule
\mdseries\itshape \ili{English} & \mdseries\itshape the vast \textbf{Great} \textbf{\trigger{Plains}} in the United States\\
\mdseries\itshape \ili{Bulgarian} & \mdseries\itshape neobyatnite \textbf{Golemi} \textbf{\trigger{ravnini}} v Săedinenite štati\\
\mdseries\itshape \ili{French} & \mdseries\itshape les \textbf{Grandes} \textbf{\trigger{Plaines}} aux Etats-Unis\\
\mdseries\itshape \ili{Greek} & \mdseries τα \textit{\textbf{Great}}  \textit{\textbf{\trigger{Plains}}} της Αμερικής\\
\mdseries\itshape \ili{Serbian} & \mdseries\itshape prostrane \textbf{Velike} \textbf{\trigger{ravnice}} u Sjedinjenim Državama\\
\lspbottomrule
\end{tabularx}
\caption{Syntactic pattern G – an example translated in all five languages.}
\end{table}


\subsection{Syntactic pattern H (single- and multiword organization name)}

Characteristics shared by the five languages: i) The trigger may either
be placed before or after the organization name in case of apposition.
Examples: \ili{Bulgarian}:  \textit{hranitelnata
}\textit{\trigger{kompaniya}}\textit{ }\textbf{\textit{“Danon”}}
‘nutritional-the \trigger{company} \textbf{Danone’}; \ili{English}:
\textbf{\textit{Apache}} \textit{\trigger{Corp.}} ii) The phrase headed
by the trigger is definite. If the trigger is a  single-word one or specified
for domain, the trigger phrase may be indefinite (except for \ili{Greek}
where the article is obligatory). Examples: \ili{English}: \textit{the}\trigger{
}\textit{\trigger{company}} \textit{of Buffett,}
\textbf{\textit{Berkshire Hathaway}}\textit{;} \ili{Bulgarian}:
\textit{investicionen }\textit{\trigger{fond}}\textit{
“}\textbf{\textit{Razvitie”}} ‘investment \trigger{fund}
\textbf{Razvitie’}; \ili{French}: \textit{la nouvelle }\textit{\trigger{compagnie}}
\textbf{\textit{Santus }}‘the new \trigger{company}
\ \textbf{Santus}\textbf{’}, \textit{notre nouvelle
}\textit{\trigger{compagnie}} \textbf{\textit{Santus}} ‘our new
\trigger{company} \textbf{Santus}’\textit{, la nouvelle
}\textit{\trigger{compagnie}} \textit{de Pierre,} \textbf{\textit{Santus
}}‘the new \trigger{company} of Pierre, \textbf{Santus}’; iii) The
organization name can be a 
MWE – either a complex personal or location
name or a fixed multiword organization name. Examples: \ili{Bulgarian}:
\textit{novoto }\textit{\trigger{učilište}} \textit{za deca s uvreden
sluh v Sofiya }\textbf{\textit{“Prof. Dr. Dečo Denev” }}‘new-the school
for children with impaired hearing in Sofia \textbf{\textit{Prof. Dr.
Dečo Denev}}’; \ili{Greek}: η μεταλλευτική
\trigger{εταιρία} \textbf{Ελληνικός Χρυσός}
‘the mining \trigger{company} \textbf{Hellas Gold}’, η
\textbf{Ελληνικός Χρυσός}, μεταλλευτική
\trigger{εταιρία} ‘the \textbf{Hellas Gold} mining
\trigger{company}’; \ili{Serbian}: \textit{\trigger{Fabrika}} \textit{mašina
“}\textbf{\textit{Ivo Lola Ribar}}” ‘Machine \trigger{Factory}
“\textbf{Ivo Lola Ribar”}’.



\ili{English}: (((DefArt | GenDet | PossPron) Adj* \trigger{Trigger}\textbf{ }(PP
\textit{in}|\textit{at = DomSpec})? (PP \textit{in}\textit{at = Loc})?) | (DefArt Adj*
\trigger{Trigger}\textbf{ }(PP \textit{in}\textit{at = DomSpec})? (PP
\textit{in}|\textit{at = Loc})?) | \trigger{Trigger}) \textbf{OrgN}



\ili{Bulgarian}: (((DefAdj (PossCL)?) | DefPossPron) Adj* \trigger{Trigger} (PP
\textit{po}|\textit{na}|\textit{za = DomSpec})? (PP \textit{v}|\textit{na}|\textit{pri = Loc})?) |
((\trigger{DefTrigger} (PossCL)?) (PP \textit{po}|\textit{na}|\textit{za = DomSpec})? (PP
\textit{v}|\textit{na}|\textit{pri = Loc})?) | \trigger{Trigger}) \textbf{OrgN}



\ili{French}: (((DefArt | PossPron) Adj* \trigger{Trigger} (PP \textit{de =
DomSpec})?) | (DefArt Adj* \trigger{Trigger} (PP \textit{de = PersN})?) |
(DefArt \trigger{Trigger})) (PP \textit{σε }DefArt \textit{=}
\textbf{OrgN})



\ili{Greek}: ((DefArt Adj* \trigger{Trigge}r (PossPron)? ((PP για = \textit{
DomSpec?}) | (\textit{Gen = DomSpec})?) | (DefArt Adj* \trigger{Trigger}
(GenDet)?) | (DefArt \trigger{Trigger})) (PP \textit{σε} DefArt
\textit{=} \textbf{OrgN})



\ili{Serbian}: (PossPron? DefAdj* \trigger{Trigger} (PP \textit{za} | \textit{GenP =
DomSpec})? (PP \textit{u}|\textit{pri}| \textit{GenP = Aff})? (PP \textit{iz}|\textit{u}| \textit{GenP} =
\textit{Loc})? \textbf{OrgN)} | (\textbf{OrgN} \trigger{Trigger})

\begin{table}
\begin{tabularx}{\textwidth}{lX}
\lsptoprule

\mdseries\itshape \ili{English} & \mdseries\itshape China's investment \trigger{bank}, \textbf{China International Capital Corporation
Limited}\\
\mdseries\itshape \ili{Bulgarian} & \mdseries\itshape \textbf{kitayskata} investicionna \trigger{banka} \textbf{China International Capital Corporation Limited}\\
\mdseries\itshape \ili{French} & \mdseries\itshape la \trigger{Banque} d’Investissements en Chine, \textbf{China International Capital Corporation Limited}\\
\mdseries \itshape \ili{Greek} & η \trigger{Τράπεζα} Επενδύσεων της Κίνας, η \itshape \textbf{China International Capital Corporation Limited}\\
\mdseries\itshape \ili{Serbian} & \mdseries\itshape kineska investiciona \trigger{banka}, \textbf{Kineska međunarodna kapitalna korporacija}\\
\lspbottomrule
\end{tabularx}

\caption{Syntactic pattern H – an example translated in the five languages.}
\end{table}



\subsection{Syntactic pattern I (multiword organization name)}

Characteristics shared by the five languages: i) The trigger is an
integral part of the organization name, thus the organization name is
always a MWE. Examples: \ili{English}: \textbf{\textit{the European
}}\textbf{\textit{\trigger{Bank}}} \textbf{\textit{for Reconstruction and
Development }}\textit{in Serbia;} \textbf{\textit{the
}}\textbf{\textit{\trigger{Association}}} \textbf{\textit{of Chartered
Certified Accountants;}} \ili{Greek}: η
\textbf{\trigger{Τράπεζα}}\textbf{ Εμπορίου και
Ανάπτυξης της Μαύρης Θάλασσας }‘\textbf{the Black Sea Trade and
Development }\textbf{\trigger{Bank}}’. ii) Organization names containing
an integral trigger are fixed, the order of constituents cannot be
changed and insertions are not allowed. Examples: \ili{Bulgarian}:
\textbf{\textit{Evropeyska }}\textbf{\textit{\trigger{banka}}}
\textbf{\textit{za văzstanovyavane i razvitie}} ‘\textbf{European
}\textbf{\trigger{Bank}} \textbf{for Reconstruction and Development}’;
\textit{novosăzdadeniyat} \textbf{\textit{Evropeyski
}}\textbf{\textit{\trigger{fond}}} \textbf{\textit{za strategičeski
investicii }}‘newly-found-the \textbf{European }\textbf{\trigger{Fund}}
\textbf{for Strategic Investment}'; iii) organization names with an
integral trigger can contain a personal, location or organization name.
Example: \ili{Serbian}: \textbf{\textit{Memorijalni
}}\textbf{\textit{\trigger{centar}}} “\textbf{\textit{Josip Broz Tito}}”
‘\textbf{Memorial }\textbf{\trigger{Center}} “\textbf{Josip Broz Tito}”’;
iv) The internal organization trigger can be specified by the same
range of modifiers and complements permissible for it in a regular use.
Example: \ili{French}: \textbf{\textit{\trigger{l’Association}}}
\textbf{\textit{des Historiens}} ‘\textbf{the
}\textbf{\trigger{Association}} \textbf{of Historians}’; v) Rarely, an
organization trigger, different from the integral trigger, can specify
the multiword organization name. Examples: \ili{Bulgarian}:
\textbf{\textit{\trigger{Săyuz}}} \textbf{\textit{na tărgovcite v
Bălgariya }}‘\textbf{\trigger{Union}} \textbf{of traders-the in
Bulgaria}’ ; \textit{\trigger{Asociaciya}}
“\textbf{\textit{\trigger{Săyuz}}} \textbf{\textit{na tărgovcite v
Bălgariya” }}‘\trigger{Association} \textbf{\trigger{Union}} \textbf{of
traders-the in Bulgaria}’.



\ili{English}: (DefArt | PossPron) Adj* \textbf{MWOrgN }(PP \textit{at}|\textit{in = Loc})?)
| (\textbf{DefMWOrgN }(PP \textit{at}|\textit{in = Loc})?) | \textbf{MWOrgN}



\ili{Bulgarian}: (((DefAdj (PossCL)?) | DefPossPron) Adj* \textbf{MWOrgN} (PP
\textit{v}|\textit{na}|\textit{pri = Loc})?) | (\textbf{DefMWOrgN} (PP \textit{v}|\textit{na}|\textit{pri =
Loc})?) | \textbf{MWOrgN}



\ili{French}: (DefArt | PossPron) Adj* \textbf{MWOrgN} (PP \textit{à} \textit{de = Loc})?



\ili{Greek}: ((DefArt | PossPron) Adj* \textbf{MWOrgN} (PP σε \textit{= Loc})?) |
(DefArt \textbf{MWOrgN} (GenP = \textit{Loc})?) | \textbf{MWOrgN}



\ili{Serbian}: (PossPron? DefAdj* \textbf{MWOrgN} (GenP = \textit{Poss})? (PP \textit{iz}|\textit{u}| \textit{GenP} = \textit{Loc})?) | \textbf{MWOrgN}

\begin{table}
\begin{tabularx}{\textwidth}{lX}
\lsptoprule

\mdseries\itshape \ili{English} & \mdseries\itshape \textbf{World} 
\textbf{\trigger{Association}} \textbf{for Small and Medium Enterprises}\\
\mdseries\itshape \ili{Bulgarian} & \mdseries\itshape \textbf{Svetovna} 
\textbf{\trigger{asociaciya}} \textbf{za malki i sredni predpriyatiya}\\
\mdseries\itshape \ili{French} & \mdseries\itshape \textbf{\trigger{Association}} 
\textbf{Mondiale des Petites et Moyennes Entreprises}\\
\mdseries \itshape \ili{Greek} & \textbf{Διεθνής} 
\textbf{\trigger{Σύνδεσμος}} \textbf{Μικρομεσαίων Επιχειρήσεων}\\
\mdseries\itshape \ili{Serbian} & \mdseries\itshape \textbf{Svetsko}
\textbf{\trigger{udruženje}} \textbf{malih i srednjih preduzeća}\\
\lspbottomrule
\end{tabularx}
\caption{Syntactic pattern I – an example translated in all five languages.}
\end{table}






\section{Comparison of the five languages}

At the semantic level, languages do not differ (significantly). The
\isi{semantic patterns} of proper names define the common semantics,
regardless of the language in which it is realized: semantic patterns
are language-neutral. Languages differ in lexical and phrasal
categories, constituency, word order permutations and alterations. The
differences in word order and alterations insert some nuances in the
expressed meaning, i.e., the viewpoint of the speaker, but they do not
alter the general meaning. The syntactic patterns of proper names show
the correspondences among languages at the syntactic level: syntactic
patterns are language-specific. A semantic pattern is a representation
that can be linked with different syntactic frames in different
languages and, vice versa, syntactic patterns from different languages
may share a single semantic pattern. Thus, syntactic patterns make
explicit the similarities and differences in the grammatical structure
of the five languages.



The structure of the language-neutral semantic and language-specific
syntactic patterns can be represented as a graph whose nodes are
semantic and syntactic patterns while the arcs represent different
languages. More than one language-specific syntactic pattern may be
linked to one language-neutral semantic frame; in such a case,
syntactic patterns are synonymous to the extent that they represent a
common semantic structure. Through this type of representation, we
offer an interlingual mapping of the syntactic structures of \isi{named
entities} in the five languages. Some of the most distinctive
grammatical characteristics of NEs in \ili{English}, \ili{Bulgarian}, \ili{French}, \ili{Greek}
and \ili{Serbian} with respect to the single and multiword morphology and
syntax will be outlined below.

\subsection{Grammatical categories of dependent constituents} %7.1. /

The syntactic patterns of \ili{English} proper {\isi{name triggers} involve
combinations of adjectival modifiers in pre-nominal position, one or
several (they can be preceded by a definite article) (‘\textit{the
great }\textit{\trigger{poet}} \textbf{\textit{Burns}}’); possessive
pronoun modifiers in pre-nominal position (‘\textit{Welcome our new
}\textit{\trigger{professor}}\textit{, }\textbf{\textit{Jennifer S.
Locke}}\textit{!}’); prepositional complements in post-nominal
position; a noun modifier in pre-nominal position, alternating with a
genitive determiner and a prepositional phrase, e.g., \textit{the Grieg
}\textbf{\textit{Piano }}\textbf{\textit{\trigger{Concerto}}} vs.
\textit{Grieg’s }\textbf{\textit{Piano
}}\textbf{\textit{\trigger{Concerto}}} vs. \textit{the
}\textbf{\textit{Piano }}\textbf{\textit{\trigger{Concerto}}} \textit{by
Grieg.}



In \ili{Bulgarian}, the syntactic patterns for proper {\isi{name triggers} exhibit
the following combinations: adjectival modifiers in pre-nominal
position (\textit{noviyat }\textit{\trigger{predseda-}} \textit{\trigger{tel}} \textbf{\textit{Petrov}} `new-the \trigger{chair}\textbf{ Petrov}');
possessive pronoun modifiers in pre-nominal position alternating with a
possessive pronoun clitic in post-nominal position (\textit{moyata
}\textbf{\textit{sestra}} \textbf{Ana} `my-the \trigger{sister}\textbf{
Ana}' vs. \textit{\trigger{sestra}}\textit{ mi }\textbf{\textit{Ana}}
'\trigger{sister} my.PossCL \textbf{Ana}'); prepositional complements in
post-nominal position (\textit{\trigger{kompaniyata}}\textit{ na Ivan}
“\textbf{Elit”} `\trigger{company}-\trigger{the} of Ivan \textbf{“Elit”}');
and a noun modifier in pre-nominal position, alternating with a PP
(\textit{karate }\textit{\trigger{instruktorăt}}\textbf{\textit{ Ivan}}
`karate \trigger{trainer-the}\textbf{ Ivan}' vs.
\textit{\trigger{instructo}}\textbf{\textit{\trigger{răt}}}\textbf{\textit{
}}\textit{po karate }\textbf{\textit{Ivan}} `\trigger{trainer-the} in
karate \textbf{Ivan}').



In \ili{French}, the phrase headed by a trigger is definite (with an
alternation of the phrase with a definite article or possessive
pronoun: \textit{la belle }\textit{\trigger{ville}} \textit{des Mille
Fontaines, }\textbf{\textit{Aix-en-Provence}} ‘the beautiful
\trigger{city} of thousand fountains, \textbf{Aix-en-Provence}’;
\textit{notre belle }\textit{\trigger{ville}}\textit{,}
\textbf{\textit{Paris}} ‘our beautiful \trigger{city}, \textbf{Paris}’).
The proper name can be introduced by a preposition: \textit{la belle
}\textit{\trigger{ville}} \textit{de }\textbf{\textit{Paris}} ‘the
beautiful \trigger{city} of \textbf{Paris}’; \textit{notre belle
}\textit{\trigger{ville}} \textit{de }\textbf{\textit{Paris}} ‘our
beautiful \trigger{city} of \textbf{Paris}’.



In \ili{Greek}, simple and multiword proper names are preceded by a definite
article, e.g., το \textbf{Παρίσι} ‘the
\textbf{Paris}’, οι \textbf{Ηνωμένες Πολιτείες}
\textbf{Αμερικής} ‘the \textbf{United States of America}’. The
phrase headed by the trigger is also definite: η όμορφη
\trigger{πόλη} των καταρρακτών, \textbf{η
Έδεσσα} ‘the beautiful \trigger{city} of waterfalls, the
\textbf{Edessa}’; η όμορφή μας \trigger{πόλη},
η \textbf{Έδεσσα} ‘the beautiful our.PossCL \trigger{city}, the
\textbf{Edessa}’. A location name can be put in the genitive case:
η όμορφη \trigger{πόλη} του
\textbf{Παρισιού} ‘the beautiful \trigger{city} of
\textbf{Paris}’. In that case, the use of the possessive pronoun clitic
is not possible: *η όμορφή μας \trigger{πόλη}
του \textbf{Παρισιού} ‘the beautiful our.PossCL
\trigger{city} of \textbf{Paris}’.



The syntactic patterns for \ili{Serbian} names comprise combinations of
adjectival modifiers in pre-nominal position (\textbf{\textit{Američka
filmska }}\textbf{\textit{\trigger{akademija}}} ‘\textbf{American Film
}\textbf{\trigger{Academy}}’); sometimes they can be found in post-nominal position
(\textit{\trigger{Episkop}} \textit{niški }\textbf{\textit{Irinej}}
‘\trigger{Bishop} of-Niš \textbf{Irinej}’); prepositional complements in
post-nominal position (\textbf{\textit{\trigger{Azotara}}}
\textbf{\textit{u Pančevu}} ‘\textbf{Fertilizer }\textbf{\trigger{Plant}}
\textbf{in Pančevo}’); complements in genitive (alternating with a
prepositional phrase) – \textbf{\textit{\trigger{Dom}}}
\textbf{\textit{zdravlja}} ‘\textbf{\trigger{House}}
\textbf{(of)-Health}.Gen (Community Health Center)’ and
\textbf{\textit{\trigger{Direkcija}}} \textbf{\textit{za upravljanje
oduzetom imovinom}} ‘\textbf{\trigger{Directorate}} \textbf{for
Management of Seized Assets}’.



Coordinated phrases are possible in all languages, e.g., \ili{Bulgarian}:
\textit{\trigger{vicepremierăt}} \textit{i }\textit{\trigger{ministăr}}
\textit{na obrazovanieto i naukata, }\textbf{\textit{Meglena Kuneva
}}‘Deputy \trigger{Prime Minister} \trigger{-the} and \trigger{Minister} of
Education-the and Science-the, \textbf{Meglena Kuneva}’\textit{; }\ili{French}:
\textbf{\textit{Martin Vetterli}}\textit{, }\textit{\trigger{professeur}}
\textit{à l'Ecole polytechnique fédérale de Lausanne et} \textit{\trigger{prési-}} \textit{\trigger{dent}} \textit{du Conseil national de la recherche
‘}\textbf{Martin Vetterli}, \trigger{Professor} at the Federal
Polytechnic School of Lausanne and \trigger{President} of the National
Research Council’; \ili{Greek}: ο \trigger{πρωθυπουργός}
και \trigger{πρόεδρος} του ΣΥΡΙΖΑ
\textbf{Αλέξης Τσίπρας} ‘the \trigger{prime minister} and
\trigger{head} of Syriza \textbf{Alexis Tsipras}’; \ili{Serbian}:\textit{
}\textbf{\textit{Hleb i kifle}} ‘\textbf{Bread and Rolls’ }(an
organization name). Both triggers and proper names can appear in a
coordinated construction (we do not encode coordination in the
syntactic patterns).



\subsection{Definiteness} %7.2. /

Definiteness is expressed either by a morpheme as in \ili{Bulgarian} and
\ili{Serbian}, or by an article as in \ili{English}, \ili{French}, and \ili{Greek}. In \ili{English},
\ili{French}, and \ili{Greek}, the definite article precedes the trigger, e.g.,
\textit{le }\textit{\trigger{premier ministre}}\textit{,
}\textbf{\textit{Justin Trudeau}} ‘the \trigger{prime minister},
\textbf{Justin Trudeau}’. There are other means to express definiteness
– i.e., the demonstrative pronouns, the possessive pronouns in \ili{English},
\ili{French}, and \ili{Serbian}, e.g., \ili{French}: \textit{notre belle
}\textit{\trigger{ville}}\textit{, }\textbf{\textit{Paris}} ‘our
beautiful \trigger{city}, \textbf{Paris}’; \ili{Serbian}: \textit{njeno rodno
}\textit{\trigger{mesto}} \textbf{\textit{Beograd}} ‘her native
\trigger{city} \textbf{Belgrade}’.



With \isi{personal names} in \ili{English}, the definite article is obligatory when
it is modified by an adjective and/or a PP complement: \textit{the
great }\textit{\trigger{poet}} \textbf{\textit{Burns, }}\textit{the
Scottish }\textit{\trigger{poet}}\textit{ }\textbf{\textit{Burns;}}
\textit{the }\textit{\trigger{poet}}\textit{ from Kosovo,
}\textbf{\textit{Fahredin Shehu}}; \textit{the }\textit{\trigger{author}}
\textit{of the Concerto, }\textbf{\textit{Edvard Grieg}}. The
possessive pronoun, the article and the genitive determiner are in
complementary distribution in \ili{English}.



The definite form in \ili{Bulgarian} is required when the trigger is modified
by an adjective or a possessive pronoun (in this case the definite
adjective is part of the first phrasal constituent: \textit{noviyat
}\textit{\trigger{ministăr}} \textbf{\textit{Valentin Dimitrov}} ‘new-the
\trigger{minister} \textbf{Valentin Dimitrov}’, and / or a prepositional
phrase; if there are no pre-nominal modifiers, the article is on the
trigger word: \textit{\trigger{ministărăt}} \textit{na finansite
}\textbf{\textit{Valentin Dimitrov}} ‘\trigger{minister-the} of
finance-the \textbf{Valentin Dimitrov}’).



In \ili{Greek}, all proper names are preceded by a definite article, e.g.,
η \textbf{Μαρία} ‘the \textbf{Maria}’, η
\textbf{Αθήνα}} ‘the \textbf{Athens}’, ο
\trigger{πρωθυπουργός} \textbf{Αλέξης Τσίπρας} ‘the
\trigger{Prime} \trigger{Minister} \textbf{Alexis Tsipras}’.



In \ili{Serbian}, the definite article is not used; furthermore neither
possessive pronouns nor adjectives are obligatory. However, when
adjectives precede proper names, they are in definite form, e.g.,
\textit{od izvesnog }\textbf{\textit{Stevice Miletića}} vs. \textit{*od
izvesna }\textbf{\textit{Stevice Miletića}} ‘from certain
\textbf{Stevica Miletić}’.


\subsection{Distribution of clitics in \ili{Bulgarian} and \ili{Greek}} %7.3. /

The possessive pronoun clitics in \ili{Bulgarian} are right-adjacent to the
definite article, e.g., in the second position in the noun phrase
(\textit{noviyat ni }\textit{\trigger{ministăr}} \textbf{\textit{Valentin
Dimitrov}} ‘new-the our.PossCL \trigger{minister} \textbf{Valentin
Dimitrov}’ vs. \textit{\trigger{ministărăt}} \textit{ni na finansite
}\textbf{\textit{Valentin Dimitrov}} ‘\trigger{minister-the} our.PossCL
of finance-the \textbf{Valentin Dimitrov}’). The possessive pronoun
clitic in \ili{Bulgarian} is also right-adjacent to an indefinite kinship
term if used without an adjectival modifier as in
\textit{\trigger{mayka}} \textit{mi} \textbf{\textit{Maria}}
‘\trigger{mother} my.PossCL \textbf{Maria}’.



In \ili{Bulgarian} the interrogative particle \textit{li} (which is always a
clitic) may appear after the first definite modifier (if not followed
by a possessive pronoun clitic) or after the whole NP, as in:
\textit{noviyat li }\textit{\trigger{direktor}} \textbf{\textit{Ivanov}}
‘new-the li.QuCL \trigger{d}\trigger{irector}\textbf{ Ivanov}’ and
\textit{noviyat }\textit{\trigger{direktor}}\textit{
}\textbf{\textit{Ivanov}}\textit{ li} `new-the
\trigger{d}\trigger{ir}\trigger{ector} \textbf{Ivanov} li.QuCL'). The
above-stated rules for the definite article, possessive pronoun clitic
and interrogative particle hold for the \isi{multiword names} too, with the
leftmost adjective being part of the proper name itself
(\textbf{\textit{Bălgarskata narodna }}\textbf{\textit{\trigger{banka}}}
‘\textbf{\ili{Bulgarian}-the National }\textbf{\trigger{Bank}}’).



\ili{Greek} pronoun clitics are right-adjacent to the proper name, e.g.,
η \textbf{Μαρία} μου ‘the \textbf{Maria
}my.PossCL’. Once there is a trigger followed by the proper name, the
possessive pronoun clitic is between the trigger and the proper name,
e.g., ο \trigger{καθηγητής} μας
\textbf{Χρήστος Τσολάκης} ‘the \trigger{professor} our.PossCL
\textbf{Christos Tsolakis}’.



In \ili{Serbian}, pronoun clitics can sometimes be used to express possession,
as in \textit{\trigger{komšija}} \textit{mi }\textbf{\textit{Asan}}
‘\trigger{neighbor} I.CL \textbf{Asan }(my neighbor Asan)’. However,
these constructions are rarely used, being considered rather obsolete
and non-standard and are therefore not included in the patterns.

\subsection{Expression of semantic and grammatical dependencies} %7.4. /

Prepositions are used to express semantic and grammatical dependencies,
such as affiliation, domain specification, location in \ili{English},
\ili{Bulgarian}, \ili{French}, \ili{Greek}, and \ili{Serbian}, and possession in \ili{English},
\ili{Bulgarian}, and \ili{French}. Semantic and grammatical dependencies can be
signified by cases in \ili{Greek} and \ili{Serbian}. In \ili{English}, possession may
also be expressed by a clitic –’\textit{s }(marking the genitive
determiner), and in \ili{Bulgarian} by the derivational suffix of possessive
adjectives.


\subsection{Word order – position of the trigger with respect to the proper noun} %7.5. /


In \ili{French}, \ili{Greek} and \ili{Serbian}, word order permutations are common for
\isi{personal names}, as the first name and surname(s) can change places: \ili{French}:
\textbf{\textit{Nicolas Sarkozy}} vs. \textbf{\textit{Sarkozy
Nicolas;}} \ili{Greek}: \textbf{Γεώργιος Κοκκινόπουλος}
‘\textbf{Georgios Kokkinopoulos}’ vs. \textbf{Κοκκινόπουλος
Γεώργιος} ‘\textbf{Kokkinopoulos Georgios}’; \ili{Serbian}: \textbf{\textit{Marko
Vitas}} vs. \textbf{\textit{Vitas Marko}}. In \ili{Serbian}, a change of the
order of the first name and the surname(s) of male persons results in a
change of the syntactic properties, as in the former case both names
inflect, while in the latter only the first name inflects, e.g., in the
genitive \textbf{\textit{Marka Vitasa}} vs. \textbf{\textit{Vitas
Marka}}.



In all languages, the trigger can appear in pre- or post-nominal
position: \ili{French}: \textit{le }\textit{\trigger{ministre}} \textit{des
Finances et des Comptes publics, }\textbf{\textit{Michel Sapin }}‘the
\trigger{Minister} of finance and of public accounts, \textbf{Michel
Sapin}’,\textbf{ }or \textbf{\textit{Michel Sapin}}\textit{,
}\textit{\trigger{ministre}} \textit{des Finances et des Comptes publics
}‘\textbf{Michel Sapin}, \trigger{Minister} of finance and of public
accounts’; \ili{Greek}: ο \trigger{Υπουργός} Οικονομικών
\textbf{Ευκλείδης Τσακαλώτος} ‘the \trigger{Minister}
of finance \textbf{Efkleidis Tsakalotos}’, or ο
\textbf{Ευκλείδης Τσακαλώτος},
\trigger{Υπουρ-} \trigger{γός} Οικονομικών ‘the \textbf{Efkleidis
Tsakalotos}, \trigger{Minister} of finance’. Some abbreviations can
appear only before or after the names, as in: \ili{Serbian}: \textit{\trigger{JP}}
“\textbf{\textit{Srbijašume”}} ‘PC (Acronym for Public Company)
Srbijašume’ but \textbf{\textit{Takovo}} \textit{d.o.o.}
‘\textbf{Takovo} (a place name) d.o.o.’.



In all languages, a complex trigger phrase is often in apposition (when
the trigger appears as an apposition, it is always separated with a
comma), e.g., \ili{English}: \textbf{\textit{Chris}}\textit{, the new
}\textit{\trigger{Professor}} \textit{of Agriculture and Forestry}; \ili{French}:
\textit{le }\textit{\trigger{ministre}} \textit{des Finances et des
Comptes publics, }\textbf{\textit{Michel Sapin }}‘the \trigger{Minister}
of finance and of public accounts, \textbf{Michel Sapin}’,\textbf{ }or
\textbf{\textit{Michel Sapin}}\textit{, }\textit{\trigger{ministre}}
\textit{des Finances et des Comptes publics }‘\textbf{Michel
Sapin},\textbf{ }\trigger{Minister} of finance and of public accounts’;
\ili{Greek}: o \trigger{Καθηγητής} \textbf{Γεώργιος
Μπαμπινιώτης} ‘the \trigger{Professor} \textbf{Georgios Babiniotis}’ or
o \textbf{Γεώργιος Μπαμπινιώτης},
\trigger{Καθηγητής} ‘the \textbf{Georgios Babiniotis},
\trigger{Professor}’.



In all languages, the head personal name can be specified by more than
one triggers in a preferred order of appearance: \ili{English}:
\textit{\trigger{Director}} \textit{General }\textit{\trigger{Prof.}}
\textbf{\textit{Smith}}; \ili{Bulgarian}: \textit{generalniyat
}\textit{\trigger{direktor}} \textit{\trigger{prof.}} \textbf{\textit{Smit
}} `general-the \trigger{director} \trigger{prof.} \textbf{Smith}';\textit{
}\ili{French}: \textit{\trigger{Directeur}} \textit{général }\textit{\trigger{Prof.}}
\textbf{\textit{Smith }}`\trigger{Director} General \trigger{Prof.}
\textbf{Smith}'; \ili{Greek}: ο Γενικός \trigger{Διευθυντής Καθηγητής}} \textbf{Σμιθ} `the General
\trigger{Director} \trigger{Professor} \textbf{Smith}'; \ili{Serbian}:
\textit{Generalni }\textit{\trigger{direktor}} \textit{\trigger{prof.}}
\textbf{\textit{Smit }}`General \trigger{director} \trigger{prof.}
\textbf{Smith}'.



\subsection{Alternations} %7.6. /

In \ili{English}, genitive determiners may alternate with a possessive
prepositional phrase.

The possessive PP in \ili{Bulgarian} may alternate with a possessive or
relational adjective in pre-position (\textit{\trigger{stolicata}}
\textit{na Italiya} \textbf{\textit{Rim}} ‘\textit{\trigger{capital-the}}
\textit{of Italy }\textbf{\textit{Rome}}’\textit{, italianskata}\trigger{
}\textit{\trigger{stolica}} \textbf{\textit{Rim}} ‘\textit{Italian-the
}\textit{\trigger{capital}} \textbf{\textit{Rome’}}). Alternations of
possessive pronouns and possessive pronoun clitics in \ili{Bulgarian} are
also observed. A noun modifier in pre-nominal position can alternate
with a prepositional phrase (\textit{ski }\textit{\trigger{instuktorăt}}
`ski \trigger{instructor-the}' vs. \textit{\trigger{instruktorăt}}
\textit{po ski }`\trigger{instructor-the} at ski').



In \ili{Greek}, the genitive phrase may alternate with the preposition
\textit{σε} ‘at’ followed by accusative case, e.g.,
\textbf{Γεώργιος Μπαμπινιώτης},
\trigger{Καθηγητής} του Πανε-πιστημίου Αθηνών
‘\textbf{Georgios Babiniotis}, \trigger{Professor} of the University of
Athens’ or \textbf{Γεώργιος Μπαμπινιώτης,}
\trigger{Καθηγητής} στο Πανεπιστήμιο Αθηνών ‘\textbf{Georgios Babiniotis}, \trigger{Professor} at the University of
Athens’. In this case, the two structures may convey a different
meaning. The alternation is not possible for all proper names, e.g.,
\textbf{Αλέξης Τσίπρας},
\trigger{Πρωθυπουργός} της Ελλάδας ‘\textbf{Alexis
Tsipras}, \trigger{Prime} \trigger{Minister} of the Greece’,
*\textbf{Αλέξης Τσίπρας},
\trigger{Πρωθυπουργός} στην Ελλάδα ‘\textbf{Alexis
Tsipras}, \trigger{Prime Minister} in the Greece’. A location proper name
describing residency at a continent, a country or a city may alternate
as an adjective modifier or a PP complement attached to the trigger of
a personal name (the same is true for \ili{English}, \ili{Bulgarian}, and \ili{Serbian}),
e.g., \ili{Greek}: ο \trigger{Πρωθυπουργός} της Ελλάδας
‘the \trigger{Prime Minister} of the Greece’ or ο Έλληνας
\trigger{Πρωθυπουργός} ‘the \ili{Greek} \trigger{Prime Minister}’; \ili{Serbian}:
\textit{\trigger{Ambasada}} \textit{Grčke} ‘\trigger{Embassy} of Greece’
vs. \textit{Grčka }\textit{\trigger{ambasada}}\textit{ ‘}\ili{Greek}
\trigger{Embassy}\textit{’} (while in \ili{French}, the adjective follows the
trigger:\textit{ le }\textit{\trigger{président}}\textit{ français
}\textbf{\textit{François Mitterrand}} ‘the \trigger{president}
of-\ili{French}.Adj \textbf{Francois Mitterrand}’).

In \ili{French}, we may have an alternation of the preposition \textit{de}
‘of’ with the preposition \textit{à} ‘at’, e.g., \textbf{\textit{Martin
Vetterli}}, \textit{\trigger{professeur}} \textit{de l'École
polytechnique fédérale de Lausanne} `\textbf{Martin Vetterli},
\trigger{professor} of the Federal Polytechnic School of Lausanne' or
\textbf{\textit{Martin Vetterli}}, \textit{\trigger{professeur}}
\textit{à l'École polytechnique fédérale de Lausanne }`\textbf{Martin
Veterli}, \trigger{professor} at the Federal Polytechnic School of
Lausanne'.




In \ili{Serbian}, syntactic alternations are permissible, to some extent, with
\isi{organization names}: a complement in the genitive case instead of a PP complement
(\textbf{\textit{\trigger{Ministarstvo}}} \textbf{\textit{rada i socijalne politike}} `\trigger{Ministry} of Labor and
Social Policy’ instead of a \textbf{\textit{\trigger{Ministarstvo}}}
\textbf{\textit{za rad i socijalnu politiku}}
`\trigger{Ministry} for Labor and Social Policy’).

The features of the triggers in the five languages are summarized in Table \ref{tab:10}.

\begin{table}
%\begin{tabularx}{llllll}
\begin{tabularx}{\textwidth}{Xccccc}
\lsptoprule
Features that concern the trigger &  \ili{English} &  \ili{Bulgarian} &  \ili{French} &  \ili{Greek} &  \ili{Serbian}\\
adjective in pre-position &  + &  + &  + &  + &  +\\
adjective in post-position &  {}- &  {}- &  + &  {}- &  +\\
PP in post-position &  + &  + &  + &  + &  +\\
genitive phrase in post-position &  {}- &  {}- &  {}- &  + &  +\\
genitive phrase in pre-position &  {}- &  {}- &  {}- &  + &  {}-\\
genitive determiner in pre-position &  + &  {}- &  {}- &  {}- &  {}-\\
definite article &  + &  {}- &  + &  + &  {}-\\
definite morpheme &  {}- &  + &  {}- &  {}- &  +\\
obligatory definiteness with modifier Adj / PP in a sentence &  + &  + &  + &  + &  +\\
possessive pronoun clitic &  {}- &  + &  {}- &  + &  {}-\\
dependencies expressed by prepositions &  + &  + &  + &  + &  +\\
dependencies expressed by cases &  {}- &  {}- &  {}- &  + &  +\\
analytically expressed dependencies &  + &  + &  + &  + &  +\\
genitive determiner and PP alternation &  + &  {}- &  {}- &  {}- &  {}-\\
genitive and PP alternation &  {}- &  {}- &  {}- &  + &  +\\
poss. pronoun and possessive clitic alternation &  {}- &  + &  {}- &  {}- &  {}-\\
noun modifier and PP alternation &  + &  + &  {}- &  + &  +\\
Features that concern the whole NE &  \ili{English} &  \ili{Bulgarian} &  \ili{French} &  \ili{Greek} &  \ili{Serbian}\\
pre-position of the trigger &  + &  + &  + &  + &  +\\
apposition &  + &  + &  + &  + &  +\\
interrogative clitic &  {}- &  + &  {}- &  {}- &  {}-\\
\lspbottomrule
\end{tabularx}
\caption{Comparison of the morphological and syntactic features of the five languages.}
\label{tab:10}
\end{table}


\section{Conclusion}

The semantic classification and the syntactic patterns of single and
\isi{multiword names} in \ili{Bulgarian}, \ili{English}, \ili{French}, \ili{Greek}, and \ili{Serbian}, may
provide reliable data for rule-based Named Entity Recognition (NER).



Linguistic features and distribution facts are used to identify MWEs in
NER tasks – both in handcrafted rule-based systems that rely heavily on
linguistic knowledge, and in machine-learning techniques. In their
research on the application of MWEs and NEs in keyphrase extraction,
\cite{nagy2011} conclude that previously known noun compounds are
beneficial in NER, and that identified NEs enhance MWE detection, as
noun compounds and multiword NEs are linguistically similar and
sometimes it is not easy to distinguish between the two.



These arguments are further supported by the tagging practice where both
compound nouns and multiword NEs are often tagged as nouns, as their
linguistic behaviour is similar to that of single-word nouns \citep{vincze2011}. Approaches such as that of \cite{nagy2011} also use
features involving NEs or pertaining to NEs (i.e., orthography and
semantics of keyphrase candidates; positions of a token belonging to a
specific NE class, as certain classes of NEs can be identified by their
position in the beginning, in the middle or at the end of a keyphrase
candidate). \cite{galicia2004} discuss the (Spanish) composite
NEs (titles of books, movies, songs, etc.) that are described in terms
of syntactic and semantic features and of \ local context and consider
discourse features such as introductory words, prepositions,
redundancy; specific sets of names, etc.



Rule-based systems usually rely on large-scale lexical resources and
grammars, often in the form of regular expressions or Finite State
Transducers (\citealt{savary2011}; \citealt{maurel2011}). Much work
has been done on rule-based NER for the five languages discussed in
this paper, although machine learning methods prevail. A set of general
NER rules with reasonable accuracy has been developed for rule-based
annotation of NEs in \ili{Bulgarian} \citep{karagiozov2012}, \ili{French} \citep{maurel2011}, \ili{Greek} \citep{farmakiotou2000},  and \ili{Serbian} \citep{krstev2013}. \citet{krstev2007} discuss semantic and morphological
(derivational and inflectional) properties of proper names in \ili{Serbian}
(plus \ili{French} and \ili{English}) taking into account the significance of
regular derivation and the properties and function of possessive and
relational adjectives produced from proper names. \cite{koeva2015} discuss a strategy for a linguistic description and
classification of \ili{Bulgarian} NEs referring to persons, and their
application in several resources (lexicons and an annotated corpus) for
the definition and evaluation of a set of NER rules.



The syntactic patterns presented in this paper are formulated as rules
comprising morphological characteristics and syntactic dependencies
related to the semantic properties of personal, location and
organization NEs in \ili{Bulgarian}, \ili{English}, \ili{French}, \ili{Greek}, and \ili{Serbian}. We
intend to further exploit the formally encoded linguistic information
in rule-based NER approaches. Moreover, as the syntactic patterns for
different languages are linked to the same semantic pattern, they can
be considered equivalent at the conceptual level and may be applied to
any task that involves multilingual processing: cross-lingual
information extraction and text classification, multilingual
summarization and machine translation. Last but not least, the
presented approach contributes to comparative language studies and may
be further extended to other word classes that show relatively regular
morphological properties and syntactic dependencies.

\newpage

\section*{Abbreviations}


\begin{table}
\begin{tabular}{ll}
\lsptoprule
Full form  & Abbreviation \\
\midrule
Adjective & Adj\\
Adverb & Adv\\
Affiliation & Aff\\
Definite article & DefArt\\
Definite adjective & DefAdj\\
Definite possessive pronoun & DefPossPron\\
Definite trigger & DefTrigger\\
(Domain) specification & (Dom)Spec\\
Genitive determiner & GenDet\\
Indefinite trigger & IndefTrigger\\
Interrogative clitic & QuCL\\
Instrumental & Ins\\
Literal translation & Lit.\\
Location & Loc\\
Multiword expression & MWE\\
Multiword location name & MWLocN\\
Multiword organization name & MWOrgN\\
Named entity & NE\\
Named entity recognition & NER \\
(Noun) Phrase in genitive & GenP\\
Personal name & PerN\\
Possessive adjective & PossAdj\\
(Possessive) pronoun clitic & (Poss)CL\\
Possessive pronoun & PossPron\\
Prepositional phrase & PP\\
\lspbottomrule
\end{tabular}
\caption{Abbreviations.}
\end{table}

%\section*{References}

\printbibliography[heading=subbibliography,notkeyword=this]
%\nocite{*}


\end{document}
